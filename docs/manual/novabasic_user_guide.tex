% =============================================================================
% NovaBASIC v1.0 User Guide — Master Document
% For the e6502 Virtual Computer
% =============================================================================
\documentclass[11pt,a4paper]{book}

% ---------------------------------------------------------------------------
% Encoding and fonts
% ---------------------------------------------------------------------------
\usepackage[utf8]{inputenc}
\usepackage[T1]{fontenc}
\usepackage{lmodern}
\usepackage{microtype}

% ---------------------------------------------------------------------------
% Page geometry
% ---------------------------------------------------------------------------
\usepackage[
  a4paper,
  top=2.5cm,
  bottom=2.5cm,
  left=3cm,
  right=2.5cm,
  headheight=24pt
]{geometry}

% ---------------------------------------------------------------------------
% Color definitions — retro but polished palette
% ---------------------------------------------------------------------------
\usepackage{xcolor}

\definecolor{retroblue}  {RGB}{28,  63,  168}   % primary heading color
\definecolor{retroink}   {RGB}{29,  29,  29}    % body text
\definecolor{retrocream} {RGB}{246, 242, 232}   % page background
\definecolor{retropale}  {RGB}{233, 238, 249}   % code block background
\definecolor{retrogray}  {RGB}{107, 114, 128}   % muted text
\definecolor{retrogreen} {RGB}{34,  139, 34}    % tips
\definecolor{retroorange}{RGB}{202, 139, 66}    % warnings
\definecolor{retrocyan}  {RGB}{0,   139, 139}   % notes

% Page-level color
\pagecolor{retrocream}
\color{retroink}

% ---------------------------------------------------------------------------
% TikZ
% ---------------------------------------------------------------------------
\usepackage{tikz}
\usetikzlibrary{calc,shapes,positioning}

% ---------------------------------------------------------------------------
% Code listings — NovaBASIC language definition
% ---------------------------------------------------------------------------
\usepackage{listings}

\lstdefinelanguage{NovaBASIC}{%
  morekeywords=[1]{%
    END, FOR, NEXT, DATA, INPUT, DIM, READ, LET, GOTO, RUN, IF, RESTORE,
    GOSUB, RETURN, REM, STOP, ON, PRINT, CONT, LIST, NEW, THEN, ELSE,
    TO, STEP, NOT, AND, OR, LOAD, SAVE, DEF, POKE, DOKE, DO, LOOP,
    CLEAR, GET, SWAP, CLS, COLOR, LOCATE, PLOT, UNPLOT, LINE, CIRCLE,
    RECT, FILL, PAINT, MODE, GCLS, GCOLOR, SPRITE, SPRITEDATA, COPPER, SOUND,
    VOLUME, INSTRUMENT, VSYNC, DIR, DEL, XMEM, XBANK, XPOKE,
    STASH, FETCH, XFREE, XRESET, XALLOC, XDIR, XDEL, XMAP, XUNMAP,
    DEC, INC, CALL, WIDTH, BITSET, BITCLR, IRQ, NMI, UNTIL, WHILE,
    OFF, SPRITESHAPE, SPRITECOLOR, MUSIC, PLAY, TEMPO, PRIORITY,
    SIDPLAY, SIDSTOP, GSAVE, GLOAD, RETIRQ, RETNMI%
  },
  morekeywords=[2]{%
    SGN, INT, ABS, USR, FRE, POS, SQR, RND, LOG, EXP, COS, SIN, TAN,
    ATN, PEEK, DEEK, SADD, LEN, STR\$, VAL, ASC, UCASE\$, LCASE\$,
    CHR\$, HEX\$, BIN\$, BITTST, MAX, MIN, PI, TWOPI, VARPTR,
    LEFT\$, RIGHT\$, MID\$, SPRITEX, SPRITEY, COLLISION, BUMPED,
    XPEEK, TAB, SPC, FN, PLAYING, MNOTE%
  },
  morecomment=[l]{REM},
  sensitive=false,
  morestring=[b]"%
}

\lstdefinestyle{basiclisting}{%
  language=NovaBASIC,
  basicstyle=\ttfamily\small,
  backgroundcolor=\color{retropale},
  frame=single,
  framerule=0pt,
  framesep=4pt,
  xleftmargin=6pt,
  xrightmargin=6pt,
  keywordstyle=[1]\color{retroblue}\bfseries,
  keywordstyle=[2]\color{retrocyan},
  commentstyle=\color{retrogray}\itshape,
  stringstyle=\color{retrogreen},
  showstringspaces=false,
  tabsize=2,
  breaklines=true,
  numbers=left,
  numberstyle=\tiny\color{retrogray},
  numbersep=8pt,
  captionpos=b%
}

\lstdefinestyle{shell}{%
  basicstyle=\ttfamily\small,
  backgroundcolor=\color{retropale!40!retrocream},
  frame=none,
  showstringspaces=false,
  breaklines=true,
  keywordstyle=\color{retrogray}%
}

\lstset{style=basiclisting}

% ---------------------------------------------------------------------------
% tcolorbox environments
% ---------------------------------------------------------------------------
\usepackage[skins,breakable]{tcolorbox}

% Note box — cyan theme
\newtcolorbox{notebox}[1][]{%
  colback=retrocyan!8!retrocream,
  colframe=retrocyan!70!black,
  coltitle=retrocyan!70!black,
  fonttitle=\bfseries\small,
  title=Note,
  boxrule=0.8pt,
  arc=2pt,
  left=6pt, right=6pt, top=4pt, bottom=4pt,
  breakable,
  #1%
}

% Warning box — orange theme
\newtcolorbox{warningbox}[1][]{%
  colback=retroorange!12!retrocream,
  colframe=retroorange,
  coltitle=retroorange!80!black,
  fonttitle=\bfseries\small,
  title=Warning,
  boxrule=0.8pt,
  arc=2pt,
  left=6pt, right=6pt, top=4pt, bottom=4pt,
  breakable,
  #1%
}

% Tip box — green theme
\newtcolorbox{tipbox}[1][]{%
  colback=retrogreen!10!retrocream,
  colframe=retrogreen!70!black,
  coltitle=retrogreen!70!black,
  fonttitle=\bfseries\small,
  title=Tip,
  boxrule=0.8pt,
  arc=2pt,
  left=6pt, right=6pt, top=4pt, bottom=4pt,
  breakable,
  #1%
}

% Try It Now box — retroblue theme with keyboard icon
\newtcolorbox{tryitbox}[1][]{%
  colback=retropale,
  colframe=retroblue,
  coltitle=white,
  fonttitle=\bfseries\small,
  title={$\triangleright$\ \ Try It Now},
  boxrule=0.8pt,
  arc=2pt,
  left=6pt, right=6pt, top=4pt, bottom=4pt,
  attach boxed title to top left={yshift=-2mm, xshift=4mm},
  boxed title style={colback=retroblue, arc=2pt, boxrule=0pt},
  breakable,
  #1%
}

% Retrobox — legacy compatibility, sharp corners
\newtcolorbox{retrobox}[1][]{%
  colback=retropale,
  colframe=retroblue,
  boxrule=0.8pt,
  arc=0pt,
  left=6pt, right=6pt, top=6pt, bottom=6pt,
  breakable,
  #1%
}

% ---------------------------------------------------------------------------
% Additional packages
% ---------------------------------------------------------------------------
\usepackage{fancyhdr}
\usepackage{titlesec}
\usepackage{titletoc}
\usepackage{longtable}
\usepackage{array}
\usepackage{booktabs}
\usepackage{epigraph}

% Epigraph layout
\setlength{\epigraphwidth}{0.60\textwidth}
\renewcommand{\epigraphflush}{flushright}
\renewcommand{\epigraphrule}{0pt}

% ---------------------------------------------------------------------------
% Chapter and section title formatting
% ---------------------------------------------------------------------------
% Chapter heading: display shape with careful font-scoping.
% The label uses a \normalsize reset before the giant number so the font
% change is contained; the format argument re-establishes the title font.
\titleformat{\chapter}[display]
  {\normalfont\Huge\bfseries\color{retroblue}}
  {%
    \normalsize\raggedleft
    {\large\scshape\color{retrogray}\chaptername\quad}%
    {\fontsize{56}{56}\selectfont\color{retroblue!20!retrocream}\thechapter}%
  }
  {0.3ex}
  {%
    {\normalfont\color{retroblue}\rule{\linewidth}{1.2pt}}\\[0.4ex]%
    \raggedright\Huge\bfseries\color{retroblue}%
  }
  [\vspace{0.3ex}{\color{retroblue!35!retrocream}\rule{\linewidth}{0.4pt}}]

\titlespacing*{\chapter}{0pt}{0pt}{24pt}

% Unnumbered chapters (\chapter*)
\titleformat{name=\chapter,numberless}[display]
  {\normalfont\Huge\bfseries\color{retroblue}}
  {}
  {0pt}
  {%
    {\normalfont\color{retroblue}\rule{\linewidth}{1.2pt}}\\[0.4ex]%
    \raggedright\Huge\bfseries\color{retroblue}%
  }
  [\vspace{0.3ex}{\color{retroblue!35!retrocream}\rule{\linewidth}{0.4pt}}]

\titleformat{\section}
  {\normalfont\large\bfseries\color{retroblue}}
  {\thesection}
  {0.7em}
  {}

\titleformat{\subsection}
  {\normalfont\normalsize\bfseries\color{retroblue!80!retroink}}
  {\thesubsection}
  {0.6em}
  {}

\titleformat{\subsubsection}
  {\normalfont\normalsize\itshape\color{retroink}}
  {\thesubsubsection}
  {0.5em}
  {}

% ---------------------------------------------------------------------------
% Page headers and footers
% ---------------------------------------------------------------------------
\pagestyle{fancy}
\fancyhf{}
\fancyhead[LE]{\small\color{retrogray}\leftmark}
\fancyhead[RO]{\small\color{retrogray}\rightmark}
\fancyfoot[C]{\small\color{retrogray}\thepage}
\renewcommand{\headrulewidth}{0.4pt}
\renewcommand{\headrule}{%
  \hbox to\headwidth{\color{retrogray!50}\leaders\hrule height\headrulewidth\hfill}%
}
\renewcommand{\footrulewidth}{0pt}

% Plain style for chapter-opening pages
\fancypagestyle{plain}{%
  \fancyhf{}%
  \fancyfoot[C]{\small\color{retrogray}\thepage}%
  \renewcommand{\headrulewidth}{0pt}%
}

% ---------------------------------------------------------------------------
% Hyperref — must be loaded near last
% ---------------------------------------------------------------------------
\usepackage[
  colorlinks=true,
  linkcolor=retroblue,
  urlcolor=retroblue,
  citecolor=retroblue,
  plainpages=false,
  pdfpagelabels=true,
  pdftitle={NovaBASIC v1.0 User Guide},
  pdfauthor={Barry Walker},
  pdfsubject={For the e6502 Virtual Computer},
  pdfkeywords={NovaBASIC, e6502, BASIC, 6502, retro computing}
]{hyperref}

% ---------------------------------------------------------------------------
% Inline code macro
% ---------------------------------------------------------------------------
\newcommand{\cmd}[1]{%
  \colorbox{retropale}{\texttt{\color{retroink}#1}}%
}

% ---------------------------------------------------------------------------
% Foreword — content lives in chapters/00-foreword.tex
% ---------------------------------------------------------------------------

% =============================================================================
% Document body
% =============================================================================
\begin{document}

% ---------------------------------------------------------------------------
% Front matter
% ---------------------------------------------------------------------------
\frontmatter

% =============================================================================
% NovaBASIC v1.0 — Cover Page
% TikZ design inspired by 1980s computer manual aesthetics
% Emits content directly; % =============================================================================
% NovaBASIC v1.0 — Cover Page
% TikZ design inspired by 1980s computer manual aesthetics
% Emits content directly; % =============================================================================
% NovaBASIC v1.0 — Cover Page
% TikZ design inspired by 1980s computer manual aesthetics
% Emits content directly; \input{cover} in the master document
% =============================================================================
\begin{titlepage}
\begin{tikzpicture}[remember picture, overlay]

  % ------------------------------------------------------------------
  % Full-page background fill
  % ------------------------------------------------------------------
  \fill[retroblue]
    (current page.south west) rectangle (current page.north east);

  % ------------------------------------------------------------------
  % Diagonal accent stripe — lighter blue, angled band left side
  % ------------------------------------------------------------------
  \fill[retroblue!70!white, opacity=0.35]
    ([xshift=-60pt, yshift=120pt]current page.south west)
    -- ([xshift=260pt, yshift=120pt]current page.south west)
    -- ([xshift=260pt]current page.north west)
    -- ([xshift=-60pt]current page.north west)
    -- cycle;

  % Secondary lighter stripe
  \fill[retroblue!50!white, opacity=0.18]
    ([xshift=180pt, yshift=120pt]current page.south west)
    -- ([xshift=340pt, yshift=120pt]current page.south west)
    -- ([xshift=340pt]current page.north west)
    -- ([xshift=180pt]current page.north west)
    -- cycle;

  % ------------------------------------------------------------------
  % Subtle dot grid — top-right corner, very low opacity
  % ------------------------------------------------------------------
  \begin{scope}[opacity=0.06]
    \foreach \gx in {0,12,...,180}{
      \foreach \gy in {0,12,...,220}{
        \fill[white]
          ([xshift=\gx pt, yshift=-\gy pt]current page.north east)
          circle (1.2pt);
      }
    }
  \end{scope}

  % ------------------------------------------------------------------
  % Small accent squares — top-left region below header space
  % ------------------------------------------------------------------
  \fill[retrocyan, opacity=0.75]
    ([xshift=52pt, yshift=-52pt]current page.north west)
    rectangle ++(18pt, 18pt);

  \fill[retrocyan, opacity=0.50]
    ([xshift=80pt, yshift=-52pt]current page.north west)
    rectangle ++(10pt, 18pt);

  \fill[white, opacity=0.30]
    ([xshift=100pt, yshift=-52pt]current page.north west)
    rectangle ++(5pt, 18pt);

  % ------------------------------------------------------------------
  % Decorative BASIC code snippet — top-right, very low opacity
  % ------------------------------------------------------------------
  \node[
    anchor=north east,
    xshift=-44pt,
    yshift=-62pt,
    text=white,
    opacity=0.12,
    font=\ttfamily\large,
    align=left
  ] at (current page.north east) {%
    10 PRINT "HELLO"\\
    20 GOTO 10\\
    RUN%
  };

  % ------------------------------------------------------------------
  % Main title: NOVABASIC
  % ------------------------------------------------------------------
  \node[
    anchor=west,
    xshift=52pt,
    yshift=60pt,
    text=white,
    font=\fontsize{62}{66}\selectfont\bfseries\sffamily
  ] at (current page.west) {NOVABASIC};

  % ------------------------------------------------------------------
  % Subtitle: User Guide v1.0
  % ------------------------------------------------------------------
  \node[
    anchor=west,
    xshift=52pt,
    yshift=18pt,
    text=white,
    font=\fontsize{24}{28}\selectfont\sffamily
  ] at (current page.west) {User Guide\enspace v1.0};

  % ------------------------------------------------------------------
  % Tagline — retrocyan, italic
  % ------------------------------------------------------------------
  \node[
    anchor=west,
    xshift=52pt,
    yshift=-16pt,
    text=retrocyan,
    font=\fontsize{14}{18}\selectfont\itshape\sffamily
  ] at (current page.west) {For the e6502 Virtual Computer};

  % ------------------------------------------------------------------
  % Version / edition info — low opacity
  % ------------------------------------------------------------------
  \node[
    anchor=west,
    xshift=52pt,
    yshift=-46pt,
    text=white,
    opacity=0.55,
    font=\fontsize{10}{14}\selectfont\sffamily
  ] at (current page.west)
    {Derived from EhBASIC 2.22p5\enspace|\enspace 2026 Edition};

  % ------------------------------------------------------------------
  % Bottom horizontal rule
  % ------------------------------------------------------------------
  \draw[white, opacity=0.45, line width=1.0pt]
    ([xshift=52pt,  yshift=62pt]current page.south west)
    -- ([xshift=-52pt, yshift=62pt]current page.south east);

  % ------------------------------------------------------------------
  % Bottom feature line
  % ------------------------------------------------------------------
  \node[
    anchor=south,
    yshift=22pt,
    text=white,
    opacity=0.80,
    font=\fontsize{9}{12}\selectfont\sffamily\bfseries
  ] at (current page.south) {%
    MOS 6502\enspace|\enspace
    320{\texttimes}200 Graphics\enspace|\enspace
    6-Voice Sound\enspace|\enspace
    512KB Expansion%
  };

\end{tikzpicture}
\end{titlepage}
 in the master document
% =============================================================================
\begin{titlepage}
\begin{tikzpicture}[remember picture, overlay]

  % ------------------------------------------------------------------
  % Full-page background fill
  % ------------------------------------------------------------------
  \fill[retroblue]
    (current page.south west) rectangle (current page.north east);

  % ------------------------------------------------------------------
  % Diagonal accent stripe — lighter blue, angled band left side
  % ------------------------------------------------------------------
  \fill[retroblue!70!white, opacity=0.35]
    ([xshift=-60pt, yshift=120pt]current page.south west)
    -- ([xshift=260pt, yshift=120pt]current page.south west)
    -- ([xshift=260pt]current page.north west)
    -- ([xshift=-60pt]current page.north west)
    -- cycle;

  % Secondary lighter stripe
  \fill[retroblue!50!white, opacity=0.18]
    ([xshift=180pt, yshift=120pt]current page.south west)
    -- ([xshift=340pt, yshift=120pt]current page.south west)
    -- ([xshift=340pt]current page.north west)
    -- ([xshift=180pt]current page.north west)
    -- cycle;

  % ------------------------------------------------------------------
  % Subtle dot grid — top-right corner, very low opacity
  % ------------------------------------------------------------------
  \begin{scope}[opacity=0.06]
    \foreach \gx in {0,12,...,180}{
      \foreach \gy in {0,12,...,220}{
        \fill[white]
          ([xshift=\gx pt, yshift=-\gy pt]current page.north east)
          circle (1.2pt);
      }
    }
  \end{scope}

  % ------------------------------------------------------------------
  % Small accent squares — top-left region below header space
  % ------------------------------------------------------------------
  \fill[retrocyan, opacity=0.75]
    ([xshift=52pt, yshift=-52pt]current page.north west)
    rectangle ++(18pt, 18pt);

  \fill[retrocyan, opacity=0.50]
    ([xshift=80pt, yshift=-52pt]current page.north west)
    rectangle ++(10pt, 18pt);

  \fill[white, opacity=0.30]
    ([xshift=100pt, yshift=-52pt]current page.north west)
    rectangle ++(5pt, 18pt);

  % ------------------------------------------------------------------
  % Decorative BASIC code snippet — top-right, very low opacity
  % ------------------------------------------------------------------
  \node[
    anchor=north east,
    xshift=-44pt,
    yshift=-62pt,
    text=white,
    opacity=0.12,
    font=\ttfamily\large,
    align=left
  ] at (current page.north east) {%
    10 PRINT "HELLO"\\
    20 GOTO 10\\
    RUN%
  };

  % ------------------------------------------------------------------
  % Main title: NOVABASIC
  % ------------------------------------------------------------------
  \node[
    anchor=west,
    xshift=52pt,
    yshift=60pt,
    text=white,
    font=\fontsize{62}{66}\selectfont\bfseries\sffamily
  ] at (current page.west) {NOVABASIC};

  % ------------------------------------------------------------------
  % Subtitle: User Guide v1.0
  % ------------------------------------------------------------------
  \node[
    anchor=west,
    xshift=52pt,
    yshift=18pt,
    text=white,
    font=\fontsize{24}{28}\selectfont\sffamily
  ] at (current page.west) {User Guide\enspace v1.0};

  % ------------------------------------------------------------------
  % Tagline — retrocyan, italic
  % ------------------------------------------------------------------
  \node[
    anchor=west,
    xshift=52pt,
    yshift=-16pt,
    text=retrocyan,
    font=\fontsize{14}{18}\selectfont\itshape\sffamily
  ] at (current page.west) {For the e6502 Virtual Computer};

  % ------------------------------------------------------------------
  % Version / edition info — low opacity
  % ------------------------------------------------------------------
  \node[
    anchor=west,
    xshift=52pt,
    yshift=-46pt,
    text=white,
    opacity=0.55,
    font=\fontsize{10}{14}\selectfont\sffamily
  ] at (current page.west)
    {Derived from EhBASIC 2.22p5\enspace|\enspace 2026 Edition};

  % ------------------------------------------------------------------
  % Bottom horizontal rule
  % ------------------------------------------------------------------
  \draw[white, opacity=0.45, line width=1.0pt]
    ([xshift=52pt,  yshift=62pt]current page.south west)
    -- ([xshift=-52pt, yshift=62pt]current page.south east);

  % ------------------------------------------------------------------
  % Bottom feature line
  % ------------------------------------------------------------------
  \node[
    anchor=south,
    yshift=22pt,
    text=white,
    opacity=0.80,
    font=\fontsize{9}{12}\selectfont\sffamily\bfseries
  ] at (current page.south) {%
    MOS 6502\enspace|\enspace
    320{\texttimes}200 Graphics\enspace|\enspace
    6-Voice Sound\enspace|\enspace
    512KB Expansion%
  };

\end{tikzpicture}
\end{titlepage}
 in the master document
% =============================================================================
\begin{titlepage}
\begin{tikzpicture}[remember picture, overlay]

  % ------------------------------------------------------------------
  % Full-page background fill
  % ------------------------------------------------------------------
  \fill[retroblue]
    (current page.south west) rectangle (current page.north east);

  % ------------------------------------------------------------------
  % Diagonal accent stripe — lighter blue, angled band left side
  % ------------------------------------------------------------------
  \fill[retroblue!70!white, opacity=0.35]
    ([xshift=-60pt, yshift=120pt]current page.south west)
    -- ([xshift=260pt, yshift=120pt]current page.south west)
    -- ([xshift=260pt]current page.north west)
    -- ([xshift=-60pt]current page.north west)
    -- cycle;

  % Secondary lighter stripe
  \fill[retroblue!50!white, opacity=0.18]
    ([xshift=180pt, yshift=120pt]current page.south west)
    -- ([xshift=340pt, yshift=120pt]current page.south west)
    -- ([xshift=340pt]current page.north west)
    -- ([xshift=180pt]current page.north west)
    -- cycle;

  % ------------------------------------------------------------------
  % Subtle dot grid — top-right corner, very low opacity
  % ------------------------------------------------------------------
  \begin{scope}[opacity=0.06]
    \foreach \gx in {0,12,...,180}{
      \foreach \gy in {0,12,...,220}{
        \fill[white]
          ([xshift=\gx pt, yshift=-\gy pt]current page.north east)
          circle (1.2pt);
      }
    }
  \end{scope}

  % ------------------------------------------------------------------
  % Small accent squares — top-left region below header space
  % ------------------------------------------------------------------
  \fill[retrocyan, opacity=0.75]
    ([xshift=52pt, yshift=-52pt]current page.north west)
    rectangle ++(18pt, 18pt);

  \fill[retrocyan, opacity=0.50]
    ([xshift=80pt, yshift=-52pt]current page.north west)
    rectangle ++(10pt, 18pt);

  \fill[white, opacity=0.30]
    ([xshift=100pt, yshift=-52pt]current page.north west)
    rectangle ++(5pt, 18pt);

  % ------------------------------------------------------------------
  % Decorative BASIC code snippet — top-right, very low opacity
  % ------------------------------------------------------------------
  \node[
    anchor=north east,
    xshift=-44pt,
    yshift=-62pt,
    text=white,
    opacity=0.12,
    font=\ttfamily\large,
    align=left
  ] at (current page.north east) {%
    10 PRINT "HELLO"\\
    20 GOTO 10\\
    RUN%
  };

  % ------------------------------------------------------------------
  % Main title: NOVABASIC
  % ------------------------------------------------------------------
  \node[
    anchor=west,
    xshift=52pt,
    yshift=60pt,
    text=white,
    font=\fontsize{62}{66}\selectfont\bfseries\sffamily
  ] at (current page.west) {NOVABASIC};

  % ------------------------------------------------------------------
  % Subtitle: User Guide v1.0
  % ------------------------------------------------------------------
  \node[
    anchor=west,
    xshift=52pt,
    yshift=18pt,
    text=white,
    font=\fontsize{24}{28}\selectfont\sffamily
  ] at (current page.west) {User Guide\enspace v1.0};

  % ------------------------------------------------------------------
  % Tagline — retrocyan, italic
  % ------------------------------------------------------------------
  \node[
    anchor=west,
    xshift=52pt,
    yshift=-16pt,
    text=retrocyan,
    font=\fontsize{14}{18}\selectfont\itshape\sffamily
  ] at (current page.west) {For the e6502 Virtual Computer};

  % ------------------------------------------------------------------
  % Version / edition info — low opacity
  % ------------------------------------------------------------------
  \node[
    anchor=west,
    xshift=52pt,
    yshift=-46pt,
    text=white,
    opacity=0.55,
    font=\fontsize{10}{14}\selectfont\sffamily
  ] at (current page.west)
    {Derived from EhBASIC 2.22p5\enspace|\enspace 2026 Edition};

  % ------------------------------------------------------------------
  % Bottom horizontal rule
  % ------------------------------------------------------------------
  \draw[white, opacity=0.45, line width=1.0pt]
    ([xshift=52pt,  yshift=62pt]current page.south west)
    -- ([xshift=-52pt, yshift=62pt]current page.south east);

  % ------------------------------------------------------------------
  % Bottom feature line
  % ------------------------------------------------------------------
  \node[
    anchor=south,
    yshift=22pt,
    text=white,
    opacity=0.80,
    font=\fontsize{9}{12}\selectfont\sffamily\bfseries
  ] at (current page.south) {%
    MOS 6502\enspace|\enspace
    320{\texttimes}200 Graphics\enspace|\enspace
    6-Voice Sound\enspace|\enspace
    512KB Expansion%
  };

\end{tikzpicture}
\end{titlepage}

\thispagestyle{empty}
\cleardoublepage

% =============================================================================
% NovaBASIC v1.0 — Copyright Page
% Emits content directly; % =============================================================================
% NovaBASIC v1.0 — Copyright Page
% Emits content directly; % =============================================================================
% NovaBASIC v1.0 — Copyright Page
% Emits content directly; \input{copyright} in the master document
% =============================================================================
\thispagestyle{empty}
\vspace*{\fill}
\begin{flushleft}

{\large\bfseries NovaBASIC v1.0 User Guide}\\[0.4em]
{\normalsize For the e6502 Virtual Computer --- 2026 Edition}

\bigskip
\noindent\rule{\textwidth}{0.4pt}
\bigskip

{\small
Copyright \textcopyright\ 2026 Barry Walker. All Rights Reserved.

\medskip
No part of this publication may be reproduced, distributed, or transmitted
in any form or by any means, including photocopying, recording, or other
electronic or mechanical methods, without the prior written permission of
the author, except in the case of brief quotations embodied in critical
reviews and certain other non-commercial uses permitted by copyright law.

\medskip
e6502 is an open source project.

\medskip
NovaBASIC is derived from Enhanced BASIC (EhBASIC) by Lee Davison.
EhBASIC is copyright \textcopyright\ Lee Davison and is used here under
the terms of its original free-for-non-commercial-use licence.
All extensions and modifications are the work of Barry Walker.

\medskip
MOS 6502 is a trademark of MOS Technology, Inc. All other trademarks
mentioned in this guide are the property of their respective owners.
Their use here is for identification purposes only.
}

\bigskip
\noindent\rule{\textwidth}{0.4pt}

\vspace{2em}
{\footnotesize\color{retrogray}
Typeset with \LaTeX\ and the \texttt{lmodern} font family.\\
Document class: \texttt{book}, 11\,pt, A4.\\
Build toolchain: \texttt{latexmk} with \texttt{pdflatex}.
}

\end{flushleft}
\vspace*{\fill}
 in the master document
% =============================================================================
\thispagestyle{empty}
\vspace*{\fill}
\begin{flushleft}

{\large\bfseries NovaBASIC v1.0 User Guide}\\[0.4em]
{\normalsize For the e6502 Virtual Computer --- 2026 Edition}

\bigskip
\noindent\rule{\textwidth}{0.4pt}
\bigskip

{\small
Copyright \textcopyright\ 2026 Barry Walker. All Rights Reserved.

\medskip
No part of this publication may be reproduced, distributed, or transmitted
in any form or by any means, including photocopying, recording, or other
electronic or mechanical methods, without the prior written permission of
the author, except in the case of brief quotations embodied in critical
reviews and certain other non-commercial uses permitted by copyright law.

\medskip
e6502 is an open source project.

\medskip
NovaBASIC is derived from Enhanced BASIC (EhBASIC) by Lee Davison.
EhBASIC is copyright \textcopyright\ Lee Davison and is used here under
the terms of its original free-for-non-commercial-use licence.
All extensions and modifications are the work of Barry Walker.

\medskip
MOS 6502 is a trademark of MOS Technology, Inc. All other trademarks
mentioned in this guide are the property of their respective owners.
Their use here is for identification purposes only.
}

\bigskip
\noindent\rule{\textwidth}{0.4pt}

\vspace{2em}
{\footnotesize\color{retrogray}
Typeset with \LaTeX\ and the \texttt{lmodern} font family.\\
Document class: \texttt{book}, 11\,pt, A4.\\
Build toolchain: \texttt{latexmk} with \texttt{pdflatex}.
}

\end{flushleft}
\vspace*{\fill}
 in the master document
% =============================================================================
\thispagestyle{empty}
\vspace*{\fill}
\begin{flushleft}

{\large\bfseries NovaBASIC v1.0 User Guide}\\[0.4em]
{\normalsize For the e6502 Virtual Computer --- 2026 Edition}

\bigskip
\noindent\rule{\textwidth}{0.4pt}
\bigskip

{\small
Copyright \textcopyright\ 2026 Barry Walker. All Rights Reserved.

\medskip
No part of this publication may be reproduced, distributed, or transmitted
in any form or by any means, including photocopying, recording, or other
electronic or mechanical methods, without the prior written permission of
the author, except in the case of brief quotations embodied in critical
reviews and certain other non-commercial uses permitted by copyright law.

\medskip
e6502 is an open source project.

\medskip
NovaBASIC is derived from Enhanced BASIC (EhBASIC) by Lee Davison.
EhBASIC is copyright \textcopyright\ Lee Davison and is used here under
the terms of its original free-for-non-commercial-use licence.
All extensions and modifications are the work of Barry Walker.

\medskip
MOS 6502 is a trademark of MOS Technology, Inc. All other trademarks
mentioned in this guide are the property of their respective owners.
Their use here is for identification purposes only.
}

\bigskip
\noindent\rule{\textwidth}{0.4pt}

\vspace{2em}
{\footnotesize\color{retrogray}
Typeset with \LaTeX\ and the \texttt{lmodern} font family.\\
Document class: \texttt{book}, 11\,pt, A4.\\
Build toolchain: \texttt{latexmk} with \texttt{pdflatex}.
}

\end{flushleft}
\vspace*{\fill}

\cleardoublepage

\tableofcontents
\cleardoublepage

% =============================================================================
% NovaBASIC v1.0 User Guide — Foreword
% =============================================================================
\chapter*{Foreword}
\addcontentsline{toc}{chapter}{Foreword}
\markboth{Foreword}{Foreword}

There is something quietly extraordinary about a chip that fits in the palm of
your hand and changed the world.  The MOS 6502, introduced in 1975, powered the
Apple~II, the Commodore~64, the Nintendo Entertainment System, the Atari~2600,
and the BBC Micro.  It put a computer on the desk of a generation that had never
seen one before.  It was cheap enough for a hobbyist to buy, simple enough to
understand completely, and fast enough to make things happen on screen that felt
genuinely alive.  Whole careers began on machines built around that little chip.
So did countless imaginations.

The years moved on.  Processors grew faster, wider, and vastly more complex.
But the 6502 never really went away --- it kept shipping in embedded systems,
and it kept living in the memories of the people who learned on it.  There is a
reason emulators, homebrew cartridges, and new assembler projects for the 6502
keep appearing in~2026: programming it is a complete, intimate experience.  You
can hold the whole machine in your head at once.

The e6502 virtual computer is not a museum exhibit.  It takes the honest
architecture of the original --- 64\,KB of address space, stack at
\texttt{\$0100}--\texttt{\$01FF}, clean interrupt model --- and pairs it with
hardware that no real 8-bit machine ever had: a 320$\times$200 pixel bitmap with
16 colors, sixteen hardware sprites each 16$\times$16 pixels and multicolor, a
four-voice synthesizer with ADSR envelopes and five waveforms, 512\,KB of banked
expansion memory, and file I/O.  It is the machine you always wished you had
back then.

NovaBASIC is the language that lives on it.  Built on Lee Davison's
\textit{Enhanced BASIC~2.22p5} --- itself a carefully crafted piece of work ---
NovaBASIC keeps everything that made classic BASIC approachable: line numbers,
\cmd{PRINT}, \cmd{FOR}/\cmd{NEXT}, \cmd{GOSUB}, the immediacy of typing a line
and seeing it run.  Then it adds the commands to drive all that new hardware,
without hiding the machine from you.

Whether you are learning to program for the first time, building a retro game,
exploring procedural graphics, or just looking for a creative sandbox that
rewards curiosity --- this is the right place.  Boot it up.  Type something.
Watch it run.

\medskip
\noindent Welcome.

\bigskip
{\raggedleft\color{retrogray}\itshape February 2026\par}

\cleardoublepage

% ---------------------------------------------------------------------------
% Main matter
% ---------------------------------------------------------------------------
\mainmatter

% =============================================================================
% NovaBASIC v1.0 User Guide — Chapter 1: Welcome to NovaBASIC
% =============================================================================
\chapter{Welcome to NovaBASIC}

\epigraph{%
  Every great journey starts with a single command.
}{%
  \textit{Unknown programmer, circa~1982}%
}

\section{What NovaBASIC Is}

NovaBASIC v1.0 is a modernized 6502 BASIC interpreter for the e6502 virtual
computer.  It is derived from Lee Davison's \textit{Enhanced BASIC~2.22p5}
(EhBASIC), one of the cleanest and most complete open-source 6502 BASIC
implementations ever written.  NovaBASIC keeps the entire EhBASIC core --- every
numeric function, every string operation, the full \cmd{DO}/\cmd{LOOP} and
\cmd{FOR}/\cmd{NEXT} machinery --- and adds a hardware command layer that lets
you drive the e6502's graphics, sound, file system, and expansion memory
directly from BASIC.

The programming model is classic line-numbered BASIC.  You type a line with a
number at the front, press Return, and it is stored in the program.  You type
\cmd{RUN} and it executes from the lowest line number.  Simple, immediate, and
still deeply satisfying.

When NovaBASIC boots, the screen clears to a blue background with white text and
you see:

\begin{retrobox}
\texttt{NovaBASIC v1.0}\\
\texttt{Derived from EhBASIC 2.22p5}\\[0.4ex]
\texttt{xxxxx BASIC bytes free}
\end{retrobox}

\noindent
The number shown is the amount of free BASIC program memory.  The cursor waits
for your first command.

\begin{notebox}
  Boot sets the screen background to color~6 (blue) and the foreground text
  color to color~1 (white).  You can change these at any time with
  \cmd{COLOR~\textit{fg},\textit{bg}}.
\end{notebox}

% ---------------------------------------------------------------------------
\section{What You Can Do}

Here is what NovaBASIC puts at your fingertips:

\begin{itemize}
  \item Write and run classic line-numbered BASIC programs in the tradition
        of the Commodore~64, Apple~II, and BBC~Micro.
  \item Draw directly to a \textbf{320$\times$200 pixel bitmap} with
        \textbf{16 colors} using \cmd{PLOT}, \cmd{LINE}, \cmd{CIRCLE},
        \cmd{RECT}, \cmd{FILL}, and \cmd{PAINT}.
  \item Animate up to \textbf{16 hardware sprites}, each 16$\times$16 pixels
        and multicolor, with per-sprite shape, position, and flip control.
  \item Play \textbf{4-voice synthesized sound} with ADSR envelopes and five
        selectable waveforms --- square, sawtooth, triangle, sine, and noise.
  \item Save and load programs and data files with \cmd{SAVE}, \cmd{LOAD},
        \cmd{DIR}, and \cmd{DEL}.
  \item Access \textbf{512\,KB of banked expansion memory} for large data
        sets, tile maps, sprite sheets, or music sequences.
  \item Drop into \textbf{6502 assembly language} via \cmd{CALL} and
        \cmd{POKE}/\cmd{PEEK} for performance-critical inner loops.
\end{itemize}

% ---------------------------------------------------------------------------
\section{How to Read This Guide}

This manual is organized so each chapter builds on the previous one.  You do
not need to read it straight through --- jump to whichever section you need ---
but if you are new to NovaBASIC the order makes sense:

\begin{description}
  \item[Chapters~1--3] Getting started, the edit-run workflow, and the core
    language: variables, arrays, control flow, operators, and built-in
    functions.
  \item[Chapters~4--5] Graphics and sprites; sound and music.  These two
    chapters cover everything you need to build a real game or demo.
  \item[Chapters~6--7] Expansion memory and low-level access.  Read these
    when your projects outgrow the built-in 64\,KB address space or when
    you want to call hand-written assembly routines.
  \item[Appendices] Complete command reference, memory map, hardware
    register summary, error codes, and system limits.
\end{description}

Cross-references appear as ``see Section~X.Y'' or ``see Appendix~A.''  Every
command name is typeset in \cmd{this style} throughout the text.

% ---------------------------------------------------------------------------
\section{Your First Program}

Let us get something on screen right now.  Type each line exactly as shown,
pressing~Return after each one, then type \cmd{RUN} and press Return.

\begin{tryitbox}
\begin{lstlisting}[style=basiclisting,numbers=none]
10 PRINT "HELLO, NOVABASIC!"
20 FOR I=1 TO 5
30   PRINT "COUNT: ";I
40 NEXT I
RUN
\end{lstlisting}
\textbf{Expected output:}\\[0.4ex]
\texttt{HELLO, NOVABASIC!}\\
\texttt{COUNT: 1}\\
\texttt{COUNT: 2}\\
\texttt{COUNT: 3}\\
\texttt{COUNT: 4}\\
\texttt{COUNT: 5}
\end{tryitbox}

\noindent
Six lines of output, then the cursor returns to the \texttt{Ready} prompt.
Congratulations --- you have just run your first NovaBASIC program.

% =============================================================================
% NovaBASIC v1.0 User Guide — Chapter 2: Your First Session
% =============================================================================
\chapter{Your First Session}

\epigraph{%
  Tell me and I forget.  Teach me and I remember.  Involve me and I learn.
}{%
  \textit{Benjamin Franklin}%
}

The fastest way to get comfortable with NovaBASIC is to use it.  This chapter
walks you through everything you need for a productive first session: entering
programs, editing them, saving and loading files, and building the habits that
will serve you on every project after this one.

% ---------------------------------------------------------------------------
\section{The Edit-Run Cycle}

NovaBASIC stores your program as an ordered list of numbered lines.  When you
type a line beginning with a number and press Return, that line is added to
the program in the correct position.  When you type \cmd{RUN}, execution begins
at the lowest line number and proceeds in order.

This is the fundamental loop:

\begin{retrobox}
\centering
\textbf{TYPE A LINE} $\longrightarrow$ \textbf{LIST} $\longrightarrow$
\textbf{RUN} $\longrightarrow$ \textbf{FIX} $\longrightarrow$ \textbf{RUN}
\end{retrobox}

\noindent
Here is a small program to enter right now:

\begin{lstlisting}[style=basiclisting]
10 PRINT "NOVABASIC IS READY"
20 X = 7
30 PRINT "SEVEN SQUARED IS "; X*X
\end{lstlisting}

Type \cmd{LIST} to see the program back.  Type \cmd{RUN} to execute it.  Now
change line~20 by retyping it with a new value:

\begin{lstlisting}[style=basiclisting,numbers=none]
20 X = 12
\end{lstlisting}

\noindent
The old line~20 is replaced.  Type \cmd{RUN} again and the result changes.
To delete a line entirely, type its number alone and press Return:

\begin{lstlisting}[style=basiclisting,numbers=none]
30
\end{lstlisting}

\noindent
Line~30 is gone.  \cmd{LIST} confirms it.

% ---------------------------------------------------------------------------
\section{Essential Editing Commands}

\begin{longtable}{>{\ttfamily}p{0.28\textwidth} p{0.64\textwidth}}
\toprule
\normalfont\textbf{Command} & \textbf{What It Does} \\
\midrule
\endhead
\bottomrule
\endfoot
LIST           & Display the entire program currently in memory. \\[2pt]
LIST 20-40     & Display only lines 20 through 40 (inclusive). \\[2pt]
NEW            & Clear the program from memory.  Cannot be undone. \\[2pt]
RUN            & Execute the program starting from the lowest line number. \\[2pt]
CONT           & Continue execution after a \cmd{STOP} statement. \\[2pt]
\end{longtable}

\begin{warningbox}
  \cmd{NEW} erases everything in memory immediately.  Save your work with
  \cmd{SAVE} before typing \cmd{NEW} if you want to keep it.
\end{warningbox}

% ---------------------------------------------------------------------------
\section{Saving and Loading}

Programs are saved to and loaded from the virtual file system using four
commands:

\begin{description}
  \item[\cmd{SAVE "name"}]
    Write the current program to disk under the given name.  The
    \texttt{.bas} extension is added automatically.
  \item[\cmd{LOAD "name"}]
    Load a saved program from disk into memory, replacing anything
    currently there.
  \item[\cmd{DIR}]
    List all saved BASIC programs.
  \item[\cmd{DEL "name"}]
    Delete a saved program.  This cannot be undone.
\end{description}

\begin{notebox}
  Filenames may contain letters (\texttt{A--Z}, \texttt{a--z}), digits
  (\texttt{0--9}), underscores, hyphens, and dots.  Maximum length is
  63~characters.  Names are case-sensitive on disk.
\end{notebox}

Here is a complete save-and-reload sequence:

\begin{lstlisting}[style=basiclisting,numbers=none]
SAVE "MYPROG"
DIR
NEW
LOAD "MYPROG"
RUN
\end{lstlisting}

\noindent
\cmd{DIR} shows \texttt{MYPROG.bas} in the listing.  After \cmd{LOAD} and
\cmd{RUN}, the program executes as if you had just typed it in.

% ---------------------------------------------------------------------------
\section{Working Efficiently}

Good habits established early make everything easier later.

\begin{tipbox}
\begin{itemize}
  \item \textbf{Number lines by tens.}  Use line numbers 10, 20, 30\ldots
    rather than 1, 2, 3.  This leaves room to insert new lines between
    existing ones without renumbering.
  \item \textbf{Use \cmd{REM} freely while learning.}  A comment line
    costs a little memory but saves a lot of confusion.  You can remove
    them once the code is stable.
  \item \textbf{Build and run in small steps.}  Add a few lines, \cmd{RUN},
    verify the output, add more.  Chasing bugs through a hundred lines you
    typed without testing is no fun.
  \item \textbf{Group lines by function.}  A common convention: 100s for
    initialisation, 200s for input, 300s for game logic, 800s for output
    routines, 900s for cleanup and quit.  Any scheme that makes sense to
    you is fine; the important thing is to pick one and use it.
\end{itemize}
\end{tipbox}

% ---------------------------------------------------------------------------
\section{The Round-Trip Exercise}

The following exercise takes you through writing, saving, clearing, reloading,
and running a program.  Every step matters --- do not skip any of them.

\begin{tryitbox}
\begin{lstlisting}[style=basiclisting]
10 PRINT "ROUND TRIP COMPLETE"
\end{lstlisting}
Then, at the prompt (no line number):
\begin{lstlisting}[style=basiclisting,numbers=none]
SAVE "ROUNDTRIP"
NEW
DIR
LOAD "ROUNDTRIP"
RUN
\end{lstlisting}
\textbf{Expected:}
\begin{itemize}
  \item After \cmd{SAVE}: the prompt returns immediately with no error.
  \item After \cmd{NEW}: \cmd{LIST} shows nothing.
  \item After \cmd{DIR}: \texttt{ROUNDTRIP.bas} appears in the listing.
  \item After \cmd{LOAD} and \cmd{RUN}: the screen prints
        \texttt{ROUND TRIP COMPLETE}.
\end{itemize}
\end{tryitbox}

\noindent
If you see \texttt{ROUND TRIP COMPLETE} at the end, you have mastered the
complete NovaBASIC workflow.  Everything else is built on top of exactly this.

% =============================================================================
% NovaBASIC v1.0 User Guide — Chapter 3: Language Fundamentals
% =============================================================================
\chapter{Language Fundamentals}

\epigraph{%
  Simplicity is the ultimate sophistication.
}{%
  \textit{Leonardo da Vinci}%
}

NovaBASIC inherits a rich language from EhBASIC~2.22p5 and extends it with
hardware-specific commands.  This chapter covers the core language features
you will use in almost every program: variables, arrays, control flow,
operators, built-in functions, and direct memory access.

% ---------------------------------------------------------------------------
\section{Variables}

NovaBASIC has two kinds of variable: \textbf{numeric} and \textbf{string}.

\begin{description}
  \item[Numeric variables] hold floating-point numbers.  Names begin with a
    letter and may contain letters and digits (e.g.\ \texttt{A},
    \texttt{SCORE}, \texttt{X1}, \texttt{HITCOUNT}).
  \item[String variables] hold text.  Names end with a dollar sign
    (e.g.\ \texttt{N\$}, \texttt{NAME\$}, \texttt{MSG\$}).
\end{description}

Variable names are \textbf{case-insensitive}: \texttt{SCORE} and
\texttt{score} refer to the same variable.  Numeric variables default to
\texttt{0}; string variables default to the empty string.  You do not need to
declare a variable before using it.

\begin{lstlisting}[style=basiclisting]
10 SCORE = 0
20 NAME$ = "PLAYER ONE"
30 LIVES = 3
40 PRINT NAME$; " HAS "; LIVES; " LIVES"
50 SCORE = SCORE + 100
60 PRINT "SCORE: "; SCORE
\end{lstlisting}

% ---------------------------------------------------------------------------
\section{Arrays}

Use \cmd{DIM} to declare an array before using it.  The default lower bound
is~0, so \cmd{DIM A(10)} creates 11 elements: \texttt{A(0)} through
\texttt{A(10)}.

\begin{lstlisting}[style=basiclisting]
10 DIM A(10)
20 FOR I=0 TO 10
30   A(I) = I * I
40 NEXT I
50 FOR I=0 TO 10
60   PRINT "A(";I;") = ";A(I)
70 NEXT I
\end{lstlisting}

\noindent
Two-dimensional arrays work the same way:

\begin{lstlisting}[style=basiclisting]
10 DIM GRID(7,7)
20 FOR R=0 TO 7
30   FOR C=0 TO 7
40     GRID(R,C) = R*8 + C
50   NEXT C
60 NEXT R
\end{lstlisting}

\begin{notebox}
  String arrays are also supported: \cmd{DIM LABEL\$(9)} creates ten
  string slots.
\end{notebox}

% ---------------------------------------------------------------------------
\section{Control Flow}

\subsection{FOR / NEXT Loops}

The workhorse of NovaBASIC iteration.  The optional \cmd{STEP} clause sets
the increment; if omitted, it defaults to~1.

\begin{lstlisting}[style=basiclisting]
10 FOR I = 1 TO 10
20   PRINT I; " ";
30 NEXT I
40 PRINT
50 REM count down with STEP
60 FOR N = 10 TO 1 STEP -1
70   PRINT N; " ";
80 NEXT N
\end{lstlisting}

\subsection{IF / THEN / ELSE}

\begin{lstlisting}[style=basiclisting]
10 INPUT "ENTER A NUMBER: "; N
20 IF N > 100 THEN PRINT "BIG" ELSE PRINT "SMALL"
30 IF N = 42 THEN PRINT "THE ANSWER"
\end{lstlisting}

\noindent
\cmd{THEN} may be followed by a statement or a line number
(as a \cmd{GOTO} shorthand).  \cmd{ELSE} is optional.

\subsection{DO / LOOP}

\cmd{DO}/\cmd{LOOP} supports four variants for flexible looping:

\begin{lstlisting}[style=basiclisting]
10 X = 1
20 DO
30   PRINT X
40   X = X + 1
50 LOOP WHILE X <= 5

100 Y = 10
110 DO UNTIL Y = 0
120   PRINT Y
130   Y = Y - 1
140 LOOP
\end{lstlisting}

\subsection{GOSUB / RETURN}

Use subroutines to avoid repeating code.  \cmd{GOSUB} branches to a line
number and \cmd{RETURN} comes back:

\begin{lstlisting}[style=basiclisting]
10 GOSUB 1000
20 GOSUB 1000
30 END
1000 REM draw border subroutine
1010 PRINT "----------"
1020 RETURN
\end{lstlisting}

\subsection{GOTO}

\cmd{GOTO} unconditionally jumps to a line number.  Use it sparingly;
\cmd{GOSUB}/\cmd{RETURN} and structured loops are usually cleaner.

% ---------------------------------------------------------------------------
\section{Operators}

\begin{longtable}{>{\ttfamily}p{0.20\textwidth} p{0.22\textwidth} p{0.48\textwidth}}
\toprule
\normalfont\textbf{Operator} & \textbf{Type} & \textbf{Description} \\
\midrule
\endhead
\bottomrule
\endfoot
+  & Arithmetic & Addition \\[2pt]
-  & Arithmetic & Subtraction or unary negation \\[2pt]
*  & Arithmetic & Multiplication \\[2pt]
/  & Arithmetic & Division \\[2pt]
\^{}  & Arithmetic & Exponentiation (\texttt{2\^{}8} = 256) \\[4pt]
=  & Comparison & Equal to \\[2pt]
<  & Comparison & Less than \\[2pt]
>  & Comparison & Greater than \\[2pt]
<= & Comparison & Less than or equal to \\[2pt]
>= & Comparison & Greater than or equal to \\[2pt]
<> & Comparison & Not equal to \\[4pt]
AND & Logical   & Logical (bitwise) AND \\[2pt]
OR  & Logical   & Logical (bitwise) OR \\[2pt]
NOT & Logical   & Logical (bitwise) NOT \\[2pt]
EOR & Logical   & Exclusive OR \\[4pt]
<< & Bit        & Left shift (\texttt{LSHIFT}) \\[2pt]
>> & Bit        & Right shift (\texttt{RSHIFT}) \\[4pt]
BITSET & Bit    & Set a bit in an integer \\[2pt]
BITCLR & Bit    & Clear a bit in an integer \\[2pt]
BITTST(n,b) & Bit & Test bit \textit{b} of integer \textit{n} \\
\end{longtable}

% ---------------------------------------------------------------------------
\section{Built-in Functions}

\subsection{Numeric Functions}

\begin{longtable}{>{\ttfamily}p{0.28\textwidth} p{0.64\textwidth}}
\toprule
\normalfont\textbf{Function} & \textbf{Description} \\
\midrule
\endhead
\bottomrule
\endfoot
INT(x)   & Truncate to integer (towards zero) \\[2pt]
ABS(x)   & Absolute value \\[2pt]
SGN(x)   & Sign: $-1$, $0$, or $1$ \\[2pt]
SQR(x)   & Square root \\[2pt]
RND(1)   & Pseudo-random number in $[0, 1)$ \\[2pt]
LOG(x)   & Natural logarithm \\[2pt]
EXP(x)   & $e^x$ \\[2pt]
SIN(x)   & Sine (radians) \\[2pt]
COS(x)   & Cosine (radians) \\[2pt]
TAN(x)   & Tangent (radians) \\[2pt]
ATN(x)   & Arctangent (radians) \\[2pt]
PI       & The constant $\pi \approx 3.14159$ \\[2pt]
TWOPI    & The constant $2\pi \approx 6.28318$ \\[2pt]
MAX(a,b) & Larger of two values \\[2pt]
MIN(a,b) & Smaller of two values \\[2pt]
PEEK(addr) & Read a byte from memory address \textit{addr} \\[2pt]
DEEK(addr) & Read a 16-bit word from address \textit{addr} \\[2pt]
FRE(0)   & Free BASIC program memory in bytes \\[2pt]
POS(0)   & Current cursor column position \\
\end{longtable}

\subsection{String Functions}

\begin{longtable}{>{\ttfamily}p{0.34\textwidth} p{0.58\textwidth}}
\toprule
\normalfont\textbf{Function} & \textbf{Description} \\
\midrule
\endhead
\bottomrule
\endfoot
LEN(s\$)          & Number of characters in the string \\[2pt]
CHR\$(n)          & Character whose ASCII code is \textit{n} \\[2pt]
ASC(s\$)          & ASCII code of the first character \\[2pt]
STR\$(n)          & Convert number to string \\[2pt]
VAL(s\$)          & Convert string to number \\[2pt]
LEFT\$(s\$,n)     & Left \textit{n} characters \\[2pt]
RIGHT\$(s\$,n)    & Right \textit{n} characters \\[2pt]
MID\$(s\$,p,n)    & \textit{n} characters starting at position \textit{p} \\[2pt]
UCASE\$(s\$)      & Convert to uppercase \\[2pt]
LCASE\$(s\$)      & Convert to lowercase \\[2pt]
HEX\$(n)          & Hexadecimal string representation of \textit{n} \\[2pt]
BIN\$(n)          & Binary string representation of \textit{n} \\
\end{longtable}

% ---------------------------------------------------------------------------
\section{Memory Access}

NovaBASIC gives you direct read and write access to the 6502 address space.
This is useful for reading hardware registers, patching values, and interfacing
with assembly routines.

\begin{description}
  \item[\cmd{PEEK(addr)}] Read one byte (0--255) from address \textit{addr}.
  \item[\cmd{POKE addr, val}] Write one byte to address \textit{addr}.
  \item[\cmd{DEEK(addr)}] Read a 16-bit little-endian word from
    \textit{addr} and \textit{addr}+1.
  \item[\cmd{DOKE addr, val}] Write a 16-bit little-endian word.
  \item[\cmd{VARPTR(v)}] Return the memory address of variable \textit{v}.
    Useful when passing variable addresses to assembly routines.
\end{description}

\begin{lstlisting}[style=basiclisting]
10 REM read and print two bytes at $0300
20 PRINT PEEK(768); PEEK(769)
30 REM write a counter value
40 POKE 768, 42
50 PRINT PEEK(768)
\end{lstlisting}

\begin{warningbox}
  Writing to the wrong address can crash the virtual machine or corrupt the
  BASIC interpreter.  Know what you are writing to before you \cmd{POKE}
  into system areas.  See the memory map in Appendix~B for safe zones.
\end{warningbox}

% ---------------------------------------------------------------------------
\section{Style Tips}

\begin{tipbox}
\begin{itemize}
  \item \textbf{One idea per line.}  Statements can be chained with colons
    (\texttt{:}), but a single clear statement per line is easier to read
    and debug.
  \item \textbf{Group by function.}  Put constants and configuration near
    the top (lines~10--90), input routines in the 100s, game logic in the
    200s--500s, display routines in the 600s, cleanup in the 900s.
  \item \textbf{Name things meaningfully.}  \texttt{LIVES} is clearer than
    \texttt{L}; \texttt{SCORE} is clearer than \texttt{S1}.  NovaBASIC
    accepts long names.
  \item \textbf{Define constants up front.}
    \texttt{10 MAXLIVES=5 : STARTSPEED=2} at the top of the program means
    you tune the game by changing one line, not hunting through hundreds.
\end{itemize}
\end{tipbox}

% ---------------------------------------------------------------------------
\section{Putting It Together}

\begin{tryitbox}
Enter and run this program.  It builds an array of random values and prints
them in indexed form.
\begin{lstlisting}[style=basiclisting]
10 DIM A(5)
20 FOR I=1 TO 5
30   A(I) = INT(RND(1) * 100)
40 NEXT I
50 FOR I=1 TO 5
60   PRINT "A(";I;") = ";A(I)
70 NEXT I
\end{lstlisting}
\textbf{Expected:} Five lines printed, each showing an index and a random
integer between 0 and~99.  The values differ every time you \cmd{RUN}.
\end{tryitbox}

\noindent
Try modifying the program: change \texttt{100} to \texttt{10} to get
single-digit values, or change \texttt{5} to \texttt{20} to see a longer
array.  The structure stays the same; only the constants change.

% =============================================================================
% Chapter 4 — Graphics and Sprites
% NovaBASIC v1.0 User Guide
% =============================================================================
\chapter{Graphics and Sprites}
\label{chap:graphics}

\epigraph{\itshape ``A blank screen is a canvas waiting to be claimed.''}%
         {--- e6502 Virtual Computer Design Notes}

\noindent
NovaBASIC gives you direct access to a 320$\times$200 pixel bitmap and a
hardware sprite layer. Drawing commands operate on 16 colors; sprites add
independently positioned, independently animated 16$\times$16 objects on top
of or behind the bitmap and text layers. This chapter covers the full
graphics pipeline from mode selection to collision detection.

% -----------------------------------------------------------------------------
\section{Display Modes}
\label{sec:display-modes}
% -----------------------------------------------------------------------------

The virtual display has two independent layers: a text layer and a graphics
bitmap layer. \cmd{MODE} selects how they are composited.

\begin{center}
\begin{tabular}{>{\ttfamily}c l}
\toprule
\textbf{Mode} & \textbf{Description} \\
\midrule
0 & Text only. The graphics bitmap is not rendered. \\
1 & Graphics over text. Bitmap is drawn on top of text characters. \\
2 & Text over graphics. Text characters are drawn on top of the bitmap. \\
\bottomrule
\end{tabular}
\end{center}

\medskip
The typical starting sequence for any graphics program is:

\begin{lstlisting}[style=basiclisting]
10 MODE 1
20 GCLS
30 GCOLOR 7
\end{lstlisting}

\cmd{MODE 1} activates pixel rendering. \cmd{GCLS} clears the bitmap to
transparent (color~0). \cmd{GCOLOR} sets the active drawing color for all
subsequent drawing commands.

\begin{notebox}
Color 0 is transparent in the graphics layer. Setting a pixel to color~0
with \cmd{PLOT 0} or \cmd{FILL} erases it, letting the text layer or
background show through. This is equivalent to \cmd{UNPLOT}.
\end{notebox}

To return to plain text output, switch back to \cmd{MODE 0}. You do not
need to clear the bitmap when switching modes; the pixel data is preserved
and will reappear if you switch back to \cmd{MODE 1} or \cmd{MODE 2}.

% -----------------------------------------------------------------------------
\section{Drawing Commands}
\label{sec:drawing-commands}
% -----------------------------------------------------------------------------

All drawing commands use the color set by \cmd{GCOLOR}. Coordinates must
fall within the screen boundaries of X~=~0--319 and Y~=~0--199; pixels
outside that range are silently clipped and no error is raised.

\subsection*{Command reference}

\begin{longtable}{>{\ttfamily\small}p{0.36\textwidth} p{0.54\textwidth}}
\toprule
\textbf{Command} & \textbf{Description} \\
\midrule
\endhead
\bottomrule
\endfoot
GCOLOR c            & Set the active drawing color. \cmd{c} is 0--15.
                      If \cmd{c}~=~0, NovaBASIC uses the current text
                      foreground color instead of transparent. \\[4pt]
GCLS                & Clear the entire graphics bitmap to transparent
                      (color~0). Does not affect the text layer. \\[4pt]
PLOT x,y            & Set the pixel at (\cmd{x},\cmd{y}) to the current
                      drawing color. \\[4pt]
UNPLOT x,y          & Set the pixel at (\cmd{x},\cmd{y}) to transparent
                      (color~0), effectively erasing it. \\[4pt]
LINE x0,y0,x1,y1    & Draw a straight line from (\cmd{x0},\cmd{y0}) to
                      (\cmd{x1},\cmd{y1}) in the current drawing color. \\[4pt]
RECT x0,y0,x1,y1    & Draw a rectangle outline. (\cmd{x0},\cmd{y0}) is the
                      top-left corner; (\cmd{x1},\cmd{y1}) is the
                      bottom-right corner. \\[4pt]
FILL x0,y0,x1,y1    & Draw a solid filled rectangle using the same corner
                      convention as \cmd{RECT}. \\[4pt]
CIRCLE cx,cy,r      & Draw a circle outline centered at (\cmd{cx},\cmd{cy})
                      with radius \cmd{r} pixels. \\[4pt]
PAINT x,y           & Flood-fill from seed point (\cmd{x},\cmd{y}),
                      replacing all connected pixels of the same color
                      with the current drawing color. \\
\end{longtable}

\subsection*{A drawing example}

The following program draws a diagonal cross, a circle, and then fills the
circle interior:

\begin{lstlisting}[style=basiclisting]
10 MODE 1 : GCLS
20 GCOLOR 9
30 LINE 0,0,319,199
40 LINE 319,0,0,199
50 GCOLOR 14
60 CIRCLE 160,100,60
70 GCOLOR 10
80 PAINT 160,100
90 VSYNC
\end{lstlisting}

Line~30--40 draws a white cross from corner to corner. Lines~60--80 add a
yellow circle outline and then flood-fill the interior with green. \cmd{VSYNC}
on line~90 holds the image for one frame before the program ends; without it
the display may update before you see the result.

\begin{tipbox}
\cmd{PAINT} stops at pixel boundaries of a different color. Make sure the
circle or region you want to fill has no gaps, otherwise the fill will leak
out into the surrounding area. If in doubt, draw the boundary in one step and
fill immediately after.
\end{tipbox}

% -----------------------------------------------------------------------------
\section{Animation with VSYNC}
\label{sec:vsync-animation}
% -----------------------------------------------------------------------------

The virtual display runs at 60~Hz. \cmd{VSYNC} suspends program execution
until the start of the next video frame. One \cmd{VSYNC} call therefore
consumes exactly one frame period ($\approx 16.7$~ms). This is the correct
tool for controlling animation speed.

A minimal animation loop that moves a point across the screen:

\begin{lstlisting}[style=basiclisting]
10 MODE 1 : GCLS : GCOLOR 11
20 X = 0 : Y = 100
30 VSYNC
40 UNPLOT X, Y
50 X = X + 2
60 IF X > 319 THEN X = 0
70 PLOT X, Y
80 GOTO 30
\end{lstlisting}

The pattern is always: wait for VSYNC, erase the old position, update
coordinates, draw the new position. Erasing before moving eliminates the
ghost trail that builds up if you draw without erasing.

\begin{tipbox}
For smooth movement, do all erase operations for a frame, update all
positions, and do all draw operations --- all within a single VSYNC period.
Never call \cmd{VSYNC} between the erase and redraw steps for the same
object; that produces a one-frame flicker every cycle.
\end{tipbox}

% -----------------------------------------------------------------------------
\section{Sprites}
\label{sec:sprites}
% -----------------------------------------------------------------------------

Sprites are hardware-accelerated 16$\times$16 pixel objects that move
independently of the bitmap. NovaBASIC supports 16 sprites (indices 0--15),
each with its own shape, position, priority, and flip state. Sprites do not
modify the bitmap; they are composited at render time.

\subsection*{Enabling and positioning sprites}

A sprite must be enabled before it becomes visible:

\begin{lstlisting}[style=basiclisting]
10 SPRITE 0, ON
20 SPRITE 0, 160, 100
\end{lstlisting}

\cmd{SPRITE n,ON} activates sprite \cmd{n}. \cmd{SPRITE n,x,y} sets its
screen position. Positions are in the same coordinate space as the bitmap
(X~=~0--319, Y~=~0--199). Sprites may be positioned partially or fully
off-screen; they are simply clipped without error.

To hide a sprite, use \cmd{SPRITE n,OFF}. This makes the sprite invisible
without erasing its shape data. You can re-enable it later with
\cmd{SPRITE n,ON} and it will reappear at its last recorded position.

\subsection*{Sprite priority}

Priority controls which layer a sprite is drawn on:

\begin{center}
\begin{tabular}{cl}
\toprule
\textbf{Priority} & \textbf{Layer position} \\
\midrule
0 & Behind all layers (below text and graphics) \\
1 & Between the text and graphics layers \\
2 & In front of all layers (above text and graphics) \\
\bottomrule
\end{tabular}
\end{center}

Priority is set via the MCP sprite tools when building shapes interactively;
it can also be arranged by designing your program so that background sprites
are enabled first and foreground sprites last.

\subsection*{Defining sprite pixels with SPRITEDATA}

Each sprite is 16~pixels wide by 16~pixels tall. Pixel data is loaded one
row at a time using \cmd{SPRITEDATA}:

\begin{center}
\cmd{SPRITEDATA n, row, b1, b2, b3, b4, b5, b6, b7, b8}
\end{center}

\begin{itemize}
  \item \cmd{n} is the sprite index (0--15).
  \item \cmd{row} is the row to define (0--15, top to bottom).
  \item \cmd{b1}--\cmd{b8} are eight byte values (0--255).
\end{itemize}

Each byte encodes \emph{two} pixels. The high nibble (upper four bits) is the
left pixel; the low nibble (lower four bits) is the right pixel. Color~0 is
transparent; colors 1--15 are the standard 16-color palette. With eight bytes
per row and two pixels per byte, each row is exactly 16 pixels wide.

\medskip
\textbf{Example:} a byte value of \cmd{\$AC} (decimal 172) draws pixel color
10 (\cmd{A} in hex) on the left and pixel color 12 (\cmd{C} in hex) on the
right. A value of \cmd{\$00} leaves both pixels transparent.

\medskip
The following example defines a simple 16$\times$16 diamond shape in color~11
(cyan) and displays it:

\begin{lstlisting}[style=basiclisting]
10 REM DEFINE A DIAMOND SPRITE
20 SPRITEDATA 0,  0, 0,   0,   0,  0,  0,   0,   0,  0
30 SPRITEDATA 0,  1, 0,   0,   0, 11,  0,   0,   0,  0
40 SPRITEDATA 0,  2, 0,   0, 177,177,  0,   0,   0,  0
50 SPRITEDATA 0,  3, 0, 187, 187,187,187,   0,   0,  0
60 SPRITEDATA 0,  4, 0, 187, 187,187,187,   0,   0,  0
70 SPRITEDATA 0,  5, 0,   0, 177,177,  0,   0,   0,  0
80 SPRITEDATA 0,  6, 0,   0,   0, 11,  0,   0,   0,  0
90 SPRITEDATA 0,  7, 0,   0,   0,  0,  0,   0,   0,  0
100 REM ROWS 8-15 REMAIN TRANSPARENT (NO SPRITEDATA = NO CHANGE)
110 SPRITE 0, ON
120 SPRITE 0, 152, 92
130 VSYNC
\end{lstlisting}

\begin{notebox}
Any row not explicitly defined by \cmd{SPRITEDATA} retains its previous pixel
data. If you are reusing a sprite slot for a new shape, define all 16 rows
(or clear the slot first). Rows you intentionally leave all-zero produce a
fully transparent row.
\end{notebox}

The bytes in lines~30--80 use decimal notation. Working with hex notation is
often more readable: \cmd{0xBB} = decimal~187, which encodes color~11 in both
nibbles (solid cyan on both pixels of that byte). In NovaBASIC you can write
hex literals directly in expressions using \cmd{\&HBB} notation.

% -----------------------------------------------------------------------------
\section{Sprite Collision Detection}
\label{sec:collision}
% -----------------------------------------------------------------------------

NovaBASIC provides two collision functions that report when sprites overlap
each other or touch non-transparent pixels on the background bitmap.

\begin{center}
\begin{tabular}{>{\ttfamily}p{0.22\textwidth} p{0.65\textwidth}}
\toprule
\textbf{Function} & \textbf{Returns} \\
\midrule
COLLISION(n) & Bitmask of other sprites currently overlapping sprite \cmd{n}. \\[4pt]
BUMPED(n)    & Bitmask indicating that sprite \cmd{n} has touched a
               non-transparent pixel in the graphics bitmap. \\
\bottomrule
\end{tabular}
\end{center}

Both functions return an integer bitmask. Bit \cmd{k} being set means sprite
\cmd{k} is involved in the collision. For \cmd{COLLISION(n)}, if the result
is non-zero then at least one other sprite overlaps sprite~\cmd{n}; use
\cmd{AND} with the appropriate bit to test for a specific sprite. For
\cmd{BUMPED(n)}, a non-zero result means sprite~\cmd{n} is touching a
non-transparent pixel in the graphics bitmap.

\begin{warningbox}
Both collision registers clear automatically when read. Read each register
exactly once per frame and store the result in a variable. If you call
\cmd{COLLISION(n)} or \cmd{BUMPED(n)} a second time in the same frame you
will get zero, missing collisions that occurred between reads.
\end{warningbox}

A practical collision loop pattern:

\begin{lstlisting}[style=basiclisting]
100 VSYNC
110 C = COLLISION(0)
120 B = BUMPED(0)
130 IF C <> 0 THEN GOSUB 500
140 IF B <> 0 THEN GOSUB 600
150 REM UPDATE POSITIONS HERE
160 GOTO 100
\end{lstlisting}

Lines~110--120 read both registers once and store them. Lines~130--140 branch
to handler routines only if a collision has occurred. All position updates
happen after the collision check so that the same frame's register values are
used consistently.

To test whether sprite~\cmd{n} specifically collided with sprite~2, check bit
2 of the \cmd{COLLISION} result:

\begin{lstlisting}[style=basiclisting]
200 C = COLLISION(0)
210 IF (C AND 4) <> 0 THEN PRINT "HIT SPRITE 2"
\end{lstlisting}

% -----------------------------------------------------------------------------
\section{Compatibility Notes}
\label{sec:sprite-compat}
% -----------------------------------------------------------------------------

\begin{warningbox}
The following commands and functions are parsed by NovaBASIC for source
compatibility but currently have no effect at runtime:

\begin{itemize}
  \item \cmd{SPRITESHAPE} --- accepted without error, silently ignored.
  \item \cmd{SPRITECOLOR} --- accepted without error, silently ignored.
  \item \cmd{SPRITEX(n)} --- always returns 0; does not reflect actual
        sprite X position.
  \item \cmd{SPRITEY(n)} --- always returns 0; does not reflect actual
        sprite Y position.
\end{itemize}

Programs that read sprite position back via \cmd{SPRITEX}/\cmd{SPRITEY} must
instead track coordinates in their own variables.
\end{warningbox}

% -----------------------------------------------------------------------------
\section{Try It Now}
\label{sec:graphics-tryit}
% -----------------------------------------------------------------------------

\begin{tryitbox}
Type and run the following program to see \cmd{MODE}, \cmd{GCLS},
\cmd{GCOLOR}, \cmd{RECT}, \cmd{CIRCLE}, and \cmd{PAINT} working together:

\begin{lstlisting}[style=basiclisting]
10 MODE 1 : GCLS : GCOLOR 10
20 RECT 10, 10, 309, 189
30 GCOLOR 14 : CIRCLE 160, 100, 50
40 GCOLOR 12 : PAINT 160, 100
\end{lstlisting}

Expected result: a green rectangle border frames the screen; inside it a
yellow circle outline encloses a solid red filled region.

\medskip
Try modifying \cmd{GCOLOR} values (1--15) and the \cmd{CIRCLE} radius to
explore the coordinate system. Then add a second \cmd{CIRCLE} call on a new
line and re-run to see both circles on the same canvas.
\end{tryitbox}

% =============================================================================
% Chapter 5 — Sound and Music
% NovaBASIC v1.0 User Guide
% Source of truth: ehbasic/basic.asm (SOUND, INSTRUMENT, VOLUME, MUSIC tokens)
%                  e6502.Avalonia/Hardware/SidChip.cs
%                  e6502.Avalonia/Hardware/MusicEngine.cs
%                  e6502.Avalonia/Hardware/MmlParser.cs
%                  e6502.Avalonia/Hardware/FileIoController.cs
% =============================================================================
\chapter{Sound and Music}
\label{chap:sound}

\epigraph{\itshape ``Music is the arithmetic of sounds as optics is the
  geometry of light.''}%
         {--- Claude Debussy}

\noindent
NovaBASIC includes a SID chip emulator --- a software recreation of the MOS
6581 Sound Interface Device made famous by the Commodore~64. Three independent
voices with four waveforms, ADSR envelopes, and a programmable filter deliver
authentic chiptune sound. On top of the SID sits a three-voice MML music
sequencer with per-frame effects including vibrato, portamento, arpeggios,
pulse-width modulation, and filter sweeps.

This chapter covers every sound command from simple one-shot notes to
full multi-voice compositions.

% -----------------------------------------------------------------------------
\section{Quick-Start Overview}
\label{sec:sound-overview}
% -----------------------------------------------------------------------------

The sound system has three layers, each building on the one below:

\begin{enumerate}
  \item \textbf{SOUND} --- play a single note on the SID chip. Specify a MIDI
        note number, duration in frames, and an optional instrument preset.
  \item \textbf{INSTRUMENT} --- define a reusable preset that sets the SID
        waveform and ADSR envelope. Up to 16 presets (slots 0--15).
  \item \textbf{MUSIC} --- load MML (Music Macro Language) sequences into up
        to three voices and play them back with tempo, looping, and
        per-frame effects.
\end{enumerate}

A minimal program that plays a note:

\begin{lstlisting}[style=basiclisting]
10 VOLUME 12
20 SOUND 60, 30
\end{lstlisting}

Line~10 sets master volume. Line~20 plays MIDI note~60 (middle~C) for
30 frames (half a second at 60~Hz).

% -----------------------------------------------------------------------------
\section{The SOUND Command}
\label{sec:sound-command}
% -----------------------------------------------------------------------------

\begin{center}
\cmd{SOUND note, duration {[}, instrument{]}}
\end{center}

\begin{itemize}
  \item \cmd{note} --- MIDI note number (0--127). Middle~C is 60; A4
        (concert pitch 440~Hz) is 69. See the MIDI note table below.
  \item \cmd{duration} --- length in 1/60-second frames. A value of 60
        plays for one second; 30 plays for half a second.
  \item \cmd{instrument} --- optional instrument preset (0--15). If omitted,
        instrument~0 is used.
\end{itemize}

If \cmd{note} or \cmd{duration} is zero, the sound is stopped immediately.

\begin{notebox}
\cmd{SOUND} triggers a one-shot sound effect through the music engine's
SFX channel. It does not interrupt music playback; the engine allocates a
voice for the effect and restores it when the sound completes.
\end{notebox}

\subsection*{Common MIDI note numbers}

\begin{center}
\begin{tabular}{lrl}
\toprule
\textbf{Note} & \textbf{MIDI} & \textbf{Approx.\ Frequency} \\
\midrule
C3  & 48 & 131 Hz \\
C4 (Middle C) & 60 & 262 Hz \\
D4  & 62 & 294 Hz \\
E4  & 64 & 330 Hz \\
F4  & 65 & 349 Hz \\
G4  & 67 & 392 Hz \\
A4  & 69 & 440 Hz \\
B4  & 71 & 494 Hz \\
C5  & 72 & 523 Hz \\
C6  & 84 & 1047 Hz \\
\bottomrule
\end{tabular}
\end{center}

\medskip
The formula is: frequency $= 440 \times 2^{(\text{midi} - 69) / 12}$.

\subsection*{A simple melody}

\begin{lstlisting}[style=basiclisting]
10 VOLUME 12
20 DATA 60, 62, 64, 65, 67, 69, 71, 72
30 FOR N = 1 TO 8
40   READ M
50   SOUND M, 15
60   FOR I = 1 TO 15 : VSYNC : NEXT I
70 NEXT N
\end{lstlisting}

Each note plays for 15 frames ($\approx 250$~ms). The \cmd{VSYNC} loop
holds the program for the same duration before the next note fires.

\begin{notebox}
\cmd{SOUND} does not block program execution. Use a \cmd{VSYNC} loop after
each \cmd{SOUND} call to create the gap between notes.
\end{notebox}

% -----------------------------------------------------------------------------
\section{The INSTRUMENT Command}
\label{sec:instrument}
% -----------------------------------------------------------------------------

\begin{center}
\cmd{INSTRUMENT id, waveform, attack, decay, sustain, release}
\end{center}

Defines a reusable sound preset in one of 16 instrument slots.

\begin{itemize}
  \item \cmd{id} --- slot number (0--15).
  \item \cmd{waveform} --- SID waveform byte:
        \texttt{\$10}~=~triangle, \texttt{\$20}~=~sawtooth,
        \texttt{\$40}~=~pulse, \texttt{\$80}~=~noise.
  \item \cmd{attack} --- attack rate (0--15). 0 is instantaneous; 15 is
        slowest.
  \item \cmd{decay} --- decay rate (0--15). How quickly the volume drops
        from peak to the sustain level.
  \item \cmd{sustain} --- sustain level (0--15). The steady-state volume
        held while the note plays. 15~=~full volume; 0~=~silent (percussive).
  \item \cmd{release} --- release rate (0--15). How quickly the volume fades
        to silence after the note ends.
\end{itemize}

\begin{lstlisting}[style=basiclisting]
10 REM BRIGHT PULSE LEAD
20 INSTRUMENT 0, $40, 0, 9, 0, 6
30 REM WARM SAWTOOTH PAD
40 INSTRUMENT 1, $20, 4, 6, 12, 8
50 REM NOISE DRUM HIT
60 INSTRUMENT 2, $80, 0, 3, 0, 2
\end{lstlisting}

\begin{notebox}
Instrument~0 is pre-initialized at boot with: pulse waveform (\texttt{\$40}),
attack~0, decay~9, sustain~0, release~6, pulse width~2048. All other slots
(1--15) start as copies of slot~0.
\end{notebox}

\subsection*{SID waveform reference}

\begin{center}
\begin{tabular}{clp{0.50\textwidth}}
\toprule
\textbf{Value} & \textbf{Name} & \textbf{Character} \\
\midrule
\texttt{\$10} & Triangle & Soft and mellow; flute-like. Good for gentle
  melodies and background pads. \\[2pt]
\texttt{\$20} & Sawtooth & Buzzy and harmonically rich. Good for brass-like
  leads and bass lines. \\[2pt]
\texttt{\$40} & Pulse    & Bold, hollow, classic chiptune sound. Pulse width
  can be modulated via MML for evolving timbres. \\[2pt]
\texttt{\$80} & Noise    & Unpitched random output. Use for drums, hi-hats,
  explosions, and ambient textures. \\
\bottomrule
\end{tabular}
\end{center}

\subsection*{ADSR envelope overview}

The four parameters shape how a note's volume changes over time:

\begin{enumerate}
  \item \textbf{Attack} ramps from silence to full amplitude.
  \item \textbf{Decay} drops from full amplitude to the sustain level.
  \item \textbf{Sustain} holds at a constant level while the note plays.
  \item \textbf{Release} fades from the sustain level to silence after the
        note ends.
\end{enumerate}

Sustain is a \emph{level} (0--15); the other three are \emph{rate} values
where 0 is fastest and 15 is slowest. This matches the original SID chip
behavior.

\subsection*{Instrument recipes}

\begin{center}
\begin{tabular}{lcccccp{0.28\textwidth}}
\toprule
\textbf{Sound} & \textbf{Wave} & \textbf{A} & \textbf{D} & \textbf{S} & \textbf{R} & \textbf{Notes} \\
\midrule
Chiptune lead   & \$40 &  0 &  9 &  0 &  6 & Sharp attack, no sustain \\
Warm pad        & \$20 &  8 &  6 & 12 & 10 & Slow fade-in, high sustain \\
Bass            & \$20 &  0 &  5 &  8 &  4 & Instant attack, medium body \\
Snare drum      & \$80 &  0 &  3 &  0 &  2 & Short noise burst \\
Hi-hat          & \$80 &  0 &  1 &  0 &  1 & Very short noise tick \\
Organ           & \$10 &  0 &  0 & 15 &  4 & Triangle at full sustain \\
Pluck           & \$40 &  0 & 12 &  0 &  8 & Fast decay, no sustain \\
\bottomrule
\end{tabular}
\end{center}

% -----------------------------------------------------------------------------
\section{VOLUME}
\label{sec:volume}
% -----------------------------------------------------------------------------

\begin{center}
\cmd{VOLUME level}
\end{center}

Sets the SID master volume. \cmd{level} is 0--15 (only the low nibble is
used). The default volume at boot is 12.

% -----------------------------------------------------------------------------
\section{The MUSIC Engine}
\label{sec:music-engine}
% -----------------------------------------------------------------------------

The music engine is a three-voice MML sequencer running on top of the SID
chip. You write melodies and rhythms as text strings using Music Macro
Language, load them into voices, and let the engine handle all the timing,
instrument switching, and per-frame effects automatically.

\subsection{Loading and Playing Sequences}

\begin{center}
\cmd{MUSIC voice, "mml-string"}
\end{center}

\begin{itemize}
  \item \cmd{voice} --- voice number 1--3.
  \item \cmd{"mml-string"} --- an MML sequence (see Section~\ref{sec:mml}).
\end{itemize}

Additional subcommands control playback:

\begin{center}
\begin{tabular}{>{\ttfamily}p{0.36\textwidth} p{0.54\textwidth}}
\toprule
\textbf{Command} & \textbf{Description} \\
\midrule
MUSIC PLAY             & Start playback of all loaded voices. \\[3pt]
MUSIC STOP             & Stop playback and silence all music voices. \\[3pt]
MUSIC TEMPO bpm        & Set tempo in beats per minute. Default is 120. \\[3pt]
MUSIC LOOP ON          & Enable looping; voices restart when all finish. \\[3pt]
MUSIC LOOP OFF         & Disable looping (default). \\[3pt]
MUSIC PRIORITY v1{[},v2{[},v3{]}} & Set voice-stealing priority for sound
  effects. Lower-numbered voices are stolen first. Default: 3,~2,~1. \\
\bottomrule
\end{tabular}
\end{center}

\subsection{A Complete Music Example}

\begin{lstlisting}[style=basiclisting]
10 VOLUME 12
20 REM DEFINE INSTRUMENTS
30 INSTRUMENT 0, $40, 0, 9, 0, 6
40 INSTRUMENT 1, $20, 0, 5, 8, 4
50 REM LOAD VOICES
60 MUSIC 1, "T140 I0 L8 O4 CDEFGAB >C"
70 MUSIC 2, "T140 I1 L4 O3 C G C G"
80 REM START PLAYBACK
90 MUSIC LOOP ON
100 MUSIC PLAY
\end{lstlisting}

Line~60 loads a melody into voice~1: tempo~140, instrument~0, eighth notes,
octave~4, ascending C~major scale. Line~70 loads a bass line into voice~2:
quarter notes, octave~3, alternating C~and~G. Line~100 starts playback;
\cmd{MUSIC LOOP ON} on line~90 means the music repeats indefinitely.

\subsection{Querying Music Status}

Two functions let you check what the music engine is doing:

\begin{center}
\begin{tabular}{>{\ttfamily}p{0.22\textwidth} p{0.65\textwidth}}
\toprule
\textbf{Function} & \textbf{Returns} \\
\midrule
PLAYING & 1 if music is currently playing, 0 if stopped. \\[3pt]
MNOTE(voice) & The MIDI note number currently sounding on \cmd{voice}
  (1--3), or 0 if that voice is silent. \\
\bottomrule
\end{tabular}
\end{center}

\begin{lstlisting}[style=basiclisting]
200 IF PLAYING THEN GOTO 200
210 PRINT "MUSIC FINISHED"
\end{lstlisting}

% =============================================================================
\section{MML Reference}
\label{sec:mml}
% =============================================================================

Music Macro Language (MML) is a compact text notation for music. Each voice
receives its own MML string. The parser is case-insensitive; all input is
converted to uppercase before processing.

\subsection{Notes and Rests}

\begin{center}
\begin{tabular}{>{\ttfamily}p{0.24\textwidth} p{0.64\textwidth}}
\toprule
\textbf{Syntax} & \textbf{Description} \\
\midrule
C D E F G A B & Play a note. Pitch is determined by the current octave. \\[3pt]
C\# \textnormal{or} C+ & Sharp (raise one semitone). \\[3pt]
C- & Flat (lower one semitone). \\[3pt]
R & Rest (silence for the note duration). \\
\bottomrule
\end{tabular}
\end{center}

\subsection{Duration}

A number following a note or rest sets its length as a note-value denominator:

\begin{center}
\begin{tabular}{crl}
\toprule
\textbf{Denominator} & \textbf{Ticks} & \textbf{Name} \\
\midrule
1  & 384 & Whole note \\
2  & 192 & Half note \\
4  &  96 & Quarter note \\
8  &  48 & Eighth note \\
16 &  24 & Sixteenth note \\
32 &  12 & Thirty-second note \\
\bottomrule
\end{tabular}
\end{center}

Internally, one quarter note equals 96 ticks.

A dot (\cmd{.}) after the duration extends it by 50\%: \cmd{C4.} plays for
$96 \times 1.5 = 144$ ticks (dotted quarter).

If no duration is given, the default length set by \cmd{L} is used (initially
a quarter note).

\subsection{Ties}

The ampersand (\cmd{\&}) ties two durations together into a single sustained
note:

\begin{center}
\cmd{C4\&8} $\rightarrow$ quarter + eighth = 144 ticks
\end{center}

Multiple ties can be chained: \cmd{C4\&4\&4} plays for $96 + 96 + 96 = 288$
ticks. A single \texttt{NoteOn} event is emitted with the combined duration.

\subsection{Octave}

\begin{center}
\begin{tabular}{>{\ttfamily}p{0.12\textwidth} p{0.76\textwidth}}
\toprule
\textbf{Cmd} & \textbf{Description} \\
\midrule
O4   & Set absolute octave (range 1--7; default 4). \\
>    & Octave up (clamped to 7). \\
<    & Octave down (clamped to 1). \\
\bottomrule
\end{tabular}
\end{center}

MIDI note calculation: \texttt{midi = (octave + 1) * 12 + semitone}, where
C=0, D=2, E=4, F=5, G=7, A=9, B=11.

\subsection{Default Length}

\cmd{L8} sets the default note/rest duration to eighth notes. All subsequent
notes and rests that omit an explicit duration will use this value.

\subsection{Tempo}

\cmd{T120} sets the tempo to 120 beats per minute. The default is 120.
Tempo can appear anywhere in the MML string and takes effect immediately.
At 120~BPM, one quarter note lasts exactly 0.5~seconds.

\begin{notebox}
Tempo is global. If multiple voices contain \cmd{T} commands, the last one
processed wins. It is best practice to set tempo in voice~1 only.
\end{notebox}

\subsection{Instrument Selection}

\cmd{I3} switches the current voice to instrument slot~3 (defined earlier
with the \cmd{INSTRUMENT} BASIC command). Instrument changes take effect on
the next note.

\subsection{Loops}

Square brackets repeat a section:

\begin{center}
\cmd{{[}CDEF{]}3} $\rightarrow$ plays C~D~E~F three times
\end{center}

The repeat count follows the closing bracket. If omitted, the default is~1
(no repetition). Loops do not nest.

\subsection{Arpeggios}

Curly braces define an arpeggio --- a rapid cycling through multiple notes:

\begin{center}
\cmd{\{CEG\}4} $\rightarrow$ cycle C, E, G at 60~Hz for one quarter note
\end{center}

Each frame advances to the next note in the list. Accidentals are supported
inside the braces (\cmd{\{C\#EG\#\}}). The arpeggio duration follows the
closing brace using standard duration syntax.

% =============================================================================
\subsection{Per-Frame Effects}
\label{sec:mml-effects}
% =============================================================================

The music engine processes the following effects on every frame (60~Hz).
These effects are set within MML and remain active until changed or a new
note resets them.

\subsubsection*{Vibrato}

\cmd{\textasciitilde6} sets vibrato depth to~6. Higher values produce wider
pitch oscillation. \cmd{\textasciitilde0} turns vibrato off.

The vibrato oscillates at approximately 2.9~Hz (sine wave). The pitch offset
is proportional to both the depth value and the current note frequency.

\subsubsection*{Portamento (Pitch Slide)}

\cmd{/} before a note causes the voice to slide from the current pitch to
the target note rather than jumping instantly.

\begin{center}
\cmd{C4 /E4} $\rightarrow$ play C, then glide smoothly to E
\end{center}

The slide rate is approximately $\frac{1}{8}$ of the frequency distance per
frame.

\subsubsection*{Pulse Width}

\cmd{@P2048} sets the SID pulse width to 2048 (range 0--4095). This only
affects the pulse waveform (\texttt{\$40}). A value of 2048 gives a 50\%
duty cycle (square wave); lower or higher values create thinner, more nasal
timbres.

\subsubsection*{Pulse Width Modulation (PWM)}

\cmd{@PS+} starts sweeping the pulse width upward; \cmd{@PS-} sweeps
downward; \cmd{@PS0} stops the sweep. The sweep rate is $\pm$32 per frame,
clamped to the 0--4095 range. PWM gives the pulse waveform a rich,
evolving character.

\begin{lstlisting}[style=basiclisting]
60 MUSIC 1, "@P1024 @PS+ O4 L2 C E G >C"
\end{lstlisting}

\subsubsection*{Filter Cutoff and Resonance}

\cmd{@F1024,8} sets the SID filter cutoff to 1024 (range 0--2047) and
resonance to 8 (range 0--15). The resonance parameter is optional; if
omitted, it defaults to~0.

\subsubsection*{Filter Mode}

\begin{center}
\begin{tabular}{>{\ttfamily}p{0.08\textwidth} p{0.50\textwidth}}
\toprule
\textbf{Cmd} & \textbf{Mode} \\
\midrule
@FL & Low-pass (cuts highs, warm sound) \\
@FB & Band-pass (emphasizes a frequency band) \\
@FH & High-pass (cuts lows, thin sound) \\
@FO & Filter off \\
\bottomrule
\end{tabular}
\end{center}

\subsubsection*{Filter Sweep}

\cmd{@FS+} sweeps the filter cutoff upward; \cmd{@FS-} sweeps downward;
\cmd{@FS0} stops the sweep. The sweep rate is $\pm$8 per frame, clamped to
0--2047.

\begin{lstlisting}[style=basiclisting]
60 MUSIC 1, "@FL @F200,12 @FS+ L4 O3 [CDEFGAB>C<]2"
\end{lstlisting}

This creates a classic filter sweep effect: low-pass filter starting at
cutoff~200 with high resonance, sweeping upward through the melody.

% =============================================================================
\subsection{MML Command Summary}
\label{sec:mml-summary}
% =============================================================================

\begin{longtable}{>{\ttfamily\small}p{0.22\textwidth} p{0.20\textwidth} p{0.44\textwidth}}
\toprule
\textbf{Command} & \textbf{Parameters} & \textbf{Description} \\
\midrule
\endhead
\bottomrule
\endfoot
A--G        & {[}\#/+/-{]}{[}dur{]}{[}.{]} & Play note \\[2pt]
R           & {[}dur{]}{[}.{]}   & Rest \\[2pt]
O           & 1--7               & Set octave \\[2pt]
>           & ---                & Octave up \\[2pt]
<           & ---                & Octave down \\[2pt]
L           & denominator        & Default note length \\[2pt]
T           & bpm                & Tempo (default 120) \\[2pt]
I           & 0--15              & Select instrument slot \\[2pt]
\&          & {[}note{]}dur      & Tie (extend note duration) \\[2pt]
{[}\ldots{]}n & repeat count     & Loop section $n$ times \\[2pt]
\{notes\}   & {[}dur{]}{[}.{]}   & Arpeggio \\[2pt]
\textasciitilde & depth (0=off)  & Vibrato \\[2pt]
/           & ---                & Portamento (next note slides) \\[2pt]
@P          & 0--4095            & Set pulse width \\[2pt]
@PS         & +, -, 0            & PWM sweep direction \\[2pt]
@F          & cutoff{[},res{]}   & Filter cutoff (0--2047) and resonance (0--15) \\[2pt]
@FL         & ---                & Low-pass filter \\[2pt]
@FB         & ---                & Band-pass filter \\[2pt]
@FH         & ---                & High-pass filter \\[2pt]
@FO         & ---                & Filter off \\[2pt]
@FS         & +, -, 0            & Filter sweep direction \\
\end{longtable}

Whitespace, tabs, newlines, and pipe characters (\cmd{|}) are ignored and
can be used freely to format MML strings for readability.

% =============================================================================
\section{SID File Playback}
\label{sec:sidplay}
% =============================================================================

NovaBASIC can load and play standard \texttt{.sid} files --- the native
music format of the Commodore~64 scene:

\begin{center}
\begin{tabular}{>{\ttfamily}p{0.36\textwidth} p{0.54\textwidth}}
\toprule
\textbf{Command} & \textbf{Description} \\
\midrule
SIDPLAY "filename" {[}, song{]} & Load and play a \texttt{.sid} file. The
  optional \cmd{song} parameter selects which sub-tune to play (default~1). \\[3pt]
SIDSTOP & Stop SID file playback. \\
\bottomrule
\end{tabular}
\end{center}

\begin{lstlisting}[style=basiclisting]
10 SIDPLAY "commando"
20 FOR I = 1 TO 600 : VSYNC : NEXT I
30 SIDSTOP
\end{lstlisting}

SID files are loaded from the \texttt{\textasciitilde/e6502-programs}
directory. The \texttt{.sid} extension is added automatically. The SID
player injects an IRQ trampoline into CPU RAM that calls the file's init
and play routines at 60~Hz.

\begin{warningbox}
SID playback takes over the SID chip directly. \cmd{SOUND} and \cmd{MUSIC}
commands will not produce audible output while a SID file is playing. Call
\cmd{SIDSTOP} before using other sound commands.
\end{warningbox}

% =============================================================================
\section{Graphics File I/O}
\label{sec:gsave-gload}
% =============================================================================

NovaBASIC can save and load VGC memory spaces to disk:

\begin{center}
\begin{tabular}{>{\ttfamily}p{0.42\textwidth} p{0.48\textwidth}}
\toprule
\textbf{Command} & \textbf{Description} \\
\midrule
GSAVE "name", space, offset, len & Save \cmd{len} bytes from VGC memory
  space starting at \cmd{offset} to a \texttt{.gfx} file. \\[3pt]
GLOAD "name", space, offset{[}, len{]} & Load a \texttt{.gfx} file into
  VGC memory space at \cmd{offset}. If \cmd{len} is omitted, the entire
  file is loaded. \\
\bottomrule
\end{tabular}
\end{center}

VGC memory spaces:

\begin{center}
\begin{tabular}{cl}
\toprule
\textbf{Space} & \textbf{Contents} \\
\midrule
0 & Character RAM (2000 bytes) \\
1 & Color RAM (2000 bytes) \\
2 & Graphics bitmap (64000 bytes) \\
3 & Sprite shape RAM (2048 bytes) \\
\bottomrule
\end{tabular}
\end{center}

\begin{lstlisting}[style=basiclisting]
10 MODE 1 : GCLS
20 GCOLOR 10 : CIRCLE 160, 100, 80
30 GSAVE "mycircle", 2, 0, 64000
40 REM LATER...
50 GLOAD "mycircle", 2, 0
\end{lstlisting}

% =============================================================================
\section{Music Engine Architecture}
\label{sec:music-architecture}
% =============================================================================

For advanced users, understanding the engine internals helps write better
music and avoid common pitfalls.

\subsection*{Voice allocation}

The music engine manages three music voices (mapped to SID voices 0--2) plus
one shared SFX voice. When \cmd{SOUND} triggers a sound effect, the engine:

\begin{enumerate}
  \item Looks for a voice with no music sequence loaded.
  \item If all voices have sequences, steals a voice according to the
        priority order (default: voice~3 first, then~2, then~1).
  \item Plays the SFX on the stolen voice; when done, restores the music
        voice.
\end{enumerate}

\subsection*{Timing}

The engine ticks at 60~Hz. Tempo is converted to ticks per frame:

\begin{center}
$\text{ticks per frame} = \frac{96 \times \text{BPM}}{3600}$
\end{center}

At 120~BPM this is 3.2 ticks per frame. A quarter note (96~ticks) takes
exactly 30 frames = 0.5~seconds.

\subsection*{Effect processing order}

Each frame, active effects are processed in this order:

\begin{enumerate}
  \item Arpeggio (cycle to next note)
  \item PWM sweep ($\pm$32 per frame, clamped 0--4095)
  \item Vibrato (sine wave at $\sim$2.9~Hz)
  \item Portamento (slide $\frac{1}{8}$ of remaining distance per frame)
  \item Filter sweep ($\pm$8 per frame, clamped 0--2047)
\end{enumerate}

% =============================================================================
\section{Composition Tips}
\label{sec:composition-tips}
% =============================================================================

\begin{tipbox}
\begin{itemize}
  \item Define all instruments before loading music sequences.
  \item Use \cmd{I} in MML to switch instruments mid-voice for timbral
        variety.
  \item Keep melody, harmony, and bass on separate voices. Each voice has
        its own instrument, octave, and effect state.
  \item Use \cmd{|} characters in MML strings as bar-line separators for
        readability: \cmd{"L8 CDEF | GABC"}.
  \item At 120~BPM: quarter~=~30 frames, eighth~=~15, sixteenth~=~7.5. Use
        \cmd{T} to control tempo rather than adjusting note lengths.
  \item Use \cmd{@PS+} and \cmd{@PS-} on pulse waveforms for rich, evolving
        textures.
  \item Combine \cmd{@FL} with \cmd{@FS+} for classic acid-bass filter
        sweeps.
  \item The noise waveform (\texttt{\$80}) with short ADSR makes convincing
        drums. Load a noise instrument and trigger it with \cmd{SOUND}
        while the music plays.
\end{itemize}
\end{tipbox}

% =============================================================================
\section{Deprecated Commands}
\label{sec:deprecated-sound}
% =============================================================================

\begin{warningbox}
The following commands from earlier versions have been superseded:

\begin{itemize}
  \item \cmd{WAVE} --- raises a syntax error. Use \cmd{INSTRUMENT} instead.
  \item \cmd{ENVELOPE} --- replaced by \cmd{INSTRUMENT}. Programs should be
        updated to use the new six-parameter syntax.
\end{itemize}
\end{warningbox}

% =============================================================================
\section{Try It Now}
\label{sec:sound-tryit}
% =============================================================================

\begin{tryitbox}
Type and run the following program to hear a three-voice arrangement with
instrument presets, MML sequences, filter effects, and looping:

\begin{lstlisting}[style=basiclisting]
10 VOLUME 12
20 INSTRUMENT 0, $40, 0, 9, 0, 6
30 INSTRUMENT 1, $20, 0, 5, 8, 4
40 INSTRUMENT 2, $80, 0, 3, 0, 2
50 MUSIC 1, "T120 I0 L8 O4 CDEFGAB >C2"
60 MUSIC 2, "T120 I1 L4 O3 C G C G"
70 MUSIC 3, "T120 I2 L8 O4 R C R C R C R C"
80 MUSIC LOOP ON
90 MUSIC PLAY
\end{lstlisting}

Expected result: a three-voice loop with a pulse-wave melody, sawtooth bass,
and noise percussion. The music repeats until you type \cmd{MUSIC STOP} in
direct mode.

\medskip
Experiments:
\begin{itemize}
  \item Change \cmd{T120} to \cmd{T180} for a faster tempo.
  \item Add vibrato to voice~1: change the MML to start with
        \cmd{"T120 I0 \textasciitilde4 L8 O4 ..."}.
  \item Add a filter sweep to voice~2:
        \cmd{"T120 I1 @FL @F200,10 @FS+ L4 O3 C G C G"}.
\end{itemize}
\end{tryitbox}

\chapter{Expansion Memory}

\begin{retrobox}
\textit{``The problem with a 64-kilobyte address space is not that it is small. \\
It is that everything you want to do next is just barely larger than it.''}
\end{retrobox}

\bigskip

The 6502 CPU can directly address 64 KB. That is enough for code, stack, screen RAM,
and a modest program, but it leaves little room for data-heavy work: large sprite
sheets, level maps, sample tables, or pre-rendered buffers quickly overflow the
available RAM. The NovaBASIC expansion memory system solves this by providing an
additional 512 KB of RAM that lives outside the CPU address space, accessible through
a set of commands and a window-mapping mechanism.

%% ============================================================
\section{Why Expansion Memory?}
%% ============================================================

The CPU address bus is 16 bits wide. After subtracting ROM, the VGC, the sound
controller, the file I/O coprocessor, the XMC registers, and screen RAM, your usable
BASIC RAM runs from \texttt{\$0280} to \texttt{\$9FFF} --- approximately 39 KB. That
is enough for programs and variables, but becomes tight the moment you try to cache a
full screen buffer, store multiple music tracks, or keep several sprite-sheet layers
in memory simultaneously.

Expansion memory (\textbf{XRAM}) gives you up to 512 KB of additional storage with no
impact on the CPU address space. You cannot execute code from XRAM directly, but you
can:

\begin{itemize}
  \item Store and retrieve arbitrary data blocks by raw offset.
  \item Name blocks and recall them without tracking raw addresses.
  \item Map 256-byte XRAM pages directly into CPU address windows for transparent
        byte-level access via \cmd{PEEK} and \cmd{POKE}.
  \item Allocate and free regions using a lightweight page-tracking allocator.
\end{itemize}

The XRAM hardware is controlled by the Expansion Memory Controller (XMC), mapped at
\texttt{\$BA00}. You normally never touch those registers directly --- the BASIC
commands handle all of it for you.

%% ============================================================
\section{Banks and Status}
%% ============================================================

XRAM is divided into \textbf{banks}, each 64 KB in size. With the default 512 KB
configuration there are \textbf{8 banks} (numbered 0 through 7). One bank is always
\emph{active}; raw-offset commands operate within that bank.

\subsection*{Checking memory status: \cmd{XMEM}}

Type \cmd{XMEM} at the prompt to print a status summary:

\begin{retrobox}
\texttt{XMEM}\\[2pt]
\texttt{8 BANKS, 512 KB XRAM, BANK 0, USED 0, FREE 2048 PAGES}
\end{retrobox}

The output shows total bank count, total XRAM capacity, the currently active bank,
and the number of 256-byte pages that are in use versus free across the entire
512 KB space. Two thousand and forty-eight free pages means 512 KB is available.

\subsection*{Selecting a bank: \cmd{XBANK n}}

\begin{retrobox}
\texttt{XBANK 3}
\end{retrobox}

Selects bank 3 as the active bank. Subsequent \cmd{XPOKE}, \cmd{XPEEK}, and raw
\cmd{STASH}/\cmd{FETCH} operations use offsets within that bank. Valid range is
0 to 7 for the default 512 KB configuration.

\begin{notebox}
Named blocks (\cmd{STASH "name"} / \cmd{FETCH "name"}) are allocated automatically
across the full XRAM space and are not limited to the active bank. The active bank
setting only affects raw-offset operations.
\end{notebox}

%% ============================================================
\section{Single-Byte Access}
%% ============================================================

For simple tasks --- writing a flag, reading a configuration byte, probing a value
--- the single-byte commands are the most direct interface.

\subsection*{\cmd{XPOKE offset,value}}

Writes \emph{value} (0--255) to \emph{offset} within the currently active bank.
Offsets run from 0 to 65535.

\begin{retrobox}
\texttt{XBANK 0}\\
\texttt{XPOKE 0,42}\\
\texttt{XPOKE 1,255}
\end{retrobox}

\subsection*{\cmd{XPEEK(offset)}}

Reads a single byte from \emph{offset} in the active bank and returns it as a
numeric value. You can use it anywhere a number is valid:

\begin{retrobox}
\texttt{PRINT XPEEK(0)}\\
\texttt{V = XPEEK(1) + XPEEK(2)}
\end{retrobox}

\begin{tipbox}
\cmd{XPOKE} and \cmd{XPEEK} are convenient for a handful of bytes or for
inspecting specific locations. For moving blocks of data use \cmd{STASH} and
\cmd{FETCH}, which issue a single hardware transfer rather than looping byte
by byte.
\end{tipbox}

%% ============================================================
\section{Bulk Transfers}
%% ============================================================

Moving data in and out of XRAM is the core workflow for anything non-trivial. The
raw bulk-transfer commands operate on explicit offsets within the active bank.

\subsection*{Raw \cmd{STASH} --- CPU RAM to XRAM}

\begin{retrobox}
\texttt{STASH ramaddr,xramoffset,length}
\end{retrobox}

Copies \emph{length} bytes starting at CPU address \emph{ramaddr} into XRAM
starting at \emph{xramoffset} within the active bank.

\subsection*{Raw \cmd{FETCH} --- XRAM to CPU RAM}

\begin{retrobox}
\texttt{FETCH ramaddr,xramoffset,length}
\end{retrobox}

Copies \emph{length} bytes from XRAM at \emph{xramoffset} back to CPU RAM starting
at \emph{ramaddr}. The destination must be within writable RAM
(\texttt{\$0000}--\texttt{\$BFFF}).

\subsection*{Example: saving and restoring a character screen buffer}

The character screen RAM occupies \texttt{\$AA00}--\texttt{\$B1CF} (2000 bytes).
You can snapshot it to XRAM bank 0 at offset 0 and restore it later:

\begin{retrobox}
\texttt{10 REM Save character screen to XRAM}\\
\texttt{20 XBANK 0}\\
\texttt{30 STASH 43520,0,2000}\\
\texttt{40 PRINT "SCREEN STASHED."}\\
\texttt{50 REM ... do other things ...}\\
\texttt{60 REM Restore character screen from XRAM}\\
\texttt{70 FETCH 43520,0,2000}\\
\texttt{80 PRINT "SCREEN RESTORED."}
\end{retrobox}

\begin{notebox}
\texttt{\$AA00} = 43520 decimal. The color RAM at \texttt{\$B1D0} is an additional
2000 bytes; stash it at offset 2000 in the same bank to capture the full display
state including colors.
\end{notebox}

%% ============================================================
\section{Named Blocks}
%% ============================================================

Raw offsets require you to track where each block lives. The named-block interface
lets you store data under a string name and retrieve it without knowing its physical
XRAM address.

\subsection*{Storing a named block: \cmd{STASH "name",ramaddr,length}}

\begin{retrobox}
\texttt{STASH "MYDATA",2048,16}
\end{retrobox}

Allocates a region in XRAM, copies 16 bytes from CPU address 2048 into it, and
registers the block under the name \texttt{MYDATA}. Names are 1 to 28 characters
and are case-insensitive. If a block with that name already exists and is large
enough for the new data, it is overwritten in place. If the existing block is too
small, it is freed and a new one is allocated automatically.

\subsection*{Loading a named block: \cmd{FETCH "name",ramaddr}}

\begin{retrobox}
\texttt{FETCH "MYDATA",2048}
\end{retrobox}

Finds the named block and copies its full contents back to CPU RAM at
\emph{ramaddr}. No length argument is required; the XMC remembers the exact size.

\subsection*{Listing named blocks: \cmd{XDIR}}

\begin{retrobox}
\texttt{XDIR}
\end{retrobox}

Prints all named blocks sorted alphabetically (case-insensitive), showing each
name and its stored size in bytes.

\subsection*{Deleting a named block: \cmd{XDEL "name"}}

\begin{retrobox}
\texttt{XDEL "MYDATA"}
\end{retrobox}

Removes the named block and releases its XRAM pages back to the free pool.

\subsection*{Complete example: named-block round-trip}

\begin{retrobox}
\texttt{10 FOR I=0 TO 15:POKE 2048+I,I+10:NEXT I}\\
\texttt{20 STASH "MYDATA",2048,16}\\
\texttt{30 FOR I=0 TO 15:POKE 2048+I,0:NEXT I}\\
\texttt{40 FETCH "MYDATA",2048}\\
\texttt{50 FOR I=0 TO 15:PRINT PEEK(2048+I);:NEXT I}\\
\texttt{60 XDEL "MYDATA"}
\end{retrobox}

Line 10 writes values 10 through 25 into RAM at address 2048. Line 20 stashes them
to XRAM under the name \texttt{MYDATA}. Line 30 zeroes the same RAM region to
confirm the data is no longer in CPU RAM. Line 40 fetches the block back. Line 50
prints the restored values --- you should see
\texttt{10 11 12 13 14 15 16 17 18 19 20 21 22 23 24 25}. Line 60 cleans up.

%% ============================================================
\section{Low-Level Allocation}
%% ============================================================

For programs that need to manage XRAM regions dynamically without names, two
lower-level commands are available.

\subsection*{\cmd{XALLOC length}}

Allocates \emph{length} bytes in XRAM and returns a numeric handle (1--255). The
handle identifies the allocated region. You are responsible for recording it; the
XMC does not associate a name.

\begin{retrobox}
\texttt{H = XALLOC(4096)}\\
\texttt{PRINT "HANDLE: ";H}
\end{retrobox}

\subsection*{\cmd{XFREE offset,length}}

Releases the XRAM range starting at \emph{offset} with the given \emph{length}.
Any named or unnamed blocks that overlap that range are removed from the allocation
table.

\begin{warningbox}
\textbf{XRESET} clears all allocation tracking in one step --- named blocks,
unnamed allocations, and page usage records are all discarded and cannot be
recovered. The raw bytes in XRAM are not erased, but every name and handle is gone.
Use \cmd{XRESET} only when you want a clean slate, such as at program startup.
\end{warningbox}

%% ============================================================
\section{Memory Windows}
%% ============================================================

For the highest-performance XRAM access --- or for assembly code that needs to use
ordinary load and store instructions against XRAM --- you can map an XRAM page
directly into the CPU address space using a \textbf{window}.

There are four windows, each exactly 256 bytes wide, at fixed CPU addresses:

\begin{center}
\begin{tabular}{lll}
\textbf{Window} & \textbf{CPU Address Range} & \textbf{Size} \\
\hline
0 & \texttt{\$BC00}--\texttt{\$BCFF} & 256 bytes \\
1 & \texttt{\$BD00}--\texttt{\$BDFF} & 256 bytes \\
2 & \texttt{\$BE00}--\texttt{\$BEFF} & 256 bytes \\
3 & \texttt{\$BF00}--\texttt{\$BFFF} & 256 bytes \\
\end{tabular}
\end{center}

When a window is mapped, any \cmd{PEEK} or \cmd{POKE} to its CPU address range reads
or writes directly into XRAM. No transfer command is needed.

\subsection*{\cmd{XMAP window,offset}}

Maps the 256-byte XRAM page that contains \emph{offset} into the specified window
(0--3). The offset is rounded down to a 256-byte page boundary automatically.

\begin{retrobox}
\texttt{10 REM Map XRAM page 0 into window 0 at \$BC00}\\
\texttt{20 XMAP 0,0}\\
\texttt{30 FOR I=0 TO 255}\\
\texttt{40 POKE 48128+I,I}\\
\texttt{50 NEXT I}\\
\texttt{60 PRINT PEEK(48128);" ";PEEK(48255)}
\end{retrobox}

\texttt{\$BC00} = 48128 decimal. After \cmd{XMAP 0,0}, writes to
\texttt{\$BC00}--\texttt{\$BCFF} go directly into XRAM. The \cmd{PRINT} on
line 60 should show \texttt{0 255}.

\subsection*{\cmd{XUNMAP window}}

Unmaps the specified window. Reads and writes to that CPU address range no longer
reach XRAM.

\begin{retrobox}
\texttt{XUNMAP 0}
\end{retrobox}

\begin{warningbox}
Mapped windows are shared address space visible to both BASIC and any running
assembly code. If two windows are mapped to the same XRAM page, or if BASIC and
assembly both access the same window, coordinate ownership carefully to avoid
corruption. Unmap windows you are finished with.
\end{warningbox}

%% ============================================================
\section{Error Codes}
%% ============================================================

When an XRAM command fails, NovaBASIC reports an error derived from the XMC error
register. The complete set:

\begin{center}
\begin{tabular}{ll}
\textbf{Code} & \textbf{Meaning} \\
\hline
0 & No error \\
1 & Address out of range \\
2 & Bad arguments \\
3 & Named block not found \\
4 & No space available \\
5 & Invalid name \\
6 & End of directory \\
\end{tabular}
\end{center}

%% ============================================================
\section{Try It Now}
%% ============================================================

\begin{retrobox}
\textbf{Named block round-trip}\\[4pt]
\texttt{10 FOR I=0 TO 15:POKE 2048+I,I+10:NEXT I}\\
\texttt{20 STASH "MYDATA",2048,16}\\
\texttt{30 FOR I=0 TO 15:POKE 2048+I,0:NEXT I}\\
\texttt{40 FETCH "MYDATA",2048}\\
\texttt{50 FOR I=0 TO 15:PRINT PEEK(2048+I);:NEXT I}\\
\texttt{60 XDEL "MYDATA"}
\end{retrobox}

Expected result: line 50 prints \texttt{10 11 12 13 14 15 16 17 18 19 20 21 22 23 24 25},
confirming the data survived the round-trip through XRAM and back.

\chapter{Assembly and Special Chips}

\begin{retrobox}
\textit{``To understand what the machine is actually doing, you have to speak \\
its language --- and that language is always closer to the metal \\
than any other abstraction on top of it.''}
\end{retrobox}

\bigskip

NovaBASIC is built on a 6502 core and runs on top of a set of memory-mapped
coprocessors. Most programs never need to touch hardware registers directly, but
knowing where everything lives gives you the full picture: why certain address ranges
are reserved, what BASIC commands actually do at the hardware level, and how to write
assembly routines that cooperate cleanly with the BASIC runtime.

%% ============================================================
\section{The Memory Map}
%% ============================================================

The full 64 KB address space is partitioned as follows:

\begin{center}
\begin{tabular}{llll}
\textbf{Range} & \textbf{Size} & \textbf{Purpose} & \textbf{Access} \\
\hline
\texttt{\$0000}--\texttt{\$00FF} & 256 B  & Zero Page                       & R/W \\
\texttt{\$0100}--\texttt{\$01FF} & 256 B  & Stack                           & R/W \\
\texttt{\$0200}--\texttt{\$027F} & 128 B  & System Vectors                  & R/W \\
\texttt{\$0280}--\texttt{\$9FFF} & ~39 KB & BASIC RAM                       & R/W \\
\texttt{\$A000}--\texttt{\$A01E} &  31 B  & VGC Registers                   & R/W \\
\texttt{\$A100}--\texttt{\$A1FF} & 256 B  & Sound Controller (VSC)          & R/W \\
\texttt{\$AA00}--\texttt{\$B1CF} & 2000 B & Character RAM (80$\times$25)    & R/W \\
\texttt{\$B1D0}--\texttt{\$B99F} & 2000 B & Color RAM (80$\times$25)        & R/W \\
\texttt{\$B9A0}--\texttt{\$B9EF} &  80 B  & File I/O Controller (FIO)       & R/W \\
\texttt{\$BA00}--\texttt{\$BA3F} &  64 B  & Expansion Memory Controller (XMC) & R/W \\
\texttt{\$BC00}--\texttt{\$BFFF} & 1024 B & XRAM Windows (when mapped)      & R/W \\
\texttt{\$C000}--\texttt{\$FFFF} &  16 KB & ROM (NovaBASIC)                 & R only \\
\end{tabular}
\end{center}

Everything from \texttt{\$A000} upward through \texttt{\$BFFF} is hardware I/O or
managed window space. Writing to ROM (\texttt{\$C000}+) has no effect.

\subsection*{System Vectors at \texttt{\$0200}}

At boot, NovaBASIC initializes a vector table at page \texttt{\$02} with the base
addresses of each hardware controller. Assembly code can read these rather than
hard-coding addresses, so programs remain compatible if the memory map is adjusted
in a future ROM version:

\begin{center}
\begin{tabular}{lll}
\textbf{Address} & \textbf{Value} & \textbf{Meaning} \\
\hline
\texttt{\$0200}--\texttt{\$0201} & \texttt{\$A000} & VGC base \\
\texttt{\$0202}--\texttt{\$0203} & \texttt{\$A010} & VGC command register \\
\texttt{\$0204}--\texttt{\$0205} & \texttt{\$AA00} & Character RAM base \\
\texttt{\$0206}--\texttt{\$0207} & \texttt{\$B1D0} & Color RAM base \\
\texttt{\$0208}--\texttt{\$0209} & \texttt{\$A100} & VSC base \\
\texttt{\$020A}--\texttt{\$020B} & \texttt{\$B9A0} & FIO base \\
\texttt{\$020C}--\texttt{\$020D} & \texttt{\$BA00} & XMC base \\
\end{tabular}
\end{center}

Each entry is a 16-bit little-endian address stored at the indicated pair.

%% ============================================================
\section{Talking to Hardware from BASIC}
%% ============================================================

Because the hardware controllers are memory-mapped, you can read and write their
registers with ordinary \cmd{PEEK} and \cmd{POKE} calls from BASIC. This is the
simplest way to experiment with hardware state or build lightweight diagnostic tools.

\subsection*{Reading the frame counter}

The VGC increments a frame counter register at \texttt{\$A008} on every display
frame. Reading it gives you a running frame tick useful for timing and animation:

\begin{retrobox}
\texttt{PRINT PEEK(40968)}
\end{retrobox}

\texttt{\$A008} = 40968 decimal. The counter wraps from 255 to 0 (it is an 8-bit
register).

\subsection*{Checking the active sprite count}

\begin{retrobox}
\texttt{PRINT "SPRITES: ";PEEK(40969)}
\end{retrobox}

\texttt{\$A009} = 40969 decimal. This read-only register holds the number of
currently enabled sprites.

\subsection*{Reading keyboard input}

\begin{retrobox}
\texttt{K = PEEK(40975)}
\end{retrobox}

\texttt{\$A00F} = 40975 decimal. This is the VGC character input register. Reading
it returns the last character received, or 0 if none. BASIC's own \cmd{INKEY\$}
command uses the same register.

\subsection*{Character output}

Writing a character code to \texttt{\$A00E} (40974 decimal) emits that character to
the current cursor position, exactly as BASIC's \cmd{PRINT} does internally:

\begin{retrobox}
\texttt{POKE 40974,65}\\
\texttt{REM prints the letter A}
\end{retrobox}

%% ============================================================
\section{The CALL and USR Interface}
%% ============================================================

NovaBASIC provides two ways to execute machine code from within a BASIC program.

\subsection*{\cmd{CALL addr}}

Performs a \texttt{JSR} to \emph{addr}. Execution resumes in BASIC after the
machine-code routine executes an \texttt{RTS}. There is no parameter passing;
\cmd{CALL} is a simple subroutine jump. You are responsible for preserving
CPU registers if the routine will return to BASIC in a clean state.

\begin{retrobox}
\texttt{10 CALL 49152}\\
\texttt{REM jumps to machine code at \$C000}
\end{retrobox}

\subsection*{\cmd{USR(x)}}

Calls a user-defined machine-code routine, passing the numeric value \emph{x}
through the 6502 floating-point accumulator (FAC). The routine can read, modify,
and return a value through the same register. \cmd{USR(x)} is a numeric function
and its result can be used in an expression:

\begin{retrobox}
\texttt{10 V = USR(42)}\\
\texttt{20 PRINT "RESULT: ";V}
\end{retrobox}

The address of the USR routine is set by storing a 16-bit pointer in the
appropriate zero-page location before calling. Consult the EhBASIC 2.22
documentation for the exact zero-page addresses used by the USR vector.

%% ============================================================
\section{Interrupts}
%% ============================================================

The 6502 supports two interrupt lines: the maskable IRQ and the non-maskable NMI.
NovaBASIC lets you handle both from within a BASIC program, which is useful for
writing interrupt-driven input handlers, timing routines, and co-operative
multitasking sketches.

\subsection*{Setting up a handler: \cmd{IRQ linenumber} and \cmd{NMI linenumber}}

\begin{retrobox}
\texttt{10 IRQ 1000}\\
\texttt{20 NMI 2000}
\end{retrobox}

When an IRQ fires, execution branches to line 1000. When an NMI fires, execution
branches to line 2000. The handlers are ordinary BASIC subroutines.

\subsection*{Returning from a handler: \cmd{RETIRQ} and \cmd{RETNMI}}

The last statement in an IRQ handler must be \cmd{RETIRQ}; the last statement in
an NMI handler must be \cmd{RETNMI}. These are not interchangeable with
\cmd{RETURN} --- they restore the correct CPU state and re-enable the interrupt
flag.

\begin{retrobox}
\texttt{1000 REM IRQ handler}\\
\texttt{1010 PRINT "IRQ FIRED"}\\
\texttt{1020 RETIRQ}\\[4pt]
\texttt{2000 REM NMI handler}\\
\texttt{2010 PRINT "NMI FIRED"}\\
\texttt{2020 RETNMI}
\end{retrobox}

\begin{warningbox}
Interrupt handlers run in the context of the BASIC interpreter. Keep them short.
Avoid file I/O, heavy computation, or anything that re-enters the interpreter in
an unexpected state. Long handlers can cause instability.
\end{warningbox}

%% ============================================================
\section{XMC Assembly Helpers}
%% ============================================================

The NovaBASIC ROM exports a set of labelled helper routines for accessing the XMC
from assembly code. Using these helpers instead of writing to XMC registers directly
keeps your code clean and gives you error detection for free.

All helpers follow the same convention: on return, \textbf{carry clear} means
success and \textbf{carry set} means an error occurred, with the XMC error code
in the accumulator.

\begin{center}
\begin{tabular}{lp{8cm}}
\textbf{Label} & \textbf{Purpose} \\
\hline
\texttt{LAB\_XM\_SETADDR} & Set the 24-bit XRAM address: A = low byte, X = mid byte, Y = high byte. \\
\texttt{LAB\_XM\_STATUS}  & Read status snapshot: A = status register, X = error code. \\
\texttt{LAB\_XM\_GETBYTE} & Read byte at current XADDR: A = value on success. \\
\texttt{LAB\_XM\_PUTBYTE} & Write byte at current XADDR: A = value to write. \\
\texttt{LAB\_XM\_STASH}   & Bulk copy RAM to XRAM (preload XMC\_RAML/H, XMC\_LENL/H). \\
\texttt{LAB\_XM\_FETCH}   & Bulk copy XRAM to RAM (preload XMC\_RAML/H, XMC\_LENL/H). \\
\texttt{LAB\_XM\_FILL}    & Fill XRAM range with a byte value (preload XMC\_DATA, XMC\_LENL/H). \\
\texttt{LAB\_XM\_ALLOC}   & Allocate a block: preload XMC\_LENL/H; XADDR and handle returned in registers. \\
\end{tabular}
\end{center}

\subsection*{Example: reading one byte from XRAM in assembly}

\begin{retrobox}
\texttt{; Set 24-bit address 0x010000 (bank 1, offset 0)}\\
\texttt{LDA \#\$00}\\
\texttt{LDX \#\$00}\\
\texttt{LDY \#\$01}\\
\texttt{JSR LAB\_XM\_SETADDR}\\[4pt]
\texttt{JSR LAB\_XM\_GETBYTE}\\
\texttt{BCS error}\\
\texttt{; A now holds the byte value}
\end{retrobox}

\begin{notebox}
The ROM helper labels are defined in the NovaBASIC assembly source
(\texttt{ehbasic/basic.asm}). If you assemble custom ROM extensions or overlays,
link against the same symbol file to pick up these addresses.
\end{notebox}

%% ============================================================
\section{VGC Register-Level Programming}
%% ============================================================

The VGC command pipeline works by writing parameters to registers
\texttt{\$A011}--\texttt{\$A01E}, then writing the command byte to \texttt{\$A010}.
The VGC executes the command synchronously. This is exactly what every graphics
BASIC command does under the hood.

From assembly you can issue any VGC command directly:

\begin{retrobox}
\texttt{; Issue GCLS command (clear graphics layer)}\\
\texttt{LDA \#\$07}\\
\texttt{STA \$A010}
\end{retrobox}

The full VGC command code reference, including copper commands
(\$1B--\$1E) and memory I/O commands (\$19--\$1A), is documented in
Appendix~\ref{chap:memmap}.

%% ============================================================
\section{SID Chip Access}
%% ============================================================

The SID chip registers at \$D400--\$D41C are write-intercepted within the ROM
address range. Assembly code can write to them directly:

\begin{retrobox}
\texttt{; Set voice 0 to sawtooth waveform, gate on}\\
\texttt{LDA \#\$21}\\
\texttt{STA \$D404}
\end{retrobox}

The SID register layout matches the MOS 6581. Per-voice registers occupy 7
bytes each (voice~0 at \$D400, voice~1 at \$D407, voice~2 at \$D40E). Filter
and volume registers are at \$D415--\$D418. See Appendix~\ref{chap:memmap}
for the full register map.

\begin{warningbox}
The BASIC \cmd{INSTRUMENT}, \cmd{SOUND}, and \cmd{MUSIC} commands manage
SID registers automatically. Direct SID register writes from assembly will
conflict with the music engine unless you stop all music and SFX first.
\end{warningbox}

%% ============================================================
\section{Copper Programming from Assembly}
%% ============================================================

From BASIC, the \texttt{COPPER} keyword provides high-level access
(see Chapter~\ref{chap:graphics}). From assembly, write parameters to
\$A011--\$A016 and the command code to \$A010:

\begin{retrobox}
\texttt{; Add copper event: at Y=50, X=0, set BgCol to color 5}\\
\texttt{LDA \#\$00 : STA \$A011}\\
\texttt{LDA \#\$00 : STA \$A012}\\
\texttt{LDA \#50  : STA \$A013}\\
\texttt{LDA \#\$01 : STA \$A014}\\
\texttt{LDA \#\$00 : STA \$A015}\\
\texttt{LDA \#\$05 : STA \$A016}\\
\texttt{LDA \#\$1B : STA \$A010}\\[4pt]
\texttt{; Enable copper}\\
\texttt{LDA \#\$1D : STA \$A010}
\end{retrobox}

See the Copper section in Chapter~\ref{chap:graphics} for a full explanation
of writable registers, the COPPER keyword syntax, and programming patterns.

%% ============================================================
\section{DMA Controller from Assembly}
\label{sec:asm-dma}
%% ============================================================

The DMA controller at \$BA60 transfers data between six unified memory
spaces. The pattern is: load parameters into registers, then write \$01
to the command register to start.

\begin{lstlisting}[style=basiclisting]
10 REM -- DMA copy 2000 bytes from CPU $6000 to char RAM --
20 POKE $BA63, 0   : REM source space = CPU RAM
30 POKE $BA64, 1   : REM dest space = Char RAM
40 POKE $BA65, 0   : POKE $BA66, $60 : POKE $BA67, 0
50 REM source addr = $6000 (low, mid, high)
60 POKE $BA68, 0   : POKE $BA69, 0  : POKE $BA6A, 0
70 REM dest addr = 0 (start of char RAM)
80 POKE $BA6B, $D0 : POKE $BA6C, $07 : POKE $BA6D, 0
90 REM length = 2000 ($07D0)
100 POKE $BA6E, 0  : REM mode = copy (not fill)
110 POKE $BA60, 1  : REM start!
120 IF PEEK($BA61) = 1 THEN 120 : REM poll until not busy
130 IF PEEK($BA61) <> 2 THEN PRINT "Error:"; PEEK($BA62)
\end{lstlisting}

For fill mode, set bit 0 of \texttt{DmaMode} (\$BA6E) and load the fill byte
into \texttt{DmaFillValue} (\$BA6F).

%% ============================================================
\section{Blitter from Assembly}
\label{sec:asm-blitter}
%% ============================================================

The blitter at \$BA80 performs 2D rectangular copies and fills with row
stride. Set up source and destination addresses, width, height, and stride,
then write \$01 to \texttt{BltCmd}.

\begin{lstlisting}[style=basiclisting]
10 REM -- Scroll color RAM up by 1 row using blitter --
20 REM Source: row 1 (offset 80), Dest: row 0 (offset 0)
30 REM Width: 80, Height: 24, Stride: 80
40 POKE $BA83, 2  : POKE $BA84, 2   : REM src/dst = color RAM
50 POKE $BA85, 80 : POKE $BA86, 0   : POKE $BA87, 0
60 REM source offset = 80 (row 1)
70 POKE $BA88, 0  : POKE $BA89, 0   : POKE $BA8A, 0
80 REM dest offset = 0 (row 0)
90 POKE $BA8B, 80 : POKE $BA8C, 0   : REM width = 80
100 POKE $BA8D, 24 : POKE $BA8E, 0  : REM height = 24
110 POKE $BA8F, 80 : POKE $BA90, 0  : REM src stride = 80
120 POKE $BA91, 80 : POKE $BA92, 0  : REM dst stride = 80
130 POKE $BA93, 0  : REM mode = copy
140 POKE $BA80, 1  : REM start!
150 IF PEEK($BA81) = 1 THEN 150
\end{lstlisting}

Color-key mode: set bit 1 of \texttt{BltMode} (\$BA93) and load the
transparent color into \texttt{BltColorKey} (\$BA95). Source pixels
matching the color key are skipped.

%% ============================================================
\section{Network Controller from Assembly}
\label{sec:asm-nic}
%% ============================================================

The NIC at \$A100 provides TCP networking. To connect as a client from
register-level code:

\begin{lstlisting}[style=basiclisting]
10 REM -- Connect to 127.0.0.1 port 8080 --
20 POKE $A102, 0          : REM slot 0
30 REM Write hostname to name buffer ($A120+)
40 H$ = "127.0.0.1"
50 FOR I = 1 TO LEN(H$)
60   POKE $A11F + I, ASC(MID$(H$, I, 1))
70 NEXT I
80 POKE $A11F + LEN(H$) + 1, 0 : REM null terminate
90 POKE $A108, $90 : POKE $A109, $1F : REM port 8080
100 POKE $A100, 1  : REM connect command
110 REM Poll slot status for connected bit
120 IF (PEEK($A118) AND 1) = 0 THEN 120
\end{lstlisting}

To send data, copy the message into CPU RAM, then set \texttt{NicDmaAddrL/H}
and \texttt{NicDmaLen}, and write \$03 to \texttt{NicCmd}. To receive, write
\$04 and read \texttt{NicMsgLen} for the actual length.

%% ============================================================
\section{Timer from Assembly}
\label{sec:asm-timer}
%% ============================================================

The timer at \$BA40 fires an IRQ every \textit{N} video frames.
To set up a 1-second timer (60 frames):

\begin{lstlisting}[style=basiclisting]
10 IRQ 1000
20 POKE $BA42, 60 : POKE $BA43, 0  : REM divisor = 60
30 POKE $BA40, 1                    : REM enable timer
40 GOTO 40
1000 PRINT "TICK ";
1010 RETIRQ
\end{lstlisting}

Reading \texttt{TimerStatus} (\$BA41) clears the pending IRQ flag. The
timer must be disabled with \cmd{POKE \$BA40, 0} before changing the divisor.

%% ============================================================
\section{Try It Now}
%% ============================================================

\begin{retrobox}
\textbf{Read the frame counter via PEEK}\\[4pt]
\texttt{10 REM Read frame counter via PEEK}\\
\texttt{20 FOR I=1 TO 60:VSYNC:NEXT I}\\
\texttt{30 PRINT "FRAMES: ";PEEK(40968)}
\end{retrobox}

\texttt{\$A008} = 40968 decimal. After 60 \cmd{VSYNC} waits (approximately one
second at 60 Hz), the frame counter register reflects the elapsed frame ticks.
The value will be somewhere in the range 0--255 because the counter wraps after
256 frames. Run the program several times and observe how the value changes.


% ---------------------------------------------------------------------------
% Appendices
% ---------------------------------------------------------------------------
\appendix

% =============================================================================
% Appendix: Command Reference
% NovaBASIC v1.0 User Guide
% Source of truth: ehbasic/basic.asm token tables (TK_* / XTK_*)
%                  e6502.Avalonia/Hardware/ (VGC, VSC, FIO, XMC implementations)
% =============================================================================
\chapter{Command Reference}

% -----------------------------------------------------------------------------
\section{NovaBASIC Command Quick Reference}
% -----------------------------------------------------------------------------

The tables below list every NovaBASIC statement in syntax order.
Arguments shown in \texttt{[brackets]} are optional.
The \cmd{LET} keyword is always optional before an assignment.

% ---- Program Control --------------------------------------------------------
\subsection*{Program Control}

\begin{longtable}{>{\ttfamily\raggedright\arraybackslash}p{0.38\textwidth} p{0.52\textwidth}}
\toprule
\normalfont\textbf{Syntax} & \textbf{Purpose} \\
\midrule
\endhead
\bottomrule
\endfoot
RUN                          & Execute program from the lowest line number. \\
LIST {[}start{[}-end{]}{]}   & Display program lines; omit range to list all. \\
NEW                          & Clear program and all variables from memory. \\
CONT                         & Continue execution after \cmd{STOP} or Ctrl-C. \\
END                          & Terminate program and return to direct mode. \\
STOP                         & Break execution; \cmd{CONT} resumes at next statement. \\
CLEAR                        & Clear all variables and arrays; program is retained. \\
\end{longtable}

% ---- Flow Control -----------------------------------------------------------
\subsection*{Flow Control}

\begin{longtable}{>{\ttfamily\raggedright\arraybackslash}p{0.38\textwidth} p{0.52\textwidth}}
\toprule
\normalfont\textbf{Syntax} & \textbf{Purpose} \\
\midrule
\endhead
\bottomrule
\endfoot
GOTO line                           & Jump unconditionally to a line number. \\
GOSUB line                          & Call subroutine at line; return address is stacked. \\
RETURN                              & Return from the most recent \cmd{GOSUB}. \\
FOR var=start TO end {[}STEP n{]}   & Begin a counted loop; step defaults to 1. \\
NEXT var                            & Advance loop variable and branch back if not done. \\
IF expr THEN \ldots{} {[}ELSE \ldots{]}  & Conditional execution; \cmd{ELSE} is optional. \\
ON expr GOTO l1,l2,\ldots{}         & Branch to the \textit{n}th line in the list. \\
ON expr GOSUB l1,l2,\ldots{}        & Call the \textit{n}th subroutine in the list. \\
DO                                  & Begin an indefinite loop body. \\
LOOP {[}WHILE expr{]}               & End loop; re-enter while \textit{expr} is true. \\
LOOP {[}UNTIL expr{]}               & End loop; re-enter until \textit{expr} becomes true. \\
\end{longtable}

% ---- Variables and Data -----------------------------------------------------
\subsection*{Variables and Data}

\begin{longtable}{>{\ttfamily\raggedright\arraybackslash}p{0.38\textwidth} p{0.52\textwidth}}
\toprule
\normalfont\textbf{Syntax} & \textbf{Purpose} \\
\midrule
\endhead
\bottomrule
\endfoot
LET var=expr                & Assign value to variable (\cmd{LET} is optional). \\
DIM var(size)               & Declare a single-dimension array. \\
DATA val,val,\ldots{}       & Embed constant values for \cmd{READ}. \\
READ var,var,\ldots{}       & Read successive values from \cmd{DATA} statements. \\
RESTORE {[}line{]}          & Reset the \cmd{DATA} pointer, optionally to a line. \\
SWAP var1,var2              & Exchange the values of two variables. \\
INC var                     & Increment a numeric variable by 1. \\
DEC var                     & Decrement a numeric variable by 1. \\
\end{longtable}

% ---- Input / Output ---------------------------------------------------------
\subsection*{Input / Output}

\begin{longtable}{>{\ttfamily\raggedright\arraybackslash}p{0.38\textwidth} p{0.52\textwidth}}
\toprule
\normalfont\textbf{Syntax} & \textbf{Purpose} \\
\midrule
\endhead
\bottomrule
\endfoot
PRINT expr{[};expr\ldots{}{]}           & Output values to screen; \cmd{;} suppresses newline. \\
INPUT {[}"prompt";{]}var                & Display optional prompt and read from keyboard. \\
GET var                                 & Read a single keypress into variable (non-blocking). \\
WIDTH n                                 & Set the output line width in characters. \\
\end{longtable}

% ---- Memory -----------------------------------------------------------------
\subsection*{Memory}

\begin{longtable}{>{\ttfamily\raggedright\arraybackslash}p{0.38\textwidth} p{0.52\textwidth}}
\toprule
\normalfont\textbf{Syntax} & \textbf{Purpose} \\
\midrule
\endhead
\bottomrule
\endfoot
POKE addr,val   & Write an 8-bit byte to a 6502 address. \\
DOKE addr,val   & Write a 16-bit word (little-endian) to a 6502 address. \\
PEEK(addr)      & Read an 8-bit byte from a 6502 address (function). \\
DEEK(addr)      & Read a 16-bit word (little-endian) from a 6502 address (function). \\
\end{longtable}

% ---- Bit Operations ---------------------------------------------------------
\subsection*{Bit Operations}

\begin{longtable}{>{\ttfamily\raggedright\arraybackslash}p{0.38\textwidth} p{0.52\textwidth}}
\toprule
\normalfont\textbf{Syntax} & \textbf{Purpose} \\
\midrule
\endhead
\bottomrule
\endfoot
BITSET var,bit      & Set bit \textit{n} (0-based) in an integer variable. \\
BITCLR var,bit      & Clear bit \textit{n} in an integer variable. \\
BITTST(val,bit)     & Return 1 if bit \textit{n} is set in \textit{val}, else 0. \\
\end{longtable}

% ---- File Operations --------------------------------------------------------
\subsection*{File Operations}

\begin{longtable}{>{\ttfamily\raggedright\arraybackslash}p{0.38\textwidth} p{0.52\textwidth}}
\toprule
\normalfont\textbf{Syntax} & \textbf{Purpose} \\
\midrule
\endhead
\bottomrule
\endfoot
SAVE "name"     & Save the current BASIC program to disk as \texttt{name.bas}. \\
LOAD "name"     & Load a previously saved program from disk. \\
DIR             & List all saved \texttt{.bas} programs. \\
DEL "name"      & Delete a saved program from disk. \\
\end{longtable}

% ---- Text Display -----------------------------------------------------------
\subsection*{Text Display}

\begin{longtable}{>{\ttfamily\raggedright\arraybackslash}p{0.38\textwidth} p{0.52\textwidth}}
\toprule
\normalfont\textbf{Syntax} & \textbf{Purpose} \\
\midrule
\endhead
\bottomrule
\endfoot
CLS             & Clear the text screen and home the cursor. \\
COLOR fg{[},bg{]}  & Set foreground and optional background color (0--15). \\
LOCATE x,y      & Move the text cursor to column \textit{x} (0--79), row \textit{y} (0--24). \\
\end{longtable}

% ---- Graphics ---------------------------------------------------------------
\subsection*{Graphics}

The graphics layer is 320$\times$200 pixels.
Colors are indices 0--15; color 0 is transparent on the graphics layer.

\begin{longtable}{>{\ttfamily\raggedright\arraybackslash}p{0.38\textwidth} p{0.52\textwidth}}
\toprule
\normalfont\textbf{Syntax} & \textbf{Purpose} \\
\midrule
\endhead
\bottomrule
\endfoot
MODE n                  & Display mode: 0=text only, 1=graphics over text, 2=text over graphics, 3=graphics+sprites only (no text). \\
GCLS                    & Clear the graphics layer to transparent (all pixels 0). \\
GCOLOR c                & Set the graphics draw color (0--15; only the low nibble is used). \\
PLOT x,y                & Set pixel at (\textit{x},\textit{y}) to the current draw color. \\
UNPLOT x,y              & Clear pixel at (\textit{x},\textit{y}) to 0 (transparent). \\
LINE x0,y0,x1,y1        & Draw a straight line between two points. \\
RECT x0,y0,x1,y1        & Draw a rectangle outline. \\
FILL x0,y0,x1,y1        & Draw a filled solid rectangle. \\
CIRCLE cx,cy,r          & Draw a circle outline centred at (\textit{cx},\textit{cy}). \\
PAINT x,y               & Flood-fill from seed point (\textit{x},\textit{y}). \\
VSYNC                   & Wait for the next video frame boundary (60 Hz). \\
COPPER ADD x,y,reg,val  & Add copper event: set \textit{reg} to \textit{val} at position (\textit{x},\textit{y}). Registers: BGCOL, MODE, SCROLLX, SCROLLY. \\
COPPER CLEAR            & Remove all events from the current target list. \\
COPPER ON               & Enable copper execution each frame. \\
COPPER OFF              & Disable copper execution. \\
COPPER LIST n           & Set target list to \textit{n} (0--127). Subsequent ADD/CLEAR edit this list. \\
COPPER LIST END         & Reset target list back to the active list. \\
COPPER USE n            & Switch active copper list to \textit{n} at next vblank (double-buffering). \\
\end{longtable}

% ---- Sprites ----------------------------------------------------------------
\subsection*{Sprites}

Sprites are 16$\times$16 pixels, 4-bit multicolor.
Up to 16 sprites are available (indices 0--15).

\begin{longtable}{>{\ttfamily\raggedright\arraybackslash}p{0.38\textwidth} p{0.52\textwidth}}
\toprule
\normalfont\textbf{Syntax} & \textbf{Purpose} \\
\midrule
\endhead
\bottomrule
\endfoot
SPRITE n,ON                     & Enable (show) sprite \textit{n}. \\
SPRITE n,OFF                    & Disable (hide) sprite \textit{n}. \\
SPRITE n,x,y                    & Set sprite \textit{n} screen position. \\
SPRITEDATA n,row,b1\ldots{}b8   & Define one row of sprite shape data (8 bytes, 16 pixels). \\
\end{longtable}

\begin{notebox}
\cmd{SPRITESHAPE} and \cmd{SPRITECOLOR} are recognized tokens (for source compatibility) but their ROM handlers currently perform no hardware action.
\end{notebox}

% ---- Sound and Music --------------------------------------------------------
\subsection*{Sound and Music}

The SID chip provides 3 voices. The music engine adds a three-voice MML
sequencer with instrument presets, tempo, looping, and per-frame effects.

\begin{longtable}{>{\ttfamily\raggedright\arraybackslash}p{0.42\textwidth} p{0.48\textwidth}}
\toprule
\normalfont\textbf{Syntax} & \textbf{Purpose} \\
\midrule
\endhead
\bottomrule
\endfoot
SOUND note,dur{[},inst{]}    & Play MIDI \textit{note} (0--127) for \textit{dur} frames (1/60~s). Optional \textit{inst} selects instrument preset (0--15, default~0). \\[3pt]
VOLUME level                 & Set SID master volume (0--15; low nibble only). \\[3pt]
INSTRUMENT id,wave,a,d,s,r   & Define instrument preset \textit{id} (0--15). \textit{wave}: \$10=tri, \$20=saw, \$40=pulse, \$80=noise. \textit{a,d,s,r}: ADSR values 0--15. \\[3pt]
MUSIC voice,"mml"            & Load MML sequence into \textit{voice} (1--3). \\[3pt]
MUSIC PLAY                   & Start music playback. \\[3pt]
MUSIC STOP                   & Stop music and silence all music voices. \\[3pt]
MUSIC TEMPO bpm              & Set playback tempo (default 120). \\[3pt]
MUSIC LOOP ON                & Enable looping (restart when all voices finish). \\[3pt]
MUSIC LOOP OFF               & Disable looping (default). \\[3pt]
MUSIC PRIORITY v1{[},v2{[},v3{]}} & Set voice-stealing priority for SFX. \\[3pt]
SIDPLAY "name"{[},song{]}   & Play a \texttt{.sid} file; optional \textit{song} number (default~1). \\[3pt]
SIDSTOP                      & Stop SID file playback. \\
\end{longtable}

\begin{warningbox}
\cmd{WAVE} is deprecated and raises a syntax error. Use \cmd{INSTRUMENT} to set waveform and ADSR in a single command.
\end{warningbox}

% ---- Graphics File I/O ------------------------------------------------------
\subsection*{Graphics File I/O}

\begin{longtable}{>{\ttfamily\raggedright\arraybackslash}p{0.42\textwidth} p{0.48\textwidth}}
\toprule
\normalfont\textbf{Syntax} & \textbf{Purpose} \\
\midrule
\endhead
\bottomrule
\endfoot
GSAVE "name",space,offset,len & Save VGC memory to a \texttt{.gfx} file.
  \textit{space}: 0=screen, 1=color, 2=gfx bitmap, 3=sprite shapes. \\[3pt]
GLOAD "name",space,offset{[},len{]} & Load a \texttt{.gfx} file into VGC memory. If \textit{len} is omitted, loads the entire file. \\
\end{longtable}

% ---- Expansion Memory -------------------------------------------------------
\subsection*{Expansion Memory}

XRAM is banked memory outside the 6502 address space, accessed via the XMC coprocessor.
See Chapter~6 for a full programming guide.

\begin{longtable}{>{\ttfamily\raggedright\arraybackslash}p{0.38\textwidth} p{0.52\textwidth}}
\toprule
\normalfont\textbf{Syntax} & \textbf{Purpose} \\
\midrule
\endhead
\bottomrule
\endfoot
XMEM                        & Print XRAM bank count and page usage statistics. \\
XBANK n                     & Select the active 64 KB XRAM bank; \textit{n} must be $<$ total banks. \\
XPOKE offset,value          & Write one byte to XRAM at the given offset in the active bank. \\
XPEEK(offset)               & Read one byte from XRAM at the given offset (function). \\
STASH ram,offset,length     & Copy \textit{length} bytes from CPU RAM to XRAM (raw, no name). \\
FETCH ram,offset,length     & Copy \textit{length} bytes from XRAM to CPU RAM (raw). \\
STASH "name",ram,length     & Store a named XRAM block from CPU RAM. \\
FETCH "name",ram            & Load a named XRAM block back into CPU RAM. \\
XDIR                        & List all named XRAM blocks and their sizes. \\
XDEL "name"                 & Delete a named XRAM block. \\
XALLOC length               & Allocate an unnamed XRAM block; returns offset in \cmd{XPEEK} result. \\
XFREE offset,length         & Release a raw XRAM range from usage tracking. \\
XRESET                      & Clear all XRAM allocation and named-block state (destructive). \\
XMAP window,offset          & Map an XRAM page to CPU window 0--3 (addresses \$BC00--\$BFFF). \\
XUNMAP window               & Unmap a CPU window (0--3). \\
\end{longtable}

\begin{warningbox}
\cmd{XRESET} destroys all named-block metadata and usage-tracking information.
The raw XRAM contents are not zeroed, but all allocation records are lost.
\end{warningbox}

% ---- Interrupts and Machine Code --------------------------------------------
\subsection*{Interrupts and Machine Code}

\begin{longtable}{>{\ttfamily\raggedright\arraybackslash}p{0.38\textwidth} p{0.52\textwidth}}
\toprule
\normalfont\textbf{Syntax} & \textbf{Purpose} \\
\midrule
\endhead
\bottomrule
\endfoot
CALL addr       & Execute machine code subroutine at 6502 address; \cmd{JSR}/\cmd{RTS} pair. \\
IRQ line        & Redirect the IRQ vector to a BASIC line number handler. \\
NMI line        & Redirect the NMI vector to a BASIC line number handler. \\
RETIRQ          & Return from an IRQ handler (re-enables interrupts). \\
RETNMI          & Return from an NMI handler. \\
\end{longtable}

% ---- Miscellaneous ----------------------------------------------------------
\subsection*{Miscellaneous}

\begin{longtable}{>{\ttfamily\raggedright\arraybackslash}p{0.38\textwidth} p{0.52\textwidth}}
\toprule
\normalfont\textbf{Syntax} & \textbf{Purpose} \\
\midrule
\endhead
\bottomrule
\endfoot
REM comment             & Program comment; rest of line is ignored by interpreter. \\
DEF FN name(var)=expr   & Define a single-line user function. \\
WAIT addr,mask{[},xor{]}   & Busy-wait until \texttt{(PEEK(addr) XOR xor) AND mask} is non-zero. \\
NULL n                  & Set the number of null (zero) bytes sent after each carriage return. \\
\end{longtable}

% =============================================================================
\section{Function Reference}
% =============================================================================

Functions return a value and may be used within any expression.
String functions are marked with a \texttt{\$} suffix.

% ---- Numeric Functions ------------------------------------------------------
\subsection*{Numeric Functions}

\begin{longtable}{>{\ttfamily\raggedright\arraybackslash}p{0.32\textwidth} p{0.58\textwidth}}
\toprule
\normalfont\textbf{Function} & \textbf{Returns} \\
\midrule
\endhead
\bottomrule
\endfoot
SGN(n)      & Sign of \textit{n}: $-1$, $0$, or $1$. \\
INT(n)      & Truncate toward zero to integer. \\
ABS(n)      & Absolute value. \\
SQR(n)      & Square root. \\
RND(n)      & Pseudo-random number in $[0, 1)$; \textit{n} seeds or advances the sequence. \\
LOG(n)      & Natural logarithm ($\ln n$). \\
EXP(n)      & $e$ raised to the power \textit{n}. \\
SIN(n)      & Sine of \textit{n} radians. \\
COS(n)      & Cosine of \textit{n} radians. \\
TAN(n)      & Tangent of \textit{n} radians. \\
ATN(n)      & Arctangent of \textit{n}, result in radians. \\
PI          & Constant $\pi \approx 3.14159\ldots$ \\
TWOPI       & Constant $2\pi \approx 6.28318\ldots$ \\
MAX(a,b)    & The larger of two numeric values. \\
MIN(a,b)    & The smaller of two numeric values. \\
FRE(x)      & Free BASIC program memory in bytes (\textit{x} is ignored). \\
POS(x)      & Current text cursor column position (\textit{x} is ignored). \\
USR(x)      & Call user machine-code routine; pass \textit{x} in FAC, return value in FAC. \\
\end{longtable}

% ---- String Functions -------------------------------------------------------
\subsection*{String Functions}

\begin{longtable}{>{\ttfamily\raggedright\arraybackslash}p{0.38\textwidth} p{0.52\textwidth}}
\toprule
\normalfont\textbf{Function} & \textbf{Returns} \\
\midrule
\endhead
\bottomrule
\endfoot
LEN(s\$)                    & Length of string in characters. \\
ASC(s\$)                    & ASCII code of the first character. \\
CHR\$(n)                    & Single-character string for ASCII code \textit{n}. \\
STR\$(n)                    & Numeric value converted to a string. \\
VAL(s\$)                    & Numeric value parsed from the leading digits of a string. \\
LEFT\$(s\$,n)               & First \textit{n} characters of string. \\
RIGHT\$(s\$,n)              & Last \textit{n} characters of string. \\
MID\$(s\$,start{[},len{]})  & Substring starting at \textit{start} (1-based), optional length. \\
UCASE\$(s\$)                & String converted to uppercase. \\
LCASE\$(s\$)                & String converted to lowercase. \\
HEX\$(n)                    & Hexadecimal string representation of integer \textit{n}. \\
BIN\$(n)                    & Binary string representation of integer \textit{n}. \\
SADD(s\$)                   & Address of the string's data in the string pool. \\
VARPTR(var)                 & Address of a numeric or string variable in memory. \\
\end{longtable}

% ---- Sprite and Graphics Functions ------------------------------------------
\subsection*{Sprite and Graphics Functions}

\begin{longtable}{>{\ttfamily\raggedright\arraybackslash}p{0.38\textwidth} p{0.52\textwidth}}
\toprule
\normalfont\textbf{Function} & \textbf{Returns} \\
\midrule
\endhead
\bottomrule
\endfoot
SPRITEX(n)      & X position of sprite \textit{n} (currently always returns 0). \\
SPRITEY(n)      & Y position of sprite \textit{n} (currently always returns 0). \\
COLLISION(n)    & Sprite-to-sprite collision bitmask for sprite \textit{n}. \\
BUMPED(n)       & Sprite-to-background collision bitmask for sprite \textit{n}. \\
\end{longtable}

\begin{notebox}
\cmd{SPRITEX()} and \cmd{SPRITEY()} are present as ROM tokens.
In v1.0 the ROM handlers return 0; position read-back is not yet implemented.
\end{notebox}

% ---- Music Functions --------------------------------------------------------
\subsection*{Music Functions}

\begin{longtable}{>{\ttfamily\raggedright\arraybackslash}p{0.38\textwidth} p{0.52\textwidth}}
\toprule
\normalfont\textbf{Function} & \textbf{Returns} \\
\midrule
\endhead
\bottomrule
\endfoot
PLAYING         & 1 if music is currently playing, 0 if stopped. \\
MNOTE(voice)    & Current MIDI note number on \textit{voice} (1--3), or 0 if silent. \\
\end{longtable}

% =============================================================================
\section{Token Index}
% =============================================================================

This index is derived directly from the \texttt{TK\_*} and \texttt{XTK\_*}
symbol definitions in \texttt{ehbasic/basic.asm}.
Single-byte tokens begin at \texttt{\$80}.
Extended two-byte tokens use a \texttt{\$FF} escape prefix followed by the
\texttt{XTK\_*} byte.

% ---- Primary Statements -----------------------------------------------------
\subsection*{Primary Statement Tokens (\texttt{TK\_END} through \texttt{TK\_NMI})}

\begin{retrobox}
\ttfamily\small
END, FOR, NEXT, DATA, INPUT, DIM, READ, LET, DEC, GOTO, RUN, IF, RESTORE,
GOSUB, RETIRQ, RETNMI, RETURN, REM, STOP, ON, NULL, INC, WAIT, LOAD, SAVE,
DEF, POKE, DOKE, CALL, DO, LOOP, PRINT, CONT, LIST, CLEAR, NEW, WIDTH, GET,
SWAP, BITSET, BITCLR, IRQ, NMI
\end{retrobox}

% ---- Graphics / Sound Statements --------------------------------------------
\subsection*{Graphics and Sound Statement Tokens (\texttt{TK\_CLS} through \texttt{TK\_VSYNC})}

\begin{retrobox}
\ttfamily\small
CLS, COLOR, LOCATE, PLOT, UNPLOT, LINE, CIRCLE, RECT, FILL, PAINT, MODE,
GCLS, GCOLOR, SPRITE, SPRITESHAPE, SPRITECOLOR, SPRITEDATA, SOUND, VOLUME,
INSTRUMENT, WAVE (deprecated), VSYNC
\end{retrobox}

% ---- Secondary Tokens -------------------------------------------------------
\subsection*{Secondary Tokens (\texttt{TK\_TAB} through \texttt{TK\_OFF})}

\begin{retrobox}
\ttfamily\small
TAB, ELSE, TO, FN, SPC, THEN, NOT, STEP, UNTIL, WHILE, OFF
\end{retrobox}

% ---- Operators --------------------------------------------------------------
\subsection*{Operator Tokens}

\begin{retrobox}
\ttfamily\small
+, -, *, /, \textasciicircum{}, AND, EOR, OR, >>{} (RSHIFT), <<{} (LSHIFT), >, =, <
\end{retrobox}

% ---- Functions --------------------------------------------------------------
\subsection*{Function Tokens (\texttt{TK\_SGN} through \texttt{TK\_BUMPED})}

\begin{retrobox}
\ttfamily\small
SGN, INT, ABS, USR, FRE, POS, SQR, RND, LOG, EXP, COS, SIN, TAN, ATN, PEEK,
DEEK, SADD, LEN, STR\$, VAL, ASC, UCASE\$, LCASE\$, CHR\$, HEX\$, BIN\$,
BITTST, MAX, MIN, PI, TWOPI, VARPTR, LEFT\$, RIGHT\$, MID\$, SPRITEX(),
SPRITEY(), COLLISION(), BUMPED()
\end{retrobox}

% ---- Extended 2-Byte Tokens -------------------------------------------------
\subsection*{Extended Two-Byte Tokens (\texttt{\$FF} prefix, \texttt{XTK\_DIR} through \texttt{XTK\_MNOTE})}

\begin{retrobox}
\ttfamily\small
DIR, DEL, XMEM, XBANK, XPOKE, XPEEK(), STASH, FETCH, XFREE, XRESET, XALLOC,
XDIR, XDEL, XMAP, XUNMAP, GSAVE, GLOAD, SIDPLAY, SIDSTOP, MUSIC, PLAYING,
MNOTE()
\end{retrobox}

% =============================================================================
% Appendix: Memory Map
% NovaBASIC v1.0 User Guide
% Source of truth: e6502.Avalonia/Hardware/VgcConstants.cs
%                  e6502.Avalonia/Hardware/VirtualGraphicsController.cs
%                  e6502.Avalonia/Hardware/VirtualSoundController.cs
%                  e6502.Avalonia/Hardware/FileIoController.cs
%                  e6502.Avalonia/Hardware/VirtualExpansionMemoryController.cs
% =============================================================================
\chapter{Memory Map}
\label{chap:memmap}

% =============================================================================
\section{Address Space Overview}
% =============================================================================

The e6502 virtual computer presents a flat 64 KB address space to the 6502 CPU.
Coprocessor regions (VGC, VSC, FIO, XMC) respond to reads and writes within
their assigned windows; all remaining space is RAM except the upper 16 KB
(\$C000--\$FFFF) which is write-protected ROM.

\begin{longtable}{>{\ttfamily\raggedright\arraybackslash}p{0.22\textwidth} >{\raggedleft\arraybackslash}p{0.14\textwidth} p{0.50\textwidth}}
\toprule
\normalfont\textbf{Address Range} & \textbf{Size} & \textbf{Region} \\
\midrule
\endhead
\bottomrule
\endfoot
\$0000--\$00FF &     256 B & Zero Page \\
\$0100--\$01FF &     256 B & CPU Stack \\
\$0200--\$027F &     128 B & System Vectors (IRQ/NMI handlers, BASIC vectors) \\
\$0280--\$9FFF &  39,680 B & BASIC Program RAM \\
\$A000--\$A01E &      31 B & Virtual Graphics Controller (VGC) registers and command interface \\
\$A100--\$A1FF &     256 B & Virtual Sound Controller (VSC) registers \\
\$AA00--\$B1CF &   2,000 B & Character RAM (80$\times$25 text cells) \\
\$B1D0--\$B99F &   2,000 B & Color RAM (80$\times$25 text cells) \\
\$B9A0--\$B9EF &      80 B & File I/O Controller (FIO) registers \\
\$BA00--\$BA3F &      64 B & Expansion Memory Controller (XMC) registers \\
\$BA40--\$BA4F &      16 B & Timer Controller registers \\
\$BA50--\$BA53 &       4 B & Music Status and Voice Note Readback \\
\$BC00--\$BFFF &   1,024 B & XMC Memory Windows (4 $\times$ 256-byte mapped pages) \\
\$C000--\$FFFF &  16,384 B & ROM (NovaBASIC interpreter) \\
\$D400--\$D41C &      29 B & SID chip registers (inside ROM range; writes intercepted) \\
\end{longtable}

\begin{notebox}
The address range \$A01F--\$A0FF and \$A200--\$A9FF are not claimed by any
coprocessor and fall through to the underlying flat RAM.
The range \$BA54--\$BBFF is similarly unallocated RAM.
SID registers at \$D400--\$D41C occupy space within the ROM address range
but are intercepted on write by the SID chip emulator.
\end{notebox}

% =============================================================================
\section{VGC Register Map}
% =============================================================================

The Virtual Graphics Controller occupies \$A000--\$A01E.
Registers \$A000--\$A00F are the core status and display registers.
Registers \$A010--\$A01E are the command register and its 14 parameter slots;
writing to \$A010 both stores the command byte and triggers immediate execution.

\subsection*{Core Registers (\$A000--\$A00F)}

\begin{longtable}{>{\ttfamily\raggedright\arraybackslash}p{0.14\textwidth} >{\ttfamily\raggedright\arraybackslash}p{0.20\textwidth} p{0.10\textwidth} p{0.42\textwidth}}
\toprule
\normalfont\textbf{Address} & \normalfont\textbf{Name} & \textbf{Access} & \textbf{Description} \\
\midrule
\endhead
\bottomrule
\endfoot
\$A000 & RegMode        & R/W & Display mode: 0=text only, 1=graphics over text, 2=text over graphics. \\
\$A001 & RegBgCol       & R/W & Background color index (0--15). \\
\$A002 & RegFgCol       & R/W & Default foreground color index (0--15); reset value is 1 (white). \\
\$A003 & RegCursorX     & R/W & Text cursor column (0--79). \\
\$A004 & RegCursorY     & R/W & Text cursor row (0--24). \\
\$A005 & RegScrollX     & R/W & Horizontal scroll offset (used by copper raster effects). \\
\$A006 & RegScrollY     & R/W & Vertical scroll offset (used by copper raster effects). \\
\$A007 & RegBank        & R/W & Reserved. \\
\$A008 & RegStatus      & RO  & Frame counter; incremented each video frame. Writes are ignored. \\
\$A009 & RegSpriteCount & RO  & Count of currently enabled sprites (0--16). Writes are ignored. \\
\$A00A & RegCursorEna   & R/W & Non-zero enables the cursor blink. \\
\$A00B & RegColSt       & RO  & Sprite-to-sprite collision bitmask; reading clears the register. \\
\$A00C & RegColBg       & RO  & Sprite-to-background collision bitmask; reading clears the register. \\
\$A00D & RegBorder      & R/W & Border color index (0--15). \\
\$A00E & RegCharOut     & R/W & Character output port; writing outputs a character to the text screen. \\
\$A00F & RegCharIn      & R/W & Character input port; reading dequeues the next keypress byte. \\
\end{longtable}

\subsection*{Command Register and Parameters (\$A010--\$A01E)}

\begin{longtable}{>{\ttfamily\raggedright\arraybackslash}p{0.14\textwidth} >{\ttfamily\raggedright\arraybackslash}p{0.20\textwidth} p{0.10\textwidth} p{0.42\textwidth}}
\toprule
\normalfont\textbf{Address} & \normalfont\textbf{Name} & \textbf{Access} & \textbf{Description} \\
\midrule
\endhead
\bottomrule
\endfoot
\$A010 & RegCmd  & R/W & Command byte; writing triggers immediate command execution. \\
\$A011 & RegP0   & R/W & Parameter 0. \\
\$A012 & RegP1   & R/W & Parameter 1. \\
\$A013 & RegP2   & R/W & Parameter 2. \\
\$A014 & RegP3   & R/W & Parameter 3. \\
\$A015 & RegP4   & R/W & Parameter 4. \\
\$A016 & RegP5   & R/W & Parameter 5. \\
\$A017 & RegP6   & R/W & Parameter 6. \\
\$A018 & RegP7   & R/W & Parameter 7. \\
\$A019 & RegP8   & R/W & Parameter 8. \\
\$A01A & RegP9   & R/W & Parameter 9. \\
\$A01B & RegP10  & R/W & Parameter 10. \\
\$A01C & RegP11  & R/W & Parameter 11. \\
\$A01D & RegP12  & R/W & Parameter 12. \\
\$A01E & RegP13  & R/W & Parameter 13. \\
\end{longtable}

Multi-byte parameters (coordinates, sprite positions) are packed
little-endian across consecutive parameter registers.
For example, a 16-bit x-coordinate uses P0 (low byte) and P1 (high byte).

% =============================================================================
\section{VGC Command Codes}
% =============================================================================

All commands are invoked by writing the command byte to \texttt{RegCmd} (\$A010).
Parameters must be loaded into \texttt{RegP0}--\texttt{RegP13} before the write.

\subsection*{Graphics Commands (\$01--\$09)}

\begin{longtable}{>{\ttfamily\raggedright\arraybackslash}p{0.10\textwidth} >{\ttfamily\raggedright\arraybackslash}p{0.18\textwidth} p{0.58\textwidth}}
\toprule
\normalfont\textbf{Code} & \normalfont\textbf{Name} & \textbf{Parameters and behavior} \\
\midrule
\endhead
\bottomrule
\endfoot
\$01 & CmdPlot    & P0/P1 = x (16-bit), P2/P3 = y (16-bit). Set pixel to current draw color. \\
\$02 & CmdUnplot  & P0/P1 = x (16-bit), P2/P3 = y (16-bit). Clear pixel to 0 (transparent). \\
\$03 & CmdLine    & P0/P1 = x0, P2/P3 = y0, P4/P5 = x1, P6/P7 = y1. Draw Bresenham line. \\
\$04 & CmdCircle  & P0/P1 = cx, P2/P3 = cy, P4/P5 = radius. Draw circle outline. \\
\$05 & CmdRect    & P0/P1 = x0, P2/P3 = y0, P4/P5 = x1, P6/P7 = y1. Draw rectangle outline. \\
\$06 & CmdFill    & P0/P1 = x0, P2/P3 = y0, P4/P5 = x1, P6/P7 = y1. Draw filled rectangle. \\
\$07 & CmdGcls    & No parameters. Clear entire graphics bitmap to 0. \\
\$08 & CmdGcolor  & P0 low nibble = color index (0--15). Set current draw color. \\
\$09 & CmdPaint   & P0/P1 = x (16-bit), P2/P3 = y (16-bit). Flood-fill from seed point. \\
\end{longtable}

\subsection*{Sprite Commands (\$10--\$18)}

\begin{longtable}{>{\ttfamily\raggedright\arraybackslash}p{0.10\textwidth} >{\ttfamily\raggedright\arraybackslash}p{0.18\textwidth} p{0.58\textwidth}}
\toprule
\normalfont\textbf{Code} & \normalfont\textbf{Name} & \textbf{Parameters and behavior} \\
\midrule
\endhead
\bottomrule
\endfoot
\$10 & CmdSprDef  & P0 = sprite (0--15), P1 = x pixel (0--15), P2 = y pixel (0--15), P3 = color nibble. Set one pixel in sprite shape. \\
\$11 & CmdSprRow  & P0 = sprite (0--15), P1 = row (0--15), P2--P9 = 8 data bytes (two 4-bit pixels per byte). Define one sprite row. \\
\$12 & CmdSprClr  & P0 = sprite (0--15). Clear all 128 bytes of sprite shape data to 0. \\
\$13 & CmdSprCopy & P0 = source sprite (0--15), P1 = destination sprite (0--15). Copy shape data. \\
\$14 & CmdSprPos  & P0 = sprite (0--15), P1/P2 = x (16-bit), P3/P4 = y (16-bit). Set screen position. \\
\$15 & CmdSprEna  & P0 = sprite (0--15). Enable sprite; increments \texttt{RegSpriteCount}. \\
\$16 & CmdSprDis  & P0 = sprite (0--15). Disable sprite; decrements \texttt{RegSpriteCount}. \\
\$17 & CmdSprFlip & P0 = sprite (0--15), P1 = flags (0=none, 1=horizontal, 2=vertical, 3=both). \\
\$18 & CmdSprPri  & P0 = sprite (0--15), P1 = priority (0=behind all, 1=between text/gfx, 2=in front). \\
\end{longtable}

\begin{tipbox}
Sprite shape data is stored host-side and is not 6502-addressable.
All sprite shape manipulation must go through the command register interface.
\end{tipbox}

\subsection*{Memory I/O Commands (\$19--\$1A)}

\begin{longtable}{>{\ttfamily\raggedright\arraybackslash}p{0.10\textwidth} >{\ttfamily\raggedright\arraybackslash}p{0.20\textwidth} p{0.56\textwidth}}
\toprule
\normalfont\textbf{Code} & \normalfont\textbf{Name} & \textbf{Parameters and behavior} \\
\midrule
\endhead
\bottomrule
\endfoot
\$19 & CmdMemRead  & P0 = memory space (0--3), P1/P2 = address (16-bit), P4 bit 0 = auto-increment. Read byte from VGC memory; result in P3. \\
\$1A & CmdMemWrite & P0 = memory space (0--3), P1/P2 = address (16-bit), P3 = data byte, P4 bit 0 = auto-increment. Write byte to VGC memory. \\
\end{longtable}

Memory spaces: 0=character RAM (2000~B), 1=color RAM (2000~B),
2=graphics bitmap (64000~B), 3=sprite shape RAM (2048~B).
Auto-increment advances the address after each read or write.

\subsection*{Copper Commands (\$1B--\$1E, \$20--\$22)}

The copper triggers register writes at specific raster positions each frame.
The VGC stores 128 independent copper lists (0--127), each holding up to 256
events. All copper state changes take effect at vblank.

\begin{longtable}{>{\ttfamily\raggedright\arraybackslash}p{0.10\textwidth} >{\ttfamily\raggedright\arraybackslash}p{0.22\textwidth} p{0.54\textwidth}}
\toprule
\normalfont\textbf{Code} & \normalfont\textbf{Name} & \textbf{Parameters and behavior} \\
\midrule
\endhead
\bottomrule
\endfoot
\$1B & CmdCopperAdd     & P0/P1 = X (16-bit), P2 = Y, P3/P4 = register (0--15 or \$A000--\$A00F), P5 = value. Adds to the target list. Replaces existing event at same position/register. Max 256 events per list. \\
\$1C & CmdCopperClear   & No parameters. Remove all events from the target list. \\
\$1D & CmdCopperEnable  & No parameters. Start executing the active copper list each frame. \\
\$1E & CmdCopperDisable & No parameters. Stop executing copper. \\
\$20 & CmdCopperList    & P0 = list index (0--127). Set target list for ADD/CLEAR. \\
\$21 & CmdCopperUse     & P0 = list index (0--127). Set pending active list (swaps at next vblank). \\
\$22 & CmdCopperListEnd & No parameters. Reset target list to the currently active list. \\
\end{longtable}

Copper-writable registers: RegMode (\$A000), RegBgCol (\$A001),
RegScrollX (\$A005), RegScrollY (\$A006).

% =============================================================================
\section{SID Chip Registers}
% =============================================================================

The SID chip occupies \$D400--\$D41C within the ROM address range. Writes
to these addresses are intercepted by the SID emulator; reads return the
underlying ROM byte. The register layout matches the original MOS~6581.

\subsection*{Per-Voice Registers (3 voices, 7 bytes each)}

\begin{longtable}{>{\ttfamily\raggedright\arraybackslash}p{0.14\textwidth} p{0.72\textwidth}}
\toprule
\normalfont\textbf{Offset} & \textbf{Description} \\
\midrule
\endhead
\bottomrule
\endfoot
+0 & Frequency low byte. \\
+1 & Frequency high byte (16-bit SID frequency units). \\
+2 & Pulse width low byte. \\
+3 & Pulse width high byte (12-bit, bits 0--11 only). \\
+4 & Control register: bit 0=gate, bit 4=triangle, bit 5=sawtooth, bit 6=pulse, bit 7=noise. \\
+5 & Attack (bits 7--4) / Decay (bits 3--0). \\
+6 & Sustain (bits 7--4) / Release (bits 3--0). \\
\end{longtable}

Voice 0: \$D400--\$D406. Voice 1: \$D407--\$D40D. Voice 2: \$D40E--\$D414.

\subsection*{Filter and Volume Registers}

\begin{longtable}{>{\ttfamily\raggedright\arraybackslash}p{0.14\textwidth} p{0.72\textwidth}}
\toprule
\normalfont\textbf{Address} & \textbf{Description} \\
\midrule
\endhead
\bottomrule
\endfoot
\$D415 & Filter cutoff low (bits 0--2). \\
\$D416 & Filter cutoff high (bits 0--7). \\
\$D417 & Resonance (bits 7--4) / Filter route (bits 3--0, one bit per voice + external). \\
\$D418 & Volume (bits 3--0) / Filter mode (bit 4=LP, bit 5=BP, bit 6=HP). \\
\end{longtable}

\begin{notebox}
The BASIC commands \cmd{INSTRUMENT}, \cmd{SOUND}, and \cmd{MUSIC} manage
SID registers automatically. Direct writes to \$D400+ are for advanced
use only and may conflict with the music engine.
\end{notebox}

% =============================================================================
\section{Timer Controller and Music Status}
% =============================================================================

\subsection*{Timer Controller (\$BA40--\$BA4F)}

The timer controller provides periodic interrupt generation for the SID
player. Configuration is handled automatically by \cmd{SIDPLAY}.

\subsection*{Music Status (\$BA50--\$BA53)}

\begin{longtable}{>{\ttfamily\raggedright\arraybackslash}p{0.14\textwidth} p{0.10\textwidth} p{0.62\textwidth}}
\toprule
\normalfont\textbf{Address} & \textbf{Access} & \textbf{Description} \\
\midrule
\endhead
\bottomrule
\endfoot
\$BA50 & RO & Status flags: bit 0 = SFX playing, bit 1 = music playing. \\
\$BA51 & RO & Voice 1 current MIDI note (0 = silent). \\
\$BA52 & RO & Voice 2 current MIDI note (0 = silent). \\
\$BA53 & RO & Voice 3 current MIDI note (0 = silent). \\
\end{longtable}

These registers are read by the \cmd{PLAYING} and \cmd{MNOTE()} functions.

% =============================================================================
\section{VSC Register Map}
% =============================================================================

The Virtual Sound Controller occupies \$A100--\$A1FF.
Writing to \$A100 (\texttt{VscCmd}) both stores the command byte and executes it.

\subsection*{VSC Registers}

\begin{longtable}{>{\ttfamily\raggedright\arraybackslash}p{0.14\textwidth} >{\ttfamily\raggedright\arraybackslash}p{0.22\textwidth} p{0.10\textwidth} p{0.40\textwidth}}
\toprule
\normalfont\textbf{Address} & \normalfont\textbf{Name} & \textbf{Access} & \textbf{Description} \\
\midrule
\endhead
\bottomrule
\endfoot
\$A100 & VscCmd        & R/W & Command byte; writing triggers execution. \\
\$A101 & VscP0         & R/W & Parameter 0. \\
\$A102 & VscP1         & R/W & Parameter 1. \\
\$A103 & VscP2         & R/W & Parameter 2. \\
\$A104 & VscP3         & R/W & Parameter 3. \\
\$A105 & VscP4         & R/W & Parameter 4. \\
\$A106 & VscP5         & R/W & Parameter 5. \\
\$A107 & VscP6         & R/W & Parameter 6. \\
\$A108 & VscP7         & R/W & Parameter 7. \\
\$A10E & VscActiveMask & RO  & Bitmask of currently playing channels (bits 0--3). \\
\$A10F & VscMasterVol  & RO  & Current master volume (0--15); set via \texttt{VscCmdVolume}. \\
\end{longtable}

\subsection*{VSC Command Codes}

\begin{longtable}{>{\ttfamily\raggedright\arraybackslash}p{0.10\textwidth} >{\ttfamily\raggedright\arraybackslash}p{0.22\textwidth} p{0.54\textwidth}}
\toprule
\normalfont\textbf{Code} & \normalfont\textbf{Name} & \textbf{Parameters} \\
\midrule
\endhead
\bottomrule
\endfoot
\$01 & VscCmdSound    & P0 = channel (masked to 0--3), P1/P2 = frequency (16-bit Hz), P3/P4 = duration (16-bit, units of 1/60 s). Frequency is clamped to 16--12000 Hz. \cmd{freq<=0} or \cmd{dur<=0} stops the channel. \\
\$02 & VscCmdVolume   & P0 low nibble = master volume (0--15). \\
\$03 & VscCmdEnvelope & P0 = channel (masked to 0--3), P1 = attack, P2 = decay, P3 = sustain (low nibble, 0--15 scale), P4 = release. Each time value is scaled to samples internally. \\
\$04 & VscCmdWave     & P0 = channel (masked to 0--3), P1 = waveform index (\texttt{P1 mod 5}): 0=square, 1=sawtooth, 2=triangle, 3=noise, 4=sine. \\
\end{longtable}

% =============================================================================
\section{FIO Register Map}
% =============================================================================

The File I/O Controller occupies \$B9A0--\$B9EF.
Writing to \$B9A0 (\texttt{FioCmd}) triggers the operation.
The caller polls \$B9A1 (\texttt{FioStatus}) for completion.

\subsection*{FIO Registers}

\begin{longtable}{>{\ttfamily\raggedright\arraybackslash}p{0.14\textwidth} >{\ttfamily\raggedright\arraybackslash}p{0.18\textwidth} p{0.10\textwidth} p{0.44\textwidth}}
\toprule
\normalfont\textbf{Address} & \normalfont\textbf{Name} & \textbf{Access} & \textbf{Description} \\
\midrule
\endhead
\bottomrule
\endfoot
\$B9A0 & FioCmd    & R/W & Command byte; writing triggers the operation. \\
\$B9A1 & FioStatus & RO  & Result status: 0=idle, 2=ok, 3=error. \\
\$B9A2 & FioErrCode & RO & Error detail code (see below). \\
\$B9A3 & FioNameLen & R/W & Filename length in bytes (1--63). \\
\$B9A4 & FioSrcL   & R/W & Source/destination address, low byte. \\
\$B9A5 & FioSrcH   & R/W & Source/destination address, high byte. \\
\$B9A6 & FioEndL   & R/W & End address, low byte (used by \cmd{SAVE} to determine program extent). \\
\$B9A7 & FioEndH   & R/W & End address, high byte. \\
\$B9A8 & FioSizeL  & RO  & Loaded data size, low byte (written by host after \cmd{LOAD} or \cmd{DIR} read). \\
\$B9A9 & FioSizeH  & RO  & Loaded data size, high byte. \\
\$B9B0--\$B9EF & FioName & R/W & Filename buffer (64 bytes ASCII, not null-terminated). \\
\end{longtable}

\subsection*{FIO Command Codes}

\begin{longtable}{>{\ttfamily\raggedright\arraybackslash}p{0.10\textwidth} >{\ttfamily\raggedright\arraybackslash}p{0.22\textwidth} p{0.54\textwidth}}
\toprule
\normalfont\textbf{Code} & \normalfont\textbf{Name} & \textbf{Behavior} \\
\midrule
\endhead
\bottomrule
\endfoot
\$01 & FioCmdSave    & Save bytes from \texttt{FioSrcL/H} to \texttt{FioEndL/H} (exclusive) to disk; prepends a 2-byte load-address. \\
\$02 & FioCmdLoad    & Load file into RAM at \texttt{FioSrcL/H}; skips the 2-byte load-address prefix; sets \texttt{FioSizeL/H}. \\
\$03 & FioCmdDirOpen & Open the program directory; populates \texttt{FioName} and \texttt{FioSizeL/H} with the first entry. \\
\$04 & FioCmdDirRead & Advance to the next directory entry; populates \texttt{FioName} and \texttt{FioSizeL/H}. \\
\$05 & FioCmdDelete  & Delete the named program from disk. \\
\$06 & FioCmdGSave   & Save VGC memory space to a \texttt{.gfx} file. FioGSpace=space, FioGAddrL/H=offset, FioGLenL/H=length. \\
\$07 & FioCmdGLoad   & Load \texttt{.gfx} file into VGC memory space. FioGSpace=space, FioGAddrL/H=offset, FioGLenL/H=max length. \\
\$08 & FioCmdSidPlay & Load and play a \texttt{.sid} file. FioSrcL=song number (1-based). \\
\$09 & FioCmdSidStop & Stop SID file playback. \\
\$0A & FioCmdInstrument & Define instrument preset. FioSrcL=id, FioSrcH=waveform, FioEndL=A, FioEndH=D, FioSizeL=S, FioSizeH=R. \\
\$0B & FioCmdSound   & Play SFX. FioSrcL=MIDI note, FioSrcH=duration (frames), FioEndL=instrument ID. \\
\$0C & FioCmdVolume  & Set SID master volume. FioSrcL=level (0--15). \\
\$0D & FioCmdMSeq    & Load MML sequence. FioSrcL=voice (1--3), FioEndL/H=string pointer, FioNameLen=string length. \\
\$0E & FioCmdMPlay   & Start music playback. \\
\$0F & FioCmdMStop   & Stop music playback. \\
\$10 & FioCmdMTempo  & Set tempo. FioSrcL/H=BPM (16-bit). \\
\$11 & FioCmdMLoop   & Set loop. FioSrcL=0 (off) or 1 (on). \\
\end{longtable}

\subsection*{FIO Status Codes}

\begin{longtable}{>{\ttfamily\raggedright\arraybackslash}p{0.12\textwidth} >{\ttfamily\raggedright\arraybackslash}p{0.20\textwidth} p{0.54\textwidth}}
\toprule
\normalfont\textbf{Value} & \normalfont\textbf{Name} & \textbf{Meaning} \\
\midrule
\endhead
\bottomrule
\endfoot
\$00 & FioStatusIdle  & No operation in progress. \\
\$02 & FioStatusOk    & Operation completed successfully. \\
\$03 & FioStatusError & Operation failed; see \texttt{FioErrCode}. \\
\end{longtable}

\subsection*{FIO Error Codes}

\begin{longtable}{>{\ttfamily\raggedright\arraybackslash}p{0.12\textwidth} >{\ttfamily\raggedright\arraybackslash}p{0.24\textwidth} p{0.50\textwidth}}
\toprule
\normalfont\textbf{Value} & \normalfont\textbf{Name} & \textbf{Meaning} \\
\midrule
\endhead
\bottomrule
\endfoot
\$00 & FioErrNone     & No error. \\
\$01 & FioErrNotFound & File not found on disk. \\
\$02 & FioErrIo       & Host I/O error (invalid name, end address $\leq$ start, OS exception). \\
\$03 & FioErrEndOfDir & No more directory entries (returned for \texttt{DirOpen} on empty dir or after last entry). \\
\end{longtable}

% =============================================================================
\section{XMC Register Map}
% =============================================================================

The Expansion Memory Controller occupies \$BA00--\$BA3F.
Writing to \$BA00 (\texttt{XmcCmd}) triggers the operation.
Memory windows (\$BC00--\$BFFF) provide direct CPU-bus access to mapped XRAM pages.

\subsection*{XMC Registers}

\begin{longtable}{>{\ttfamily\raggedright\arraybackslash}p{0.14\textwidth} >{\ttfamily\raggedright\arraybackslash}p{0.22\textwidth} p{0.10\textwidth} p{0.40\textwidth}}
\toprule
\normalfont\textbf{Address} & \normalfont\textbf{Name} & \textbf{Access} & \textbf{Description} \\
\midrule
\endhead
\bottomrule
\endfoot
\$BA00 & XmcCmd        & R/W & Command byte; writing triggers execution. \\
\$BA01 & XmcStatus     & RO  & Result status: 0=idle, 2=ok, 3=error. \\
\$BA02 & XmcErrCode    & RO  & Error detail code (see below). \\
\$BA03 & XmcCfg        & R/W & Reserved. \\
\$BA04 & XmcAddrL      & R/W & XRAM address, low byte. \\
\$BA05 & XmcAddrM      & R/W & XRAM address, middle byte. \\
\$BA06 & XmcAddrH      & R/W & XRAM address, high byte. \\
\$BA07 & XmcRamL       & R/W & CPU RAM address, low byte. \\
\$BA08 & XmcRamH       & R/W & CPU RAM address, high byte. \\
\$BA09 & XmcLenL       & R/W & Transfer length, low byte. \\
\$BA0A & XmcLenH       & R/W & Transfer length, high byte. \\
\$BA0B & XmcData       & R/W & Byte data port (used by \texttt{GetByte}/\texttt{PutByte}). \\
\$BA0C & XmcBank       & R/W & Default 64 KB bank selector. \\
\$BA0D & XmcBanks      & RO  & Total number of 64 KB banks available (read-only). \\
\$BA0E & XmcPagesUsedL & RO  & Used 256-byte pages, low byte. \\
\$BA0F & XmcPagesUsedH & RO  & Used 256-byte pages, high byte. \\
\$BA10 & XmcPagesFreeL & RO  & Free 256-byte pages, low byte. \\
\$BA11 & XmcPagesFreeH & RO  & Free 256-byte pages, high byte. \\
\$BA12 & XmcNameLen    & R/W & Name length for named block operations (1--28). \\
\$BA13 & XmcHandle     & RO  & Block handle returned by \texttt{Alloc}/\texttt{NStash}/\texttt{DirRead}. \\
\$BA14 & XmcDirCountL  & RO  & Count of named blocks, low byte. \\
\$BA15 & XmcDirCountH  & RO  & Count of named blocks, high byte. \\
\$BA16 & XmcWinCtl     & R/W & Window enable bitmask (bit 0=window 0, bit 1=window 1, etc.). \\
\$BA18 & XmcWin0AL     & R/W & Window 0 mapped XRAM base address, low byte. \\
\$BA19 & XmcWin0AM     & R/W & Window 0 mapped XRAM base address, middle byte. \\
\$BA1A & XmcWin0AH     & R/W & Window 0 mapped XRAM base address, high byte. \\
\$BA1B & XmcWin1AL     & R/W & Window 1 mapped XRAM base address, low byte. \\
\$BA1C & XmcWin1AM     & R/W & Window 1 mapped XRAM base address, middle byte. \\
\$BA1D & XmcWin1AH     & R/W & Window 1 mapped XRAM base address, high byte. \\
\$BA1E & XmcWin2AL     & R/W & Window 2 mapped XRAM base address, low byte. \\
\$BA1F & XmcWin2AM     & R/W & Window 2 mapped XRAM base address, middle byte. \\
\$BA20 & XmcWin2AH     & R/W & Window 2 mapped XRAM base address, high byte. \\
\$BA21 & XmcWin3AL     & R/W & Window 3 mapped XRAM base address, low byte. \\
\$BA22 & XmcWin3AM     & R/W & Window 3 mapped XRAM base address, middle byte. \\
\$BA23 & XmcWin3AH     & R/W & Window 3 mapped XRAM base address, high byte. \\
\$BA24--\$BA3F & XmcName & R/W & ASCII name buffer (28 bytes, not null-terminated). \\
\end{longtable}

\subsection*{XMC Command Codes}

\begin{longtable}{>{\ttfamily\raggedright\arraybackslash}p{0.10\textwidth} >{\ttfamily\raggedright\arraybackslash}p{0.24\textwidth} p{0.52\textwidth}}
\toprule
\normalfont\textbf{Code} & \normalfont\textbf{Name} & \textbf{Behavior} \\
\midrule
\endhead
\bottomrule
\endfoot
\$01 & XmcCmdGetByte    & Read byte at \texttt{XmcAddrL/M/H} into \texttt{XmcData}. \\
\$02 & XmcCmdPutByte    & Write \texttt{XmcData} to \texttt{XmcAddrL/M/H}; marks page used. \\
\$03 & XmcCmdStash      & Copy \texttt{XmcLenL/H} bytes from CPU RAM at \texttt{XmcRamL/H} to XRAM at \texttt{XmcAddrL/M/H}. \texttt{len=0} is a no-op success. \\
\$04 & XmcCmdFetch      & Copy \texttt{XmcLenL/H} bytes from XRAM at \texttt{XmcAddrL/M/H} to CPU RAM at \texttt{XmcRamL/H}. \texttt{len=0} is a no-op success. \\
\$05 & XmcCmdFill       & Fill \texttt{XmcLenL/H} bytes in XRAM starting at \texttt{XmcAddrL/M/H} with \texttt{XmcData}. \\
\$07 & XmcCmdStats      & Refresh the \texttt{PagesUsed}/\texttt{PagesFree}/\texttt{DirCount} read-only registers. \\
\$08 & XmcCmdResetUsage & Clear all usage tracking, block records, and named-block metadata (destructive). \\
\$09 & XmcCmdRelease    & Mark XRAM range (\texttt{XmcAddrL/M/H}, \texttt{XmcLenL/H}) as free; removes any overlapping block records. \\
\$0A & XmcCmdAlloc      & Allocate \texttt{XmcLenL/H} bytes; sets \texttt{XmcAddrL/M/H}, \texttt{XmcHandle}, and \texttt{XmcBank}. \\
\$0B & XmcCmdNStash     & Named stash: create or update named block from CPU RAM; name read from \texttt{XmcName}/\texttt{XmcNameLen}. \\
\$0C & XmcCmdNFetch     & Named fetch: copy named block to CPU RAM at \texttt{XmcRamL/H}; \texttt{len=0} fetches full block. \\
\$0D & XmcCmdNDelete    & Delete named block by name. \\
\$0E & XmcCmdNDirOpen   & Open named-block directory; emits first entry to registers. \\
\$0F & XmcCmdNDirRead   & Advance to the next named-block directory entry. \\
\end{longtable}

\subsection*{XMC Status Codes}

\begin{longtable}{>{\ttfamily\raggedright\arraybackslash}p{0.12\textwidth} >{\ttfamily\raggedright\arraybackslash}p{0.26\textwidth} p{0.48\textwidth}}
\toprule
\normalfont\textbf{Value} & \normalfont\textbf{Name} & \textbf{Meaning} \\
\midrule
\endhead
\bottomrule
\endfoot
\$00 & XmcStatusIdle  & No operation in progress. \\
\$02 & XmcStatusOk    & Operation completed successfully. \\
\$03 & XmcStatusError & Operation failed; see \texttt{XmcErrCode}. \\
\end{longtable}

\subsection*{XMC Error Codes}

\begin{longtable}{>{\ttfamily\raggedright\arraybackslash}p{0.12\textwidth} >{\ttfamily\raggedright\arraybackslash}p{0.26\textwidth} p{0.48\textwidth}}
\toprule
\normalfont\textbf{Value} & \normalfont\textbf{Name} & \textbf{Meaning} \\
\midrule
\endhead
\bottomrule
\endfoot
\$00 & XmcErrNone     & No error. \\
\$01 & XmcErrRange    & XRAM address or length out of bounds. \\
\$02 & XmcErrBadArgs  & Invalid arguments (e.g., \texttt{len<=0} for \texttt{Alloc}, unknown command). \\
\$03 & XmcErrNotFound & Named block not found. \\
\$04 & XmcErrNoSpace  & No contiguous free pages of the required size, or handle pool exhausted. \\
\$05 & XmcErrName     & Name length is 0 or exceeds 28, or name is blank after trimming. \\
\$06 & XmcErrEndOfDir & No more named-block directory entries. \\
\end{longtable}

% =============================================================================
\section{System Vectors}
% =============================================================================

The address range \$0200--\$027F is the system vector table.
Each entry is a 16-bit little-endian pointer initialized from ROM at cold start.
BASIC uses the lower portion; the upper portion is reserved for future use.

\begin{longtable}{>{\ttfamily\raggedright\arraybackslash}p{0.14\textwidth} p{0.60\textwidth}}
\toprule
\normalfont\textbf{Address} & \textbf{Purpose} \\
\midrule
\endhead
\bottomrule
\endfoot
\$0200--\$0201 & IRQ handler vector (2 bytes, little-endian). Set by the \cmd{IRQ} statement. \\
\$0202--\$0203 & NMI handler vector (2 bytes, little-endian). Set by the \cmd{NMI} statement. \\
\$0204--\$020D & Reserved BASIC internal vectors (warm-start, error, output, input hooks). \\
\$020E--\$027F & Reserved for future system use. \\
\end{longtable}

\begin{notebox}
The exact layout of \$0204--\$020D is inherited from EhBASIC 2.22p5 and tracks
the standard warm-start, error, and I/O indirection vectors.
Refer to \texttt{ehbasic/basic.asm} for the definitive symbol assignments.
\end{notebox}

% =============================================================================
% Appendix: Limits, Errors, and Edge Cases
% NovaBASIC v1.0 User Guide
% Source of truth: ehbasic/basic.asm
%                  e6502.Avalonia/Hardware/FileIoController.cs
%                  e6502.Avalonia/Hardware/VirtualGraphicsController.cs
%                  e6502.Avalonia/Hardware/VirtualSoundController.cs
%                  e6502.Avalonia/Hardware/VirtualExpansionMemoryController.cs
% =============================================================================
\chapter{Limits, Errors, and Edge Cases}

This appendix documents all known numeric limits, argument validation rules,
and edge-case behaviors derived from a direct audit of the ROM source and the
Avalonia hardware controller implementations.
Where ROM behavior and host behavior differ, both are noted.

% =============================================================================
\section{Numeric Argument Conversion Rules}
% =============================================================================

Most command arguments are passed through one of two shared ROM helpers
before reaching the hardware layer.
Understanding their constraints prevents unexpected function-call errors.

\begin{longtable}{>{\ttfamily\raggedright\arraybackslash}p{0.26\textwidth} p{0.64\textwidth}}
\toprule
\normalfont\textbf{Helper} & \textbf{Behavior} \\
\midrule
\endhead
\bottomrule
\endfoot
LAB\_GTBY & Converts the FAC (floating-point accumulator) to an unsigned byte.
             Accepts values 0--255.
             Any value outside this range, or any negative value, raises a
             function-call error before the command reaches the hardware. \\
LAB\_GTWRD & Converts the FAC to an unsigned 16-bit integer.
              Accepts values 0--65535.
              Negative values or values above 65535 raise a function-call error. \\
\end{longtable}

\begin{warningbox}
Commands that accept addresses (\cmd{POKE}, \cmd{DOKE}, \cmd{CALL}, \cmd{WAIT}, \cmd{STASH}, \cmd{FETCH}\ldots)
route through \cmd{LAB\_GTWRD}.
Passing a value such as \texttt{-1} will raise an error rather than
wrapping to \texttt{\$FFFF}.
\end{warningbox}

% =============================================================================
\section{File Command Limits}
% =============================================================================

\begin{longtable}{p{0.30\textwidth} p{0.60\textwidth}}
\toprule
\textbf{Topic} & \textbf{Behavior} \\
\midrule
\endhead
\bottomrule
\endfoot
Filename length &
  The ROM filename parser accepts 1--63 characters.
  A length of 0 or greater than 63 causes the FIO controller to return an
  I/O error (the \cmd{ReadFilename} guard in \texttt{FileIoController.cs}). \\
Allowed filename characters &
  The host implementation enforces the pattern \texttt{[A-Za-z0-9\_.\textbackslash-]+}.
  Any character outside this set causes \cmd{ReadFilename} to return \texttt{null}
  and the operation to fail with \cmd{FioErrIo}. \\
\texttt{.bas} extension &
  If the filename does not already end in \texttt{.bas} (case-insensitive),
  the host appends it automatically before forming the filesystem path. \\
Missing file &
  \cmd{LOAD "name"} and \cmd{DEL "name"} on a non-existent file set
  \texttt{FioStatus=\$03} and \texttt{FioErrCode=\$01} (\cmd{FioErrNotFound}).
  The ROM interprets this as ``File not found''. \\
I/O fault &
  Any OS-level exception during \cmd{SAVE}/\cmd{LOAD}/\cmd{DEL} sets
  \texttt{FioStatus=\$03} and \texttt{FioErrCode=\$02} (\cmd{FioErrIo}).
  The ROM surfaces this as ``I/O Error''. \\
\cmd{SAVE} end address &
  If \texttt{FioEndH:FioEndL} $\leq$ \texttt{FioSrcH:FioSrcL}, the save
  is rejected immediately with \cmd{FioErrIo} before the file is opened. \\
\cmd{DIR} on empty catalog &
  \cmd{DirOpen} sets \texttt{FioStatus=\$03}/\texttt{FioErrCode=\$03}
  (\cmd{FioErrEndOfDir}) when no \texttt{.bas} files exist.
  The ROM \cmd{DIR} handler treats this as a silent empty listing. \\
\cmd{DIR} after last entry &
  Each \cmd{DirRead} beyond the final file sets \cmd{FioErrEndOfDir};
  the ROM stops iterating. \\
\end{longtable}

% =============================================================================
\section{Graphics and Sprite Edge Cases}
% =============================================================================

\begin{longtable}{p{0.32\textwidth} p{0.58\textwidth}}
\toprule
\textbf{Command / Feature} & \textbf{Limit and edge behavior} \\
\midrule
\endhead
\bottomrule
\endfoot
\cmd{GCOLOR c} &
  Only the low nibble of \textit{c} is used (\texttt{c \& 0x0F}).
  Values 0--15 are valid; values above 15 wrap silently into the 0--15 range. \\
Color 0 on the graphics layer &
  Color index 0 means transparent on the graphics bitmap.
  \cmd{UNPLOT x,y} explicitly sets a pixel to 0 to restore transparency. \\
\cmd{PLOT}/\cmd{UNPLOT}/\cmd{PAINT} bounds &
  The ROM dispatches coordinates without host-side pre-clipping.
  The host \texttt{BlockGraphics} implementation performs pixel-level bounds
  checks; out-of-range coordinates are silently ignored. \\
\cmd{LINE}/\cmd{RECT}/\cmd{CIRCLE}/\cmd{FILL} bounds &
  Drawing operations are clipped by the host renderer (\texttt{BlockGraphics.cs}).
  Portions of the shape outside the 320$\times$200 pixel area are dropped;
  no error is raised. \\
\cmd{FILL} rectangle &
  Coordinates are clamped to the screen boundary before drawing;
  the swap of x0/x1 or y0/y1 to ensure a positive rectangle is handled by
  the host. \\
\cmd{SPRITE n,\ldots} invalid index &
  Sprite index \textit{n} must be 0--15.
  The VGC command handlers check \texttt{n >= MaxSprites} and return immediately
  without error; no ROM-level error is raised. \\
\cmd{SPRITEDATA n,row,\ldots} &
  Row must be 0--15.
  The \texttt{CmdSprRow} handler checks both sprite index and row; an invalid
  row causes the command to be silently ignored. \\
\cmd{SPRITESHAPE}/\cmd{SPRITECOLOR} &
  These are tokenised and recognised by the ROM parser.
  However, the ROM handlers for both commands currently perform no hardware
  action.
  They exist for source compatibility; no VGC command is issued. \\
\cmd{SPRITEX(n)}/\cmd{SPRITEY(n)} &
  The ROM functions return 0 in v1.0.
  Sprite position read-back from the VGC is not yet implemented. \\
\cmd{COLLISION(n)}/\cmd{BUMPED(n)} &
  The VGC updates \texttt{RegColSt} and \texttt{RegColBg} each frame;
  the ROM reads the appropriate register and clears it on read.
  A given bit is set if sprite \textit{n} participated in a collision that frame. \\
Default sprite priority &
  On reset all sprites default to priority 2 (in front of everything).
  This matches the \texttt{SpritePriInFront} constant. \\
\cmd{CmdSprFlip} flags &
  Only bits 0--1 of the flags byte are used (\texttt{flags \& 0x03}):
  bit 0 = horizontal flip, bit 1 = vertical flip. \\
\cmd{CmdSprPri} clamping &
  Priority values above 2 are clamped to 2 (\texttt{Math.Min(value, 2)}).
  Values 0, 1, and 2 are the only meaningful levels. \\
\end{longtable}

% =============================================================================
\section{Sound Limits and Behavior}
% =============================================================================

\begin{longtable}{p{0.32\textwidth} p{0.58\textwidth}}
\toprule
\textbf{Command / Feature} & \textbf{Limit and edge behavior} \\
\midrule
\endhead
\bottomrule
\endfoot
Channel masking &
  The channel index is masked to the range 0--3 by the host:
  \texttt{ch = P0 \& (ChannelCount - 1)}.
  Values above 3 wrap into range; no error is raised. \\
Frequency clamping &
  Frequency is clamped to 16--12000 Hz by the host
  (\texttt{Math.Clamp(freq, 16, 12000)}).
  The ROM passes the raw 16-bit value; out-of-range values are silently
  adjusted rather than rejected. \\
Stopping a channel &
  If \textit{freq} $\leq$ 0 or \textit{dur} $\leq$ 0, the channel is immediately
  deactivated (\texttt{Active = false}, \texttt{NotePos} and \texttt{NoteTotalSamples}
  reset to 0).
  This is the correct way to stop a playing note before its duration expires. \\
Duration units &
  Duration is in units of 1/60 second (video frames).
  Internally: \texttt{samples = durationTicks * 44100 / 60}. \\
Master volume &
  Only the low nibble of the volume byte is used (\texttt{P0 \& 0x0F}).
  The default master volume on power-on is 12. \\
Waveform wrapping &
  The waveform index is wrapped: \texttt{waveform = P1 \% 5}.
  Values 0--4 select square, sawtooth, triangle, noise, and sine respectively. \\
Sustain level &
  In \cmd{ENVELOPE}, the sustain parameter (P3) uses only the low nibble
  (0--15 scale), converted to a floating-point level by the host:
  \texttt{SustainLevel = (P3 \& 0x0F) / 15.0}. \\
Envelope time parameters &
  Attack (P1), decay (P2), and release (P4) are converted to sample counts:
  \texttt{samples = max(1, value * 44100 / 240)}.
  A value of 0 yields 1 sample (instantaneous). \\
\end{longtable}

% =============================================================================
\section{XRAM Limits and Failure Modes}
% =============================================================================

\begin{longtable}{p{0.32\textwidth} p{0.58\textwidth}}
\toprule
\textbf{Command / Feature} & \textbf{Limit and edge behavior} \\
\midrule
\endhead
\bottomrule
\endfoot
\cmd{XBANK n} &
  The ROM verifies that \textit{n} is less than the value stored at
  \texttt{XmcBanks} (\$BA0D).
  An out-of-range bank number triggers a function-call error in the ROM
  before any XMC command is issued. \\
Window number (\cmd{XMAP}/\cmd{XUNMAP}) &
  Window must be 0--3.
  The ROM validates this; an invalid window triggers a function-call error. \\
Window address space &
  The four CPU-visible windows occupy \$BC00--\$BFFF (4 $\times$ 256 bytes).
  Window 0 maps to \$BC00, window 1 to \$BD00, window 2 to \$BE00,
  window 3 to \$BF00. \\
Unmapped window reads/writes &
  If a window is not enabled in \texttt{XmcWinCtl}, the XMC does not own that
  address and the read/write falls through to flat RAM.
  No error is returned. \\
Named block name length &
  The ROM enforces a 1--28 byte name (\texttt{XmcNameLen} is capped at 28 by
  the name buffer size: \$BA24--\$BA3F = 28 bytes).
  The host trims whitespace; a blank name after trimming is rejected with
  \cmd{XmcErrName}. \\
Named block name case &
  Name lookup is case-insensitive in the host
  (\texttt{StringComparer.OrdinalIgnoreCase}).
  Storing ``\texttt{SPRITE}'' and retrieving ``\texttt{sprite}'' will succeed. \\
\cmd{XALLOC len} with \texttt{len<=0} &
  The ROM passes zero through \cmd{LAB\_GTWRD}, which itself rejects negative
  values.
  The XMC command handler rejects \texttt{len<=0} with \cmd{XmcErrBadArgs}. \\
\cmd{XALLOC} with no free space &
  If no contiguous run of the required pages exists, or the handle pool (1--255)
  is exhausted, the command fails with \cmd{XmcErrNoSpace}. \\
\cmd{STASH}/\cmd{FETCH} (raw) with \texttt{len=0} &
  The XMC host treats a zero-length raw transfer as a no-op and returns
  \cmd{XmcStatusOk}.
  No data is moved and no pages are marked. \\
\cmd{FETCH "name",ram} (named fetch) &
  The ROM sends \texttt{XmcLenL/H = 0} for the named-fetch command.
  The host interprets \texttt{requested=0} as ``fetch the entire stored block'':
  \texttt{len = (requested <= 0) ? block.Length : min(requested, block.Length)}. \\
\cmd{STASH "name",ram,len} over existing block &
  If the named block already exists and the new length fits in the allocated
  pages, only \texttt{block.Length} is updated (no reallocation).
  If it does not fit, the old block is freed and a new allocation is attempted. \\
\cmd{XRESET} &
  Clears the \texttt{\_usedPages} array, resets \texttt{\_usedPageCount} to 0,
  and removes all block and name records.
  The raw XRAM byte array is \emph{not} zeroed; data remains but is inaccessible
  through the allocation system. \\
\cmd{XFREE off,len} &
  Frees all usage-tracking pages in the given range and removes any tracked
  blocks (named or unnamed) whose page range overlaps with the freed region. \\
\cmd{XPOKE}/\cmd{XPEEK} bank offset &
  The ROM constructs the XRAM address as
  \texttt{bank * 65536 + offset}.
  The host validates the resulting 24-bit address against the total XRAM size;
  out-of-range addresses return \cmd{XmcErrRange}. \\
RAM range validation for STASH/FETCH &
  The host prevents writes to ROM space: any \cmd{FETCH} operation whose
  destination range would extend into \$C000 or above is rejected with
  \cmd{XmcErrRange}.
  Reads (\cmd{STASH}) may source from ROM addresses, allowing code capture. \\
\end{longtable}

% =============================================================================
\section{Status and Error Code Quick Reference}
% =============================================================================

\subsection*{File I/O Controller (FIO) Codes}

\begin{longtable}{>{\ttfamily\raggedright\arraybackslash}p{0.12\textwidth} >{\ttfamily\raggedright\arraybackslash}p{0.26\textwidth} p{0.50\textwidth}}
\toprule
\normalfont\textbf{Value} & \normalfont\textbf{Symbol} & \textbf{Meaning} \\
\midrule
\endhead
\bottomrule
\endfoot
\multicolumn{3}{l}{\normalfont\textit{Status codes (\$B9A1)}} \\
\midrule
\$00 & FioStatusIdle  & No operation has been issued since last reset. \\
\$02 & FioStatusOk    & Last operation succeeded. \\
\$03 & FioStatusError & Last operation failed; check error code. \\
\midrule
\multicolumn{3}{l}{\normalfont\textit{Error codes (\$B9A2)}} \\
\midrule
\$00 & FioErrNone     & No error. \\
\$01 & FioErrNotFound & File not found on disk. \\
\$02 & FioErrIo       & Host I/O error (bad name, OS exception, end address $\leq$ start). \\
\$03 & FioErrEndOfDir & Directory enumeration exhausted. \\
\end{longtable}

\subsection*{Expansion Memory Controller (XMC) Codes}

\begin{longtable}{>{\ttfamily\raggedright\arraybackslash}p{0.12\textwidth} >{\ttfamily\raggedright\arraybackslash}p{0.26\textwidth} p{0.50\textwidth}}
\toprule
\normalfont\textbf{Value} & \normalfont\textbf{Symbol} & \textbf{Meaning} \\
\midrule
\endhead
\bottomrule
\endfoot
\multicolumn{3}{l}{\normalfont\textit{Status codes (\$BA01)}} \\
\midrule
\$00 & XmcStatusIdle  & No operation in progress. \\
\$02 & XmcStatusOk    & Last operation succeeded. \\
\$03 & XmcStatusError & Last operation failed; check error code. \\
\midrule
\multicolumn{3}{l}{\normalfont\textit{Error codes (\$BA02)}} \\
\midrule
\$00 & XmcErrNone     & No error. \\
\$01 & XmcErrRange    & XRAM address or transfer endpoint out of XRAM bounds, or FETCH would write into ROM. \\
\$02 & XmcErrBadArgs  & Invalid argument (\texttt{len<=0} for Alloc, unknown command byte). \\
\$03 & XmcErrNotFound & Named block not found in directory. \\
\$04 & XmcErrNoSpace  & No contiguous free pages available, or handle pool (1--255) exhausted. \\
\$05 & XmcErrName     & Name length 0 or $>$28, or name is blank after trimming. \\
\$06 & XmcErrEndOfDir & No more named-block directory entries. \\
\end{longtable}

\begin{tipbox}
The BASIC runtime maps both FIO and XMC error returns to one of three
user-visible messages: ``\texttt{File not found}'', ``\texttt{I/O Error}'',
or a function-call error.
For low-level programs that \cmd{POKE} the controller registers directly,
use the tables above to interpret the raw status and error bytes.
\end{tipbox}


\end{document}
