% =============================================================================
% Chapter 8 — DMA and Blitter
% NovaBASIC v1.0 User Guide
% Source of truth: e6502.Avalonia/Hardware/DmaController.cs
%                  e6502.Avalonia/Hardware/BlitterController.cs
%                  e6502.Avalonia/Hardware/CompositeBusDevice.cs
% =============================================================================
\chapter{DMA and Blitter}
\label{chap:dma-blitter}

\epigraph{\itshape ``Speed is not a luxury --- it is the difference between
  a program that feels alive and one that feels like it is apologising.''}%
         {--- e6502 Virtual Computer Design Notes}

\noindent
Every POKE loop that copies a screen buffer byte-by-byte is burning CPU cycles
that your program could be spending on game logic, animation, or music.
NovaBASIC includes two hardware accelerators that move data without occupying
the CPU: the \textbf{DMA controller} for flat bulk transfers and the
\textbf{blitter} for two-dimensional rectangular copies and fills.
Both work across six unified memory spaces --- CPU RAM, character RAM, color
RAM, the graphics bitmap, sprite shape memory, and expansion RAM --- so data
can be moved between any combination of those regions in a single command.
This chapter covers both controllers, their BASIC keywords, and the memory
space model they share.

% -----------------------------------------------------------------------------
\section{The DMA Controller}
\label{sec:dma-controller}
% -----------------------------------------------------------------------------

\textbf{Direct Memory Access (DMA)} is a hardware technique that transfers
a block of bytes between two memory addresses without executing a load/store
loop on the CPU. The CPU hands a source address, destination address, and
length to the controller and the transfer happens autonomously. From the
perspective of a BASIC program the operation appears instantaneous.

The DMA controller understands six \textbf{unified memory spaces}. Each space
is identified by a small integer that you pass as a parameter:

\begin{center}
\begin{tabular}{>{\ttfamily}c l r l}
\toprule
\textbf{ID} & \textbf{Space} & \textbf{Size} & \textbf{Notes} \\
\midrule
0 & CPU RAM          & 64 KB        & Full 6502 address space \\
1 & Character RAM    & 2,000 bytes  & 80$\times$25 text cells \\
2 & Color RAM        & 2,000 bytes  & 80$\times$25 color attributes \\
3 & Graphics Bitmap  & 64,000 bytes & 320$\times$200 pixels, 4-bit color \\
4 & Sprite Shapes    & 2,048 bytes  & 16 sprites $\times$ 128 bytes each \\
5 & Expansion RAM    & up to 512 KB & Uses current XBANK \\
\bottomrule
\end{tabular}
\end{center}

\medskip
For spaces 1 through 4, the address parameter is a zero-based byte offset
into that hardware memory region --- so address 0 in space~1 is the first
character cell, address 80 is the start of the second row, and so on.

\subsection*{\cmd{DMACOPY srcSpace, srcAddr, dstSpace, dstAddr, length}}

Copies \emph{length} bytes from offset \emph{srcAddr} in \emph{srcSpace} to
offset \emph{dstAddr} in \emph{dstSpace}.

\begin{lstlisting}[style=basiclisting]
10 REM Copy 2000 bytes from CPU RAM at $6000 to character RAM
20 DMACOPY 0, 24576, 1, 0, 2000
\end{lstlisting}

Here space~0 is CPU RAM, \texttt{24576} is \$6000 decimal, space~1 is
character RAM, and the destination offset is 0 (top-left cell). After line~20
executes, the 2000-byte buffer you prepared in CPU RAM has been written
directly into the text screen.

\begin{lstlisting}[style=basiclisting]
10 REM Load graphics data from XRAM bank 0 into the bitmap
20 XBANK 0
30 DMACOPY 5, 0, 3, 0, 64000
\end{lstlisting}

Space~5 is expansion RAM. The current XBANK (set on line~20) determines which
512 KB bank is addressed. Line~30 copies all 64,000 bytes of the graphics
bitmap from the start of XRAM bank~0.

\subsection*{\cmd{DMAFILL dstSpace, dstAddr, length, value}}

Fills \emph{length} bytes starting at offset \emph{dstAddr} in
\emph{dstSpace} with the constant byte \emph{value}.

\begin{lstlisting}[style=basiclisting]
10 REM Clear character screen with space characters (ASCII 32)
20 DMAFILL 1, 0, 2000, 32
\end{lstlisting}

\begin{lstlisting}[style=basiclisting]
10 REM Fill color RAM with color 7 (white-on-black attribute)
20 DMAFILL 2, 0, 2000, 7
\end{lstlisting}

\subsection*{Status functions}

\begin{center}
\begin{tabular}{ll}
\toprule
\textbf{Function} & \textbf{Returns} \\
\midrule
\cmd{DMASTATUS} & 0 = idle, 1 = busy, 2 = done, 3 = error \\
\cmd{DMAERR}    & 0 = no error; non-zero = hardware error code \\
\cmd{DMACOUNT}  & Number of bytes transferred by the last operation \\
\bottomrule
\end{tabular}
\end{center}

\begin{notebox}
\cmd{DMACOPY} and \cmd{DMAFILL} poll the controller internally and raise a
BASIC error if the transfer ends in an error state. The status functions are
provided for advanced use --- for example, if you launch a transfer via direct
register writes (\cmd{POKE}) and want to monitor completion yourself.
\end{notebox}

% -----------------------------------------------------------------------------
\section{The Blitter}
\label{sec:blitter}
% -----------------------------------------------------------------------------

The DMA controller treats memory as a flat one-dimensional sequence of bytes.
The \textbf{blitter} adds a second dimension: it understands that memory is
arranged as rows of a fixed width and can copy or fill rectangular regions
without touching the bytes between rows.

The key concept is \textbf{stride}: the number of bytes from the beginning of
one row to the beginning of the next. For the 80-column text screen, stride is
80. For the 320-pixel-wide graphics bitmap, stride is 160 (320 pixels at 4
bits each, packed into 2 nibbles per byte). If source and destination have
different strides --- for example when copying a narrow tile into a wider
canvas --- you specify each independently.

\subsection*{\cmd{BLITCOPY srcSpace, srcAddr, srcStride, dstSpace, dstAddr, dstStride, width, height}}

Copies a rectangle \emph{width} bytes wide by \emph{height} rows tall from
the source region to the destination region.

\begin{lstlisting}[style=basiclisting]
10 REM Scroll the character screen up by one row.
20 REM Copy rows 1-24 (offset 80) to rows 0-23 (offset 0),
30 REM then blank the last row.
40 BLITCOPY 1, 80, 80, 1, 0, 80, 80, 24
50 DMAFILL 1, 1920, 80, 32
\end{lstlisting}

Line~40 copies 24 rows of 80 characters, shifting the entire screen up.
Line~50 fills the vacated bottom row (offset 1920 = 24$\times$80) with spaces.

\begin{lstlisting}[style=basiclisting]
10 REM Copy a 10x8 tile from CPU RAM at $3000 into color RAM
20 REM at column 5, row 3 (offset = 3*80+5 = 245).
30 BLITCOPY 0, 12288, 10, 2, 245, 80, 10, 8
\end{lstlisting}

The source stride is 10 because the tile is stored compactly (row-by-row,
10 bytes per row) in CPU RAM. The destination stride is 80 because color RAM
is 80 columns wide.

\subsection*{\cmd{BLITFILL dstSpace, dstAddr, dstStride, width, height, value}}

Fills a rectangle \emph{width} bytes wide by \emph{height} rows tall in
\emph{dstSpace} with the constant byte \emph{value}, respecting row stride.

\begin{lstlisting}[style=basiclisting]
10 REM Draw a 20x5 colored rectangle starting at column 10, row 2.
20 REM offset = 2*80+10 = 170. Stride = 80.
30 BLITFILL 2, 170, 80, 20, 5, 9
\end{lstlisting}

Without the blitter this would require a nested FOR/NEXT loop: one outer loop
over five rows and an inner loop over twenty columns, each issuing a POKE.
\cmd{BLITFILL} replaces all of that with a single hardware call.

\subsection*{Color-key transparency}

At the register level, bit~1 of the blitter mode register (\texttt{\$BA93})
enables \textbf{color-key} mode: source bytes that equal the color key value
(stored in \texttt{\$BA95}) are skipped, leaving the destination byte
unchanged. This is sprite-style transparency for arbitrary memory regions.
The \cmd{BLITCOPY} keyword does not expose this flag directly; use
\cmd{POKE \$BA93, 2} before issuing a raw blitter start
(\cmd{POKE \$BA80, 1}) to enable it.

\subsection*{Status functions}

\begin{center}
\begin{tabular}{ll}
\toprule
\textbf{Function} & \textbf{Returns} \\
\midrule
\cmd{BLITSTATUS} & 0 = idle, 1 = busy, 2 = done, 3 = error \\
\cmd{BLITERR}    & 0 = no error; non-zero = hardware error code \\
\cmd{BLITCOUNT}  & Number of bytes written by the last operation \\
\bottomrule
\end{tabular}
\end{center}

\begin{tipbox}
Use \cmd{DMACOPY} and \cmd{DMAFILL} for flat, one-dimensional transfers ---
copying a saved screen buffer, filling a region of expansion RAM, or moving a
block of arbitrary bytes. Use \cmd{BLITCOPY} and \cmd{BLITFILL} when your data
is inherently two-dimensional and you need to respect row boundaries, such as
scrolling text, drawing rectangles, or working with tile graphics. Choosing the
right tool avoids writing stride arithmetic in BASIC loops.
\end{tipbox}

% -----------------------------------------------------------------------------
\section{Memory Spaces Reference}
\label{sec:dma-memory-spaces}
% -----------------------------------------------------------------------------

The table below summarises all six space IDs accepted by both the DMA
controller and the blitter. Address offsets within each space are always
zero-based.

\begin{longtable}{>{\ttfamily}c l r p{5.8cm}}
\toprule
\textbf{ID} & \textbf{Space} & \textbf{Size} & \textbf{Notes} \\
\midrule
\endhead
\bottomrule
\endfoot
0 & CPU RAM         & 64 KB        &
  Full 6502 address space. Offsets are absolute CPU addresses. Writes to
  ROM (\$C000--\$FFFF) are silently ignored. \\[4pt]
1 & Character RAM   & 2,000 bytes  &
  80$\times$25 text cells, row-major. Offset 0 is top-left; offset 79 is
  top-right; offset 80 is the start of row~1. \\[4pt]
2 & Color RAM       & 2,000 bytes  &
  One color attribute byte per text cell, same layout as character RAM.
  Low nibble = foreground, high nibble = background. \\[4pt]
3 & Graphics Bitmap & 64,000 bytes &
  320$\times$200 pixels, two 4-bit pixels packed per byte (low nibble =
  left pixel). Row stride is 160 bytes. \\[4pt]
4 & Sprite Shapes   & 2,048 bytes  &
  16 sprite slots, 128 bytes each. Each slot holds a 16$\times$16 pixel
  image, two pixels per byte. \\[4pt]
5 & Expansion RAM   & up to 512 KB &
  The current XBANK setting is automatically applied as the high address
  byte. Set \cmd{XBANK} before any DMA or blitter operation that uses
  space~5. \\
\end{longtable}

\begin{notebox}
When space~5 (XRAM) is the source or destination, the DMA and blitter
controllers read the current XBANK register at the moment the transfer starts.
Changing XBANK mid-transfer is not safe; always set the desired bank before
issuing the command.
\end{notebox}

% -----------------------------------------------------------------------------
\section{Try It Now}
\label{sec:dma-try-it}
% -----------------------------------------------------------------------------

\begin{tryitbox}
The program below combines all four commands to produce a continuously
scrolling rainbow of color bars using only hardware accelerators --- no inner
loops, no per-byte POKEs.

\begin{lstlisting}[style=basiclisting]
10  REM -- DMA & Blitter Demo --
20  REM Clear screen
30  DMAFILL 1, 0, 2000, 32
40  DMAFILL 2, 0, 2000, 0
50  REM Draw 25 horizontal color bars, one per row
60  FOR I = 0 TO 24
70    BLITFILL 2, I*80, 80, 80, 1, I MOD 16
80  NEXT I
90  REM Scroll colors up continuously
100 BLITCOPY 2, 80, 80, 2, 0, 80, 80, 24
110 BLITFILL 2, 24*80, 80, 80, 1, RND(1)*16
120 VSYNC
130 GOTO 100
\end{lstlisting}

Lines~30--40 clear both character and color RAM in two hardware calls.
Lines~60--80 paint 25 horizontal bars of different colors by calling
\cmd{BLITFILL} once per row. Lines~100--130 form an animation loop:
\cmd{BLITCOPY} shifts the entire color layer up by one row, \cmd{BLITFILL}
inserts a new random-color bar at the bottom, and \cmd{VSYNC} holds the
frame rate at 60~Hz.

\medskip
\textbf{Experiments to try:}
\begin{itemize}
  \item Change the fill value on line~30 to something other than 32 (ASCII
        space) to see different characters fill the screen --- try 42
        (asterisk) or 64 (at-sign).
  \item Replace space~2 on lines~70, 100, and~110 with space~1 to scroll
        character codes instead of colors.
  \item Try \cmd{DMACOPY 1, 0, 2, 0, 2000} to mirror the character RAM
        layout into color RAM and see the effect on the display.
  \item Use space~5 (XRAM) as a destination: stash a full color-screen
        snapshot with \cmd{DMACOPY 2, 0, 5, 0, 2000} and restore it later.
\end{itemize}
\end{tryitbox}
