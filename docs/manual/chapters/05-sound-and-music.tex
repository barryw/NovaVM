% =============================================================================
% Chapter 5 — Sound and Music
% NovaBASIC v1.0 User Guide
% =============================================================================
\chapter{Sound and Music}
\label{chap:sound}

\epigraph{\itshape ``Music is the arithmetic of sounds as optics is the
  geometry of light.''}%
         {--- Claude Debussy}

\noindent
NovaBASIC includes a four-voice software synthesizer with independently
configurable waveforms, ADSR volume envelopes, and frequency control per
channel. Audio renders in real time at 44100~Hz through OpenAL. This chapter
explains every command and shows practical patterns for playing notes,
building textures, and composing multi-voice music.

% -----------------------------------------------------------------------------
\section{The Sound System}
\label{sec:sound-overview}
% -----------------------------------------------------------------------------

The synthesizer exposes four identical voices on channels 0 through 3. Each
channel maintains its own:

\begin{itemize}
  \item \textbf{Waveform} --- square, sawtooth, triangle, noise, or sine
        (set with \cmd{WAVE}).
  \item \textbf{ADSR envelope} --- attack, decay, sustain, release shaping
        (set with \cmd{ENVELOPE}).
  \item \textbf{Frequency and duration} --- triggered with \cmd{SOUND}.
\end{itemize}

A single \cmd{VOLUME} command sets the global master output level for all
channels. The typical initialization sequence at the top of a program is:

\begin{lstlisting}[style=basiclisting]
10 VOLUME 12
20 WAVE 0, 0
30 ENVELOPE 0, 2, 4, 10, 4
\end{lstlisting}

Line~10 sets master volume to 12 (out of 15). Line~20 selects a square wave
on channel~0. Line~30 sets a fast attack, moderate decay, high sustain, and
moderate release on channel~0. After this setup, \cmd{SOUND} commands on
channel~0 will play with that voice character until any of these settings
are changed.

% -----------------------------------------------------------------------------
\section{Playing Notes}
\label{sec:playing-notes}
% -----------------------------------------------------------------------------

\subsection*{SOUND syntax}

\begin{center}
\cmd{SOUND channel, frequency, duration}
\end{center}

\begin{itemize}
  \item \cmd{channel} --- voice index 0--3. The value is masked with
        \cmd{\& 0x03} so larger values wrap back into range silently.
  \item \cmd{frequency} --- pitch in Hertz, clamped to 16--12000~Hz.
  \item \cmd{duration} --- note length in 1/60-second ticks. A value of
        60 plays for one second; 30 plays for a half second; 8 plays for
        about 133~milliseconds.
\end{itemize}

If \cmd{frequency} is zero or negative, or \cmd{duration} is zero or
negative, the channel is stopped immediately.

\subsection*{Useful note frequencies}

\begin{center}
\begin{tabular}{lrr}
\toprule
\textbf{Note} & \textbf{Frequency (Hz)} & \textbf{Approx. duration for a beat at 120~bpm} \\
\midrule
Middle C (C4) & 262 & 30 ticks (half second) \\
D4            & 294 & 30 ticks \\
E4            & 330 & 30 ticks \\
F4            & 349 & 30 ticks \\
G4            & 392 & 30 ticks \\
A4            & 440 & 30 ticks \\
B4            & 494 & 30 ticks \\
C5            & 524 & 30 ticks \\
\bottomrule
\end{tabular}
\end{center}

\subsection*{A simple melody}

The following program plays a rising C major scale on channel~0 using a
triangle wave:

\begin{lstlisting}[style=basiclisting]
10 VOLUME 12 : WAVE 0, 2
20 ENVELOPE 0, 1, 4, 12, 4
30 DATA 262, 294, 330, 349, 392, 440, 494, 524
40 FOR N = 1 TO 8
50   READ F
60   SOUND 0, F, 24
70   FOR I = 1 TO 24 : VSYNC : NEXT I
80 NEXT N
\end{lstlisting}

Each note plays for 24 ticks ($\approx 400$~ms) and the \cmd{VSYNC} loop
holds the program for exactly the same duration before triggering the next
note, keeping the melody in rhythm.

\begin{notebox}
The \cmd{duration} parameter in \cmd{SOUND} controls how long the synthesizer
sustains that note in its internal queue; it does not pause program execution.
Use a \cmd{VSYNC} loop after each \cmd{SOUND} call to create the gap between
notes.
\end{notebox}

% -----------------------------------------------------------------------------
\section{Waveforms}
\label{sec:waveforms}
% -----------------------------------------------------------------------------

\cmd{WAVE channel, waveform} selects the oscillator type for a channel. The
waveform index is taken modulo~5, so values 5, 10, etc.\ wrap back to~0.

\begin{center}
\begin{tabular}{clp{0.55\textwidth}}
\toprule
\textbf{Index} & \textbf{Name} & \textbf{Character} \\
\midrule
0 & Square   & Bold, hollow, retro chiptune feel. Classic 8-bit game sound. \\[2pt]
1 & Sawtooth & Buzzy and harmonically rich. Good for brass-like lead sounds
               and bass lines. \\[2pt]
2 & Triangle & Soft and mellow, closer to a flute or a gentle bell.
               Lower-intensity background pads. \\[2pt]
3 & Noise    & Unpitched percussive texture. Use for drum hits, explosions,
               static, and ambient rumble. \\[2pt]
4 & Sine     & Pure, smooth, and clean. No overtones; ideal for test tones
               and ethereal pads. \\
\bottomrule
\end{tabular}
\end{center}

Waveform changes take effect on the next \cmd{SOUND} trigger; they do not
interrupt a note that is already playing on that channel.

% -----------------------------------------------------------------------------
\section{ADSR Envelopes}
\label{sec:envelopes}
% -----------------------------------------------------------------------------

A volume envelope shapes how a note's amplitude changes over time. NovaBASIC
implements a standard four-stage ADSR envelope:

\begin{center}
\cmd{ENVELOPE channel, attack, decay, sustain, release}
\end{center}

\begin{itemize}
  \item \textbf{Attack} --- ramp from silence (0.0) to full amplitude (1.0).
        Byte value; converted to samples at 44100~Hz / 240 per unit.
        A value of~0 gives an instantaneous start; larger values give a
        slower fade-in.
  \item \textbf{Decay} --- ramp from full amplitude down to the sustain
        level. Same timing conversion as attack.
  \item \textbf{Sustain} --- the amplitude held while the note duration
        runs. Only the low nibble is used (0--15), mapped linearly to
        0.0--1.0. A value of 15 sustains at full amplitude after the decay
        phase; a value of~0 gives a purely percussive shape.
  \item \textbf{Release} --- ramp from the sustain level back to silence
        after the note duration expires. Same timing conversion as attack
        and decay.
\end{itemize}

\begin{notebox}
Sustain is a \emph{level} (0--15), not a duration. Attack, decay, and release
are \emph{timing} values (byte duration counts). This is the standard ADSR
model: sustain holds until the note ends, then release fades from there.
\end{notebox}

\subsection*{Envelope comparison}

A short percussive hit vs.\ a slow rising pad:

\begin{lstlisting}[style=basiclisting]
10 VOLUME 12
20 REM PERCUSSIVE: FAST ATTACK, FAST DECAY, NO SUSTAIN, MEDIUM RELEASE
30 WAVE 0, 3
40 ENVELOPE 0, 0, 3, 0, 8
50 SOUND 0, 200, 10
60 FOR I = 1 TO 20 : VSYNC : NEXT I
70 REM SLOW PAD: SLOW ATTACK, MEDIUM DECAY, HIGH SUSTAIN, SLOW RELEASE
80 WAVE 1, 4
90 ENVELOPE 1, 20, 8, 12, 20
100 SOUND 1, 440, 60
110 FOR I = 1 TO 90 : VSYNC : NEXT I
\end{lstlisting}

Channel~0 uses noise (\cmd{WAVE 0,3}) with envelope (0, 3, 0, 8) for a tight
drum-like thud: instant attack, quick decay to zero, short release. Channel~1
uses sine (\cmd{WAVE 1,4}) with envelope (20, 8, 12, 20) for a slow fade-in,
a gentle decay to a sustained level of 12/15, and a long fade-out.

% -----------------------------------------------------------------------------
\section{Multi-Voice Music}
\label{sec:multi-voice}
% -----------------------------------------------------------------------------

Because each channel is independent, you can trigger notes on multiple
channels simultaneously. Issue consecutive \cmd{SOUND} commands on different
channels before calling \cmd{VSYNC}; both will start rendering within the
same frame.

The following example plays a melody on channel~0 and a bass accompaniment
on channel~1:

\begin{lstlisting}[style=basiclisting]
10 VOLUME 12
20 REM MELODY: TRIANGLE, FAST ENVELOPE
30 WAVE 0, 2 : ENVELOPE 0, 1, 4, 10, 4
40 REM BASS: SAWTOOTH, SLOWER ENVELOPE
50 WAVE 1, 1 : ENVELOPE 1, 2, 6, 8, 6
60 REM MELODY FREQUENCIES
70 DATA 330, 294, 262, 294, 330, 330, 330, 0
80 REM BASS FREQUENCIES (HALF SPEED - ONE PER TWO MELODY NOTES)
90 DATA 131, 131, 131, 131
100 FOR N = 1 TO 8
110   READ MF
120   IF (N AND 1) = 1 THEN READ BF : SOUND 1, BF, 48
130   SOUND 0, MF, 24
140   FOR I = 1 TO 24 : VSYNC : NEXT I
150 NEXT N
\end{lstlisting}

Lines~70--90 define the note sequences as \cmd{DATA}. The melody plays a
new note every 24 ticks. The bass plays one note every 48 ticks by reading
a bass frequency only on odd melody steps (line~120 tests bit~0 of \cmd{N}).
Both channels render concurrently between \cmd{VSYNC} calls.

\begin{tipbox}
Channels do not block each other. A note triggered on channel~0 and a note
triggered on channel~1 in the same frame play simultaneously with no
additional effort.
\end{tipbox}

% -----------------------------------------------------------------------------
\section{Composition Tips}
\label{sec:composition-tips}
% -----------------------------------------------------------------------------

\begin{tipbox}
\begin{itemize}
  \item Set \cmd{VOLUME} and all \cmd{ENVELOPE}/\cmd{WAVE} values once at
        program startup, not inside the note loop. Changing them mid-note
        can cause audible clicks.
  \item Keep melody, harmony, and percussion on separate channels so each
        voice can have its own waveform and envelope character.
  \item Use \cmd{VSYNC} loops between note events for tempo control.
        Each \cmd{VSYNC} is exactly 1/60 second. At 120 beats per minute
        (2 beats/second), one beat = 30 VSYNC cycles.
  \item Store music as \cmd{DATA} statements. This separates the note data
        from the playback engine and makes it easy to edit pitches,
        durations, and timing without touching the logic.
  \item Use waveform~3 (noise) on a spare channel for drum and percussion
        sounds. A short envelope (fast attack, zero sustain) on the noise
        channel gives convincing hi-hat and snare textures.
\end{itemize}
\end{tipbox}

% -----------------------------------------------------------------------------
\section{Try It Now}
\label{sec:sound-tryit}
% -----------------------------------------------------------------------------

\begin{tryitbox}
Type and run the following program. It uses a sine wave with a smooth
envelope to play an ascending chromatic phrase of eight notes:

\begin{lstlisting}[style=basiclisting]
10 VOLUME 12
20 ENVELOPE 0, 2, 6, 10, 6
30 WAVE 0, 4
40 FOR N = 0 TO 7
50   F = 262 * 2 ^ (N / 12)
60   SOUND 0, INT(F), 8
70   FOR I = 1 TO 8 : VSYNC : NEXT I
80 NEXT N
\end{lstlisting}

Expected result: eight notes rising by one semitone each, starting at
middle~C (262~Hz) and ascending chromatically, played with a sine wave and a
smooth ADSR envelope. Each note lasts approximately 133~milliseconds (8/60~s).

\medskip
Experiments to try:

\begin{itemize}
  \item Change \cmd{WAVE 0,4} to \cmd{WAVE 0,0} (square) and re-run to
        hear the same phrase with a retro chiptune character.
  \item Change the base frequency from 262 to 440 to start the phrase from
        concert A instead of middle~C.
  \item Increase the loop counter from 7 to 11 for a full chromatic
        octave.
\end{itemize}
\end{tryitbox}
