% =============================================================================
% Chapter 5 — Sound and Music
% NovaBASIC v1.0 User Guide
% Source of truth: ehbasic/basic.asm (SOUND, INSTRUMENT, VOLUME, MUSIC tokens)
%                  e6502.Avalonia/Hardware/SidChip.cs
%                  e6502.Avalonia/Hardware/MusicEngine.cs
%                  e6502.Avalonia/Hardware/MmlParser.cs
%                  e6502.Avalonia/Hardware/FileIoController.cs
% =============================================================================
\chapter{Sound and Music}
\label{chap:sound}

\epigraph{\itshape ``Music is the arithmetic of sounds as optics is the
  geometry of light.''}%
         {--- Claude Debussy}

\noindent
NovaBASIC includes a SID chip emulator --- a software recreation of the MOS
6581 Sound Interface Device made famous by the Commodore~64. Three independent
voices with four waveforms, ADSR envelopes, and a programmable filter deliver
authentic chiptune sound. On top of the SID sits a three-voice MML music
sequencer with per-frame effects including vibrato, portamento, arpeggios,
pulse-width modulation, and filter sweeps.

This chapter covers every sound command from simple one-shot notes to
full multi-voice compositions.

% -----------------------------------------------------------------------------
\section{Quick-Start Overview}
\label{sec:sound-overview}
% -----------------------------------------------------------------------------

The sound system has three layers, each building on the one below:

\begin{enumerate}
  \item \textbf{SOUND} --- play a single note on the SID chip. Specify a MIDI
        note number, duration in frames, and an optional instrument preset.
  \item \textbf{INSTRUMENT} --- define a reusable preset that sets the SID
        waveform and ADSR envelope. Up to 16 presets (slots 0--15).
  \item \textbf{MUSIC} --- load MML (Music Macro Language) sequences into up
        to three voices and play them back with tempo, looping, and
        per-frame effects.
\end{enumerate}

A minimal program that plays a note:

\begin{lstlisting}[style=basiclisting]
10 VOLUME 12
20 SOUND 60, 30
\end{lstlisting}

Line~10 sets master volume. Line~20 plays MIDI note~60 (middle~C) for
30 frames (half a second at 60~Hz).

% -----------------------------------------------------------------------------
\section{The SOUND Command}
\label{sec:sound-command}
% -----------------------------------------------------------------------------

\begin{center}
\cmd{SOUND note, duration {[}, instrument{]}}
\end{center}

\begin{itemize}
  \item \cmd{note} --- MIDI note number (0--127). Middle~C is 60; A4
        (concert pitch 440~Hz) is 69. See the MIDI note table below.
  \item \cmd{duration} --- length in 1/60-second frames. A value of 60
        plays for one second; 30 plays for half a second.
  \item \cmd{instrument} --- optional instrument preset (0--15). If omitted,
        instrument~0 is used.
\end{itemize}

If \cmd{note} or \cmd{duration} is zero, the sound is stopped immediately.

\begin{notebox}
\cmd{SOUND} triggers a one-shot sound effect through the music engine's
SFX channel. It does not interrupt music playback; the engine allocates a
voice for the effect and restores it when the sound completes.
\end{notebox}

\subsection*{Common MIDI note numbers}

\begin{center}
\begin{tabular}{lrl}
\toprule
\textbf{Note} & \textbf{MIDI} & \textbf{Approx.\ Frequency} \\
\midrule
C3  & 48 & 131 Hz \\
C4 (Middle C) & 60 & 262 Hz \\
D4  & 62 & 294 Hz \\
E4  & 64 & 330 Hz \\
F4  & 65 & 349 Hz \\
G4  & 67 & 392 Hz \\
A4  & 69 & 440 Hz \\
B4  & 71 & 494 Hz \\
C5  & 72 & 523 Hz \\
C6  & 84 & 1047 Hz \\
\bottomrule
\end{tabular}
\end{center}

\medskip
The formula is: frequency $= 440 \times 2^{(\text{midi} - 69) / 12}$.

\subsection*{A simple melody}

\begin{lstlisting}[style=basiclisting]
10 VOLUME 12
20 DATA 60, 62, 64, 65, 67, 69, 71, 72
30 FOR N = 1 TO 8
40   READ M
50   SOUND M, 15
60   FOR I = 1 TO 15 : VSYNC : NEXT I
70 NEXT N
\end{lstlisting}

Each note plays for 15 frames ($\approx 250$~ms). The \cmd{VSYNC} loop
holds the program for the same duration before the next note fires.

\begin{notebox}
\cmd{SOUND} does not block program execution. Use a \cmd{VSYNC} loop after
each \cmd{SOUND} call to create the gap between notes.
\end{notebox}

% -----------------------------------------------------------------------------
\section{The INSTRUMENT Command}
\label{sec:instrument}
% -----------------------------------------------------------------------------

\begin{center}
\cmd{INSTRUMENT id, waveform, attack, decay, sustain, release}
\end{center}

Defines a reusable sound preset in one of 16 instrument slots.

\begin{itemize}
  \item \cmd{id} --- slot number (0--15).
  \item \cmd{waveform} --- SID waveform byte:
        \texttt{\$10}~=~triangle, \texttt{\$20}~=~sawtooth,
        \texttt{\$40}~=~pulse, \texttt{\$80}~=~noise.
  \item \cmd{attack} --- attack rate (0--15). 0 is instantaneous; 15 is
        slowest.
  \item \cmd{decay} --- decay rate (0--15). How quickly the volume drops
        from peak to the sustain level.
  \item \cmd{sustain} --- sustain level (0--15). The steady-state volume
        held while the note plays. 15~=~full volume; 0~=~silent (percussive).
  \item \cmd{release} --- release rate (0--15). How quickly the volume fades
        to silence after the note ends.
\end{itemize}

\begin{lstlisting}[style=basiclisting]
10 REM BRIGHT PULSE LEAD
20 INSTRUMENT 0, $40, 0, 9, 0, 6
30 REM WARM SAWTOOTH PAD
40 INSTRUMENT 1, $20, 4, 6, 12, 8
50 REM NOISE DRUM HIT
60 INSTRUMENT 2, $80, 0, 3, 0, 2
\end{lstlisting}

\begin{notebox}
Instrument~0 is pre-initialized at boot with: pulse waveform (\texttt{\$40}),
attack~0, decay~9, sustain~0, release~6, pulse width~2048. All other slots
(1--15) start as copies of slot~0.
\end{notebox}

\subsection*{SID waveform reference}

\begin{center}
\begin{tabular}{clp{0.50\textwidth}}
\toprule
\textbf{Value} & \textbf{Name} & \textbf{Character} \\
\midrule
\texttt{\$10} & Triangle & Soft and mellow; flute-like. Good for gentle
  melodies and background pads. \\[2pt]
\texttt{\$20} & Sawtooth & Buzzy and harmonically rich. Good for brass-like
  leads and bass lines. \\[2pt]
\texttt{\$40} & Pulse    & Bold, hollow, classic chiptune sound. Pulse width
  can be modulated via MML for evolving timbres. \\[2pt]
\texttt{\$80} & Noise    & Unpitched random output. Use for drums, hi-hats,
  explosions, and ambient textures. \\
\bottomrule
\end{tabular}
\end{center}

\subsection*{ADSR envelope overview}

The four parameters shape how a note's volume changes over time:

\begin{enumerate}
  \item \textbf{Attack} ramps from silence to full amplitude.
  \item \textbf{Decay} drops from full amplitude to the sustain level.
  \item \textbf{Sustain} holds at a constant level while the note plays.
  \item \textbf{Release} fades from the sustain level to silence after the
        note ends.
\end{enumerate}

Sustain is a \emph{level} (0--15); the other three are \emph{rate} values
where 0 is fastest and 15 is slowest. This matches the original SID chip
behavior.

\subsection*{Instrument recipes}

\begin{center}
\begin{tabular}{lcccccp{0.28\textwidth}}
\toprule
\textbf{Sound} & \textbf{Wave} & \textbf{A} & \textbf{D} & \textbf{S} & \textbf{R} & \textbf{Notes} \\
\midrule
Chiptune lead   & \$40 &  0 &  9 &  0 &  6 & Sharp attack, no sustain \\
Warm pad        & \$20 &  8 &  6 & 12 & 10 & Slow fade-in, high sustain \\
Bass            & \$20 &  0 &  5 &  8 &  4 & Instant attack, medium body \\
Snare drum      & \$80 &  0 &  3 &  0 &  2 & Short noise burst \\
Hi-hat          & \$80 &  0 &  1 &  0 &  1 & Very short noise tick \\
Organ           & \$10 &  0 &  0 & 15 &  4 & Triangle at full sustain \\
Pluck           & \$40 &  0 & 12 &  0 &  8 & Fast decay, no sustain \\
\bottomrule
\end{tabular}
\end{center}

% -----------------------------------------------------------------------------
\section{VOLUME}
\label{sec:volume}
% -----------------------------------------------------------------------------

\begin{center}
\cmd{VOLUME level}
\end{center}

Sets the SID master volume. \cmd{level} is 0--15 (only the low nibble is
used). The default volume at boot is 12.

% -----------------------------------------------------------------------------
\section{The MUSIC Engine}
\label{sec:music-engine}
% -----------------------------------------------------------------------------

The music engine is a three-voice MML sequencer running on top of the SID
chip. You write melodies and rhythms as text strings using Music Macro
Language, load them into voices, and let the engine handle all the timing,
instrument switching, and per-frame effects automatically.

\subsection{Loading and Playing Sequences}

\begin{center}
\cmd{MUSIC voice, "mml-string"}
\end{center}

\begin{itemize}
  \item \cmd{voice} --- voice number 1--3.
  \item \cmd{"mml-string"} --- an MML sequence (see Section~\ref{sec:mml}).
\end{itemize}

Additional subcommands control playback:

\begin{center}
\begin{tabular}{>{\ttfamily}p{0.36\textwidth} p{0.54\textwidth}}
\toprule
\textbf{Command} & \textbf{Description} \\
\midrule
MUSIC PLAY             & Start playback of all loaded voices. \\[3pt]
MUSIC STOP             & Stop playback and silence all music voices. \\[3pt]
MUSIC TEMPO bpm        & Set tempo in beats per minute. Default is 120. \\[3pt]
MUSIC LOOP ON          & Enable looping; voices restart when all finish. \\[3pt]
MUSIC LOOP OFF         & Disable looping (default). \\[3pt]
MUSIC PRIORITY v1{[},v2{[},v3{]}} & Set voice-stealing priority for sound
  effects. Lower-numbered voices are stolen first. Default: 3,~2,~1. \\
\bottomrule
\end{tabular}
\end{center}

\subsection{A Complete Music Example}

\begin{lstlisting}[style=basiclisting]
10 VOLUME 12
20 REM DEFINE INSTRUMENTS
30 INSTRUMENT 0, $40, 0, 9, 0, 6
40 INSTRUMENT 1, $20, 0, 5, 8, 4
50 REM LOAD VOICES
60 MUSIC 1, "T140 I0 L8 O4 CDEFGAB >C"
70 MUSIC 2, "T140 I1 L4 O3 C G C G"
80 REM START PLAYBACK
90 MUSIC LOOP ON
100 MUSIC PLAY
\end{lstlisting}

Line~60 loads a melody into voice~1: tempo~140, instrument~0, eighth notes,
octave~4, ascending C~major scale. Line~70 loads a bass line into voice~2:
quarter notes, octave~3, alternating C~and~G. Line~100 starts playback;
\cmd{MUSIC LOOP ON} on line~90 means the music repeats indefinitely.

\subsection{Querying Music Status}

Two functions let you check what the music engine is doing:

\begin{center}
\begin{tabular}{>{\ttfamily}p{0.22\textwidth} p{0.65\textwidth}}
\toprule
\textbf{Function} & \textbf{Returns} \\
\midrule
PLAYING & 1 if music is currently playing, 0 if stopped. \\[3pt]
MNOTE(voice) & The MIDI note number currently sounding on \cmd{voice}
  (1--3), or 0 if that voice is silent. \\
\bottomrule
\end{tabular}
\end{center}

\begin{lstlisting}[style=basiclisting]
200 IF PLAYING THEN GOTO 200
210 PRINT "MUSIC FINISHED"
\end{lstlisting}

% =============================================================================
\section{MML Reference}
\label{sec:mml}
% =============================================================================

Music Macro Language (MML) is a compact text notation for music. Each voice
receives its own MML string. The parser is case-insensitive; all input is
converted to uppercase before processing.

\subsection{Notes and Rests}

\begin{center}
\begin{tabular}{>{\ttfamily}p{0.24\textwidth} p{0.64\textwidth}}
\toprule
\textbf{Syntax} & \textbf{Description} \\
\midrule
C D E F G A B & Play a note. Pitch is determined by the current octave. \\[3pt]
C\# \textnormal{or} C+ & Sharp (raise one semitone). \\[3pt]
C- & Flat (lower one semitone). \\[3pt]
R & Rest (silence for the note duration). \\
\bottomrule
\end{tabular}
\end{center}

\subsection{Duration}

A number following a note or rest sets its length as a note-value denominator:

\begin{center}
\begin{tabular}{crl}
\toprule
\textbf{Denominator} & \textbf{Ticks} & \textbf{Name} \\
\midrule
1  & 384 & Whole note \\
2  & 192 & Half note \\
4  &  96 & Quarter note \\
8  &  48 & Eighth note \\
16 &  24 & Sixteenth note \\
32 &  12 & Thirty-second note \\
\bottomrule
\end{tabular}
\end{center}

Internally, one quarter note equals 96 ticks.

A dot (\cmd{.}) after the duration extends it by 50\%: \cmd{C4.} plays for
$96 \times 1.5 = 144$ ticks (dotted quarter).

If no duration is given, the default length set by \cmd{L} is used (initially
a quarter note).

\subsection{Ties}

The ampersand (\cmd{\&}) ties two durations together into a single sustained
note:

\begin{center}
\cmd{C4\&8} $\rightarrow$ quarter + eighth = 144 ticks
\end{center}

Multiple ties can be chained: \cmd{C4\&4\&4} plays for $96 + 96 + 96 = 288$
ticks. A single \texttt{NoteOn} event is emitted with the combined duration.

\subsection{Octave}

\begin{center}
\begin{tabular}{>{\ttfamily}p{0.12\textwidth} p{0.76\textwidth}}
\toprule
\textbf{Cmd} & \textbf{Description} \\
\midrule
O4   & Set absolute octave (range 1--7; default 4). \\
>    & Octave up (clamped to 7). \\
<    & Octave down (clamped to 1). \\
\bottomrule
\end{tabular}
\end{center}

MIDI note calculation: \texttt{midi = (octave + 1) * 12 + semitone}, where
C=0, D=2, E=4, F=5, G=7, A=9, B=11.

\subsection{Default Length}

\cmd{L8} sets the default note/rest duration to eighth notes. All subsequent
notes and rests that omit an explicit duration will use this value.

\subsection{Tempo}

\cmd{T120} sets the tempo to 120 beats per minute. The default is 120.
Tempo can appear anywhere in the MML string and takes effect immediately.
At 120~BPM, one quarter note lasts exactly 0.5~seconds.

\begin{notebox}
Tempo is global. If multiple voices contain \cmd{T} commands, the last one
processed wins. It is best practice to set tempo in voice~1 only.
\end{notebox}

\subsection{Instrument Selection}

\cmd{I3} switches the current voice to instrument slot~3 (defined earlier
with the \cmd{INSTRUMENT} BASIC command). Instrument changes take effect on
the next note.

\subsection{Loops}

Square brackets repeat a section:

\begin{center}
\cmd{{[}CDEF{]}3} $\rightarrow$ plays C~D~E~F three times
\end{center}

The repeat count follows the closing bracket. If omitted, the default is~1
(no repetition). Loops do not nest.

\subsection{Arpeggios}

Curly braces define an arpeggio --- a rapid cycling through multiple notes:

\begin{center}
\cmd{\{CEG\}4} $\rightarrow$ cycle C, E, G at 60~Hz for one quarter note
\end{center}

Each frame advances to the next note in the list. Accidentals are supported
inside the braces (\cmd{\{C\#EG\#\}}). The arpeggio duration follows the
closing brace using standard duration syntax.

% =============================================================================
\subsection{Per-Frame Effects}
\label{sec:mml-effects}
% =============================================================================

The music engine processes the following effects on every frame (60~Hz).
These effects are set within MML and remain active until changed or a new
note resets them.

\subsubsection*{Vibrato}

\cmd{\textasciitilde6} sets vibrato depth to~6. Higher values produce wider
pitch oscillation. \cmd{\textasciitilde0} turns vibrato off.

The vibrato oscillates at approximately 2.9~Hz (sine wave). The pitch offset
is proportional to both the depth value and the current note frequency.

\subsubsection*{Portamento (Pitch Slide)}

\cmd{/} before a note causes the voice to slide from the current pitch to
the target note rather than jumping instantly.

\begin{center}
\cmd{C4 /E4} $\rightarrow$ play C, then glide smoothly to E
\end{center}

The slide rate is approximately $\frac{1}{8}$ of the frequency distance per
frame.

\subsubsection*{Pulse Width}

\cmd{@P2048} sets the SID pulse width to 2048 (range 0--4095). This only
affects the pulse waveform (\texttt{\$40}). A value of 2048 gives a 50\%
duty cycle (square wave); lower or higher values create thinner, more nasal
timbres.

\subsubsection*{Pulse Width Modulation (PWM)}

\cmd{@PS+} starts sweeping the pulse width upward; \cmd{@PS-} sweeps
downward; \cmd{@PS0} stops the sweep. The sweep rate is $\pm$32 per frame,
clamped to the 0--4095 range. PWM gives the pulse waveform a rich,
evolving character.

\begin{lstlisting}[style=basiclisting]
60 MUSIC 1, "@P1024 @PS+ O4 L2 C E G >C"
\end{lstlisting}

\subsubsection*{Filter Cutoff and Resonance}

\cmd{@F1024,8} sets the SID filter cutoff to 1024 (range 0--2047) and
resonance to 8 (range 0--15). The resonance parameter is optional; if
omitted, it defaults to~0.

\subsubsection*{Filter Mode}

\begin{center}
\begin{tabular}{>{\ttfamily}p{0.08\textwidth} p{0.50\textwidth}}
\toprule
\textbf{Cmd} & \textbf{Mode} \\
\midrule
@FL & Low-pass (cuts highs, warm sound) \\
@FB & Band-pass (emphasizes a frequency band) \\
@FH & High-pass (cuts lows, thin sound) \\
@FO & Filter off \\
\bottomrule
\end{tabular}
\end{center}

\subsubsection*{Filter Sweep}

\cmd{@FS+} sweeps the filter cutoff upward; \cmd{@FS-} sweeps downward;
\cmd{@FS0} stops the sweep. The sweep rate is $\pm$8 per frame, clamped to
0--2047.

\begin{lstlisting}[style=basiclisting]
60 MUSIC 1, "@FL @F200,12 @FS+ L4 O3 [CDEFGAB>C<]2"
\end{lstlisting}

This creates a classic filter sweep effect: low-pass filter starting at
cutoff~200 with high resonance, sweeping upward through the melody.

% =============================================================================
\subsection{MML Command Summary}
\label{sec:mml-summary}
% =============================================================================

\begin{longtable}{>{\ttfamily\small}p{0.22\textwidth} p{0.20\textwidth} p{0.44\textwidth}}
\toprule
\textbf{Command} & \textbf{Parameters} & \textbf{Description} \\
\midrule
\endhead
\bottomrule
\endfoot
A--G        & {[}\#/+/-{]}{[}dur{]}{[}.{]} & Play note \\[2pt]
R           & {[}dur{]}{[}.{]}   & Rest \\[2pt]
O           & 1--7               & Set octave \\[2pt]
>           & ---                & Octave up \\[2pt]
<           & ---                & Octave down \\[2pt]
L           & denominator        & Default note length \\[2pt]
T           & bpm                & Tempo (default 120) \\[2pt]
I           & 0--15              & Select instrument slot \\[2pt]
\&          & {[}note{]}dur      & Tie (extend note duration) \\[2pt]
{[}\ldots{]}n & repeat count     & Loop section $n$ times \\[2pt]
\{notes\}   & {[}dur{]}{[}.{]}   & Arpeggio \\[2pt]
\textasciitilde & depth (0=off)  & Vibrato \\[2pt]
/           & ---                & Portamento (next note slides) \\[2pt]
@P          & 0--4095            & Set pulse width \\[2pt]
@PS         & +, -, 0            & PWM sweep direction \\[2pt]
@F          & cutoff{[},res{]}   & Filter cutoff (0--2047) and resonance (0--15) \\[2pt]
@FL         & ---                & Low-pass filter \\[2pt]
@FB         & ---                & Band-pass filter \\[2pt]
@FH         & ---                & High-pass filter \\[2pt]
@FO         & ---                & Filter off \\[2pt]
@FS         & +, -, 0            & Filter sweep direction \\
\end{longtable}

Whitespace, tabs, newlines, and pipe characters (\cmd{|}) are ignored and
can be used freely to format MML strings for readability.

% =============================================================================
\section{SID File Playback}
\label{sec:sidplay}
% =============================================================================

NovaBASIC can load and play standard \texttt{.sid} files --- the native
music format of the Commodore~64 scene:

\begin{center}
\begin{tabular}{>{\ttfamily}p{0.36\textwidth} p{0.54\textwidth}}
\toprule
\textbf{Command} & \textbf{Description} \\
\midrule
SIDPLAY "filename" {[}, song{]} & Load and play a \texttt{.sid} file. The
  optional \cmd{song} parameter selects which sub-tune to play (default~1). \\[3pt]
SIDSTOP & Stop SID file playback. \\
\bottomrule
\end{tabular}
\end{center}

\begin{lstlisting}[style=basiclisting]
10 SIDPLAY "commando"
20 FOR I = 1 TO 600 : VSYNC : NEXT I
30 SIDSTOP
\end{lstlisting}

SID files are loaded from the \texttt{\textasciitilde/e6502-programs}
directory. The \texttt{.sid} extension is added automatically. The SID
player injects an IRQ trampoline into CPU RAM that calls the file's init
and play routines at 60~Hz.

\begin{warningbox}
SID playback takes over the SID chip directly. \cmd{SOUND} and \cmd{MUSIC}
commands will not produce audible output while a SID file is playing. Call
\cmd{SIDSTOP} before using other sound commands.
\end{warningbox}

% =============================================================================
\section{Graphics File I/O}
\label{sec:gsave-gload}
% =============================================================================

NovaBASIC can save and load VGC memory spaces to disk:

\begin{center}
\begin{tabular}{>{\ttfamily}p{0.42\textwidth} p{0.48\textwidth}}
\toprule
\textbf{Command} & \textbf{Description} \\
\midrule
GSAVE "name", space, offset, len & Save \cmd{len} bytes from VGC memory
  space starting at \cmd{offset} to a \texttt{.gfx} file. \\[3pt]
GLOAD "name", space, offset{[}, len{]} & Load a \texttt{.gfx} file into
  VGC memory space at \cmd{offset}. If \cmd{len} is omitted, the entire
  file is loaded. \\
\bottomrule
\end{tabular}
\end{center}

VGC memory spaces:

\begin{center}
\begin{tabular}{cl}
\toprule
\textbf{Space} & \textbf{Contents} \\
\midrule
0 & Character RAM (2000 bytes) \\
1 & Color RAM (2000 bytes) \\
2 & Graphics bitmap (64000 bytes) \\
3 & Sprite shape RAM (2048 bytes) \\
\bottomrule
\end{tabular}
\end{center}

\begin{lstlisting}[style=basiclisting]
10 MODE 1 : GCLS
20 GCOLOR 10 : CIRCLE 160, 100, 80
30 GSAVE "mycircle", 2, 0, 64000
40 REM LATER...
50 GLOAD "mycircle", 2, 0
\end{lstlisting}

% =============================================================================
\section{Music Engine Architecture}
\label{sec:music-architecture}
% =============================================================================

For advanced users, understanding the engine internals helps write better
music and avoid common pitfalls.

\subsection*{Voice allocation}

The music engine manages three music voices (mapped to SID voices 0--2) plus
one shared SFX voice. When \cmd{SOUND} triggers a sound effect, the engine:

\begin{enumerate}
  \item Looks for a voice with no music sequence loaded.
  \item If all voices have sequences, steals a voice according to the
        priority order (default: voice~3 first, then~2, then~1).
  \item Plays the SFX on the stolen voice; when done, restores the music
        voice.
\end{enumerate}

\subsection*{Timing}

The engine ticks at 60~Hz. Tempo is converted to ticks per frame:

\begin{center}
$\text{ticks per frame} = \frac{96 \times \text{BPM}}{3600}$
\end{center}

At 120~BPM this is 3.2 ticks per frame. A quarter note (96~ticks) takes
exactly 30 frames = 0.5~seconds.

\subsection*{Effect processing order}

Each frame, active effects are processed in this order:

\begin{enumerate}
  \item Arpeggio (cycle to next note)
  \item PWM sweep ($\pm$32 per frame, clamped 0--4095)
  \item Vibrato (sine wave at $\sim$2.9~Hz)
  \item Portamento (slide $\frac{1}{8}$ of remaining distance per frame)
  \item Filter sweep ($\pm$8 per frame, clamped 0--2047)
\end{enumerate}

% =============================================================================
\section{Composition Tips}
\label{sec:composition-tips}
% =============================================================================

\begin{tipbox}
\begin{itemize}
  \item Define all instruments before loading music sequences.
  \item Use \cmd{I} in MML to switch instruments mid-voice for timbral
        variety.
  \item Keep melody, harmony, and bass on separate voices. Each voice has
        its own instrument, octave, and effect state.
  \item Use \cmd{|} characters in MML strings as bar-line separators for
        readability: \cmd{"L8 CDEF | GABC"}.
  \item At 120~BPM: quarter~=~30 frames, eighth~=~15, sixteenth~=~7.5. Use
        \cmd{T} to control tempo rather than adjusting note lengths.
  \item Use \cmd{@PS+} and \cmd{@PS-} on pulse waveforms for rich, evolving
        textures.
  \item Combine \cmd{@FL} with \cmd{@FS+} for classic acid-bass filter
        sweeps.
  \item The noise waveform (\texttt{\$80}) with short ADSR makes convincing
        drums. Load a noise instrument and trigger it with \cmd{SOUND}
        while the music plays.
\end{itemize}
\end{tipbox}

% =============================================================================
\section{Deprecated Commands}
\label{sec:deprecated-sound}
% =============================================================================

\begin{warningbox}
The following commands from earlier versions have been superseded:

\begin{itemize}
  \item \cmd{WAVE} --- raises a syntax error. Use \cmd{INSTRUMENT} instead.
  \item \cmd{ENVELOPE} --- replaced by \cmd{INSTRUMENT}. Programs should be
        updated to use the new six-parameter syntax.
\end{itemize}
\end{warningbox}

% =============================================================================
\section{Try It Now}
\label{sec:sound-tryit}
% =============================================================================

\begin{tryitbox}
Type and run the following program to hear a three-voice arrangement with
instrument presets, MML sequences, filter effects, and looping:

\begin{lstlisting}[style=basiclisting]
10 VOLUME 12
20 INSTRUMENT 0, $40, 0, 9, 0, 6
30 INSTRUMENT 1, $20, 0, 5, 8, 4
40 INSTRUMENT 2, $80, 0, 3, 0, 2
50 MUSIC 1, "T120 I0 L8 O4 CDEFGAB >C2"
60 MUSIC 2, "T120 I1 L4 O3 C G C G"
70 MUSIC 3, "T120 I2 L8 O4 R C R C R C R C"
80 MUSIC LOOP ON
90 MUSIC PLAY
\end{lstlisting}

Expected result: a three-voice loop with a pulse-wave melody, sawtooth bass,
and noise percussion. The music repeats until you type \cmd{MUSIC STOP} in
direct mode.

\medskip
Experiments:
\begin{itemize}
  \item Change \cmd{T120} to \cmd{T180} for a faster tempo.
  \item Add vibrato to voice~1: change the MML to start with
        \cmd{"T120 I0 \textasciitilde4 L8 O4 ..."}.
  \item Add a filter sweep to voice~2:
        \cmd{"T120 I1 @FL @F200,10 @FS+ L4 O3 C G C G"}.
\end{itemize}
\end{tryitbox}
