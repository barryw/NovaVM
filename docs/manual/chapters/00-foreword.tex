% =============================================================================
% NovaBASIC v1.0 User Guide — Foreword
% =============================================================================
\chapter*{Foreword}
\addcontentsline{toc}{chapter}{Foreword}
\markboth{Foreword}{Foreword}

There is something quietly extraordinary about a chip that fits in the palm of
your hand and changed the world.  The MOS 6502, introduced in 1975, powered the
Apple~II, the Commodore~64, the Nintendo Entertainment System, the Atari~2600,
and the BBC Micro.  It put a computer on the desk of a generation that had never
seen one before.  It was cheap enough for a hobbyist to buy, simple enough to
understand completely, and fast enough to make things happen on screen that felt
genuinely alive.  Whole careers began on machines built around that little chip.
So did countless imaginations.

The years moved on.  Processors grew faster, wider, and vastly more complex.
But the 6502 never really went away --- it kept shipping in embedded systems,
and it kept living in the memories of the people who learned on it.  There is a
reason emulators, homebrew cartridges, and new assembler projects for the 6502
keep appearing in~2026: programming it is a complete, intimate experience.  You
can hold the whole machine in your head at once.

The e6502 virtual computer is not a museum exhibit.  It takes the honest
architecture of the original --- 64\,KB of address space, stack at
\texttt{\$0100}--\texttt{\$01FF}, clean interrupt model --- and pairs it with
hardware that no real 8-bit machine ever had: a 320$\times$200 pixel bitmap with
16 colors, sixteen hardware sprites each 16$\times$16 pixels and multicolor, a
four-voice synthesizer with ADSR envelopes and five waveforms, 512\,KB of banked
expansion memory, and file I/O.  It is the machine you always wished you had
back then.

NovaBASIC is the language that lives on it.  Built on Lee Davison's
\textit{Enhanced BASIC~2.22p5} --- itself a carefully crafted piece of work ---
NovaBASIC keeps everything that made classic BASIC approachable: line numbers,
\cmd{PRINT}, \cmd{FOR}/\cmd{NEXT}, \cmd{GOSUB}, the immediacy of typing a line
and seeing it run.  Then it adds the commands to drive all that new hardware,
without hiding the machine from you.

Whether you are learning to program for the first time, building a retro game,
exploring procedural graphics, or just looking for a creative sandbox that
rewards curiosity --- this is the right place.  Boot it up.  Type something.
Watch it run.

\medskip
\noindent Welcome.

\bigskip
{\raggedleft\color{retrogray}\itshape February 2026\par}
