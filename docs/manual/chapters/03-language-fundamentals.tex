% =============================================================================
% NovaBASIC v1.0 User Guide — Chapter 3: Language Fundamentals
% =============================================================================
\chapter{Language Fundamentals}

\epigraph{%
  Simplicity is the ultimate sophistication.
}{%
  \textit{Leonardo da Vinci}%
}

NovaBASIC inherits a rich language from EhBASIC~2.22p5 and extends it with
hardware-specific commands.  This chapter covers the core language features
you will use in almost every program: variables, arrays, control flow,
operators, built-in functions, and direct memory access.

% ---------------------------------------------------------------------------
\section{Variables}

NovaBASIC has two kinds of variable: \textbf{numeric} and \textbf{string}.

\begin{description}
  \item[Numeric variables] hold floating-point numbers.  Names begin with a
    letter and may contain letters and digits (e.g.\ \texttt{A},
    \texttt{SCORE}, \texttt{X1}, \texttt{HITCOUNT}).
  \item[String variables] hold text.  Names end with a dollar sign
    (e.g.\ \texttt{N\$}, \texttt{NAME\$}, \texttt{MSG\$}).
\end{description}

Variable names are \textbf{case-insensitive}: \texttt{SCORE} and
\texttt{score} refer to the same variable.  Numeric variables default to
\texttt{0}; string variables default to the empty string.  You do not need to
declare a variable before using it.

\begin{lstlisting}[style=basiclisting]
10 SCORE = 0
20 NAME$ = "PLAYER ONE"
30 LIVES = 3
40 PRINT NAME$; " HAS "; LIVES; " LIVES"
50 SCORE = SCORE + 100
60 PRINT "SCORE: "; SCORE
\end{lstlisting}

% ---------------------------------------------------------------------------
\section{Arrays}

Use \cmd{DIM} to declare an array before using it.  The default lower bound
is~0, so \cmd{DIM A(10)} creates 11 elements: \texttt{A(0)} through
\texttt{A(10)}.

\begin{lstlisting}[style=basiclisting]
10 DIM A(10)
20 FOR I=0 TO 10
30   A(I) = I * I
40 NEXT I
50 FOR I=0 TO 10
60   PRINT "A(";I;") = ";A(I)
70 NEXT I
\end{lstlisting}

\noindent
Two-dimensional arrays work the same way:

\begin{lstlisting}[style=basiclisting]
10 DIM GRID(7,7)
20 FOR R=0 TO 7
30   FOR C=0 TO 7
40     GRID(R,C) = R*8 + C
50   NEXT C
60 NEXT R
\end{lstlisting}

\begin{notebox}
  String arrays are also supported: \cmd{DIM LABEL\$(9)} creates ten
  string slots.
\end{notebox}

% ---------------------------------------------------------------------------
\section{Control Flow}

\subsection{FOR / NEXT Loops}

The workhorse of NovaBASIC iteration.  The optional \cmd{STEP} clause sets
the increment; if omitted, it defaults to~1.

\begin{lstlisting}[style=basiclisting]
10 FOR I = 1 TO 10
20   PRINT I; " ";
30 NEXT I
40 PRINT
50 REM count down with STEP
60 FOR N = 10 TO 1 STEP -1
70   PRINT N; " ";
80 NEXT N
\end{lstlisting}

\subsection{IF / THEN / ELSE}

\begin{lstlisting}[style=basiclisting]
10 INPUT "ENTER A NUMBER: "; N
20 IF N > 100 THEN PRINT "BIG" ELSE PRINT "SMALL"
30 IF N = 42 THEN PRINT "THE ANSWER"
\end{lstlisting}

\noindent
\cmd{THEN} may be followed by a statement or a line number
(as a \cmd{GOTO} shorthand).  \cmd{ELSE} is optional.

\subsection{DO / LOOP}

\cmd{DO}/\cmd{LOOP} supports four variants for flexible looping:

\begin{lstlisting}[style=basiclisting]
10 X = 1
20 DO
30   PRINT X
40   X = X + 1
50 LOOP WHILE X <= 5

100 Y = 10
110 DO UNTIL Y = 0
120   PRINT Y
130   Y = Y - 1
140 LOOP
\end{lstlisting}

\subsection{GOSUB / RETURN}

Use subroutines to avoid repeating code.  \cmd{GOSUB} branches to a line
number and \cmd{RETURN} comes back:

\begin{lstlisting}[style=basiclisting]
10 GOSUB 1000
20 GOSUB 1000
30 END
1000 REM draw border subroutine
1010 PRINT "----------"
1020 RETURN
\end{lstlisting}

\subsection{GOTO}

\cmd{GOTO} unconditionally jumps to a line number.  Use it sparingly;
\cmd{GOSUB}/\cmd{RETURN} and structured loops are usually cleaner.

% ---------------------------------------------------------------------------
\section{Operators}

\begin{longtable}{>{\ttfamily}p{0.20\textwidth} p{0.22\textwidth} p{0.48\textwidth}}
\toprule
\normalfont\textbf{Operator} & \textbf{Type} & \textbf{Description} \\
\midrule
\endhead
\bottomrule
\endfoot
+  & Arithmetic & Addition \\[2pt]
-  & Arithmetic & Subtraction or unary negation \\[2pt]
*  & Arithmetic & Multiplication \\[2pt]
/  & Arithmetic & Division \\[2pt]
\^{}  & Arithmetic & Exponentiation (\texttt{2\^{}8} = 256) \\[4pt]
=  & Comparison & Equal to \\[2pt]
<  & Comparison & Less than \\[2pt]
>  & Comparison & Greater than \\[2pt]
<= & Comparison & Less than or equal to \\[2pt]
>= & Comparison & Greater than or equal to \\[2pt]
<> & Comparison & Not equal to \\[4pt]
AND & Logical   & Logical (bitwise) AND \\[2pt]
OR  & Logical   & Logical (bitwise) OR \\[2pt]
NOT & Logical   & Logical (bitwise) NOT \\[2pt]
EOR & Logical   & Exclusive OR \\[4pt]
<< & Bit        & Left shift (\texttt{LSHIFT}) \\[2pt]
>> & Bit        & Right shift (\texttt{RSHIFT}) \\[4pt]
BITSET & Bit    & Set a bit in an integer \\[2pt]
BITCLR & Bit    & Clear a bit in an integer \\[2pt]
BITTST(n,b) & Bit & Test bit \textit{b} of integer \textit{n} \\
\end{longtable}

% ---------------------------------------------------------------------------
\section{Built-in Functions}

\subsection{Numeric Functions}

\begin{longtable}{>{\ttfamily}p{0.28\textwidth} p{0.64\textwidth}}
\toprule
\normalfont\textbf{Function} & \textbf{Description} \\
\midrule
\endhead
\bottomrule
\endfoot
INT(x)   & Truncate to integer (towards zero) \\[2pt]
ABS(x)   & Absolute value \\[2pt]
SGN(x)   & Sign: $-1$, $0$, or $1$ \\[2pt]
SQR(x)   & Square root \\[2pt]
RND(1)   & Pseudo-random number in $[0, 1)$ \\[2pt]
LOG(x)   & Natural logarithm \\[2pt]
EXP(x)   & $e^x$ \\[2pt]
SIN(x)   & Sine (radians) \\[2pt]
COS(x)   & Cosine (radians) \\[2pt]
TAN(x)   & Tangent (radians) \\[2pt]
ATN(x)   & Arctangent (radians) \\[2pt]
PI       & The constant $\pi \approx 3.14159$ \\[2pt]
TWOPI    & The constant $2\pi \approx 6.28318$ \\[2pt]
MAX(a,b) & Larger of two values \\[2pt]
MIN(a,b) & Smaller of two values \\[2pt]
PEEK(addr) & Read a byte from memory address \textit{addr} \\[2pt]
DEEK(addr) & Read a 16-bit word from address \textit{addr} \\[2pt]
FRE(0)   & Free BASIC program memory in bytes \\[2pt]
POS(0)   & Current cursor column position \\
\end{longtable}

\subsection{String Functions}

\begin{longtable}{>{\ttfamily}p{0.34\textwidth} p{0.58\textwidth}}
\toprule
\normalfont\textbf{Function} & \textbf{Description} \\
\midrule
\endhead
\bottomrule
\endfoot
LEN(s\$)          & Number of characters in the string \\[2pt]
CHR\$(n)          & Character whose ASCII code is \textit{n} \\[2pt]
ASC(s\$)          & ASCII code of the first character \\[2pt]
STR\$(n)          & Convert number to string \\[2pt]
VAL(s\$)          & Convert string to number \\[2pt]
LEFT\$(s\$,n)     & Left \textit{n} characters \\[2pt]
RIGHT\$(s\$,n)    & Right \textit{n} characters \\[2pt]
MID\$(s\$,p,n)    & \textit{n} characters starting at position \textit{p} \\[2pt]
UCASE\$(s\$)      & Convert to uppercase \\[2pt]
LCASE\$(s\$)      & Convert to lowercase \\[2pt]
HEX\$(n)          & Hexadecimal string representation of \textit{n} \\[2pt]
BIN\$(n)          & Binary string representation of \textit{n} \\
\end{longtable}

% ---------------------------------------------------------------------------
\section{Memory Access}

NovaBASIC gives you direct read and write access to the 6502 address space.
This is useful for reading hardware registers, patching values, and interfacing
with assembly routines.

\begin{description}
  \item[\cmd{PEEK(addr)}] Read one byte (0--255) from address \textit{addr}.
  \item[\cmd{POKE addr, val}] Write one byte to address \textit{addr}.
  \item[\cmd{DEEK(addr)}] Read a 16-bit little-endian word from
    \textit{addr} and \textit{addr}+1.
  \item[\cmd{DOKE addr, val}] Write a 16-bit little-endian word.
  \item[\cmd{VARPTR(v)}] Return the memory address of variable \textit{v}.
    Useful when passing variable addresses to assembly routines.
\end{description}

\begin{lstlisting}[style=basiclisting]
10 REM read and print two bytes at $0300
20 PRINT PEEK(768); PEEK(769)
30 REM write a counter value
40 POKE 768, 42
50 PRINT PEEK(768)
\end{lstlisting}

\begin{warningbox}
  Writing to the wrong address can crash the virtual machine or corrupt the
  BASIC interpreter.  Know what you are writing to before you \cmd{POKE}
  into system areas.  See the memory map in Appendix~B for safe zones.
\end{warningbox}

% ---------------------------------------------------------------------------
\section{Style Tips}

\begin{tipbox}
\begin{itemize}
  \item \textbf{One idea per line.}  Statements can be chained with colons
    (\texttt{:}), but a single clear statement per line is easier to read
    and debug.
  \item \textbf{Group by function.}  Put constants and configuration near
    the top (lines~10--90), input routines in the 100s, game logic in the
    200s--500s, display routines in the 600s, cleanup in the 900s.
  \item \textbf{Name things meaningfully.}  \texttt{LIVES} is clearer than
    \texttt{L}; \texttt{SCORE} is clearer than \texttt{S1}.  NovaBASIC
    accepts long names.
  \item \textbf{Define constants up front.}
    \texttt{10 MAXLIVES=5 : STARTSPEED=2} at the top of the program means
    you tune the game by changing one line, not hunting through hundreds.
\end{itemize}
\end{tipbox}

% ---------------------------------------------------------------------------
\section{Putting It Together}

\begin{tryitbox}
Enter and run this program.  It builds an array of random values and prints
them in indexed form.
\begin{lstlisting}[style=basiclisting]
10 DIM A(5)
20 FOR I=1 TO 5
30   A(I) = INT(RND(1) * 100)
40 NEXT I
50 FOR I=1 TO 5
60   PRINT "A(";I;") = ";A(I)
70 NEXT I
\end{lstlisting}
\textbf{Expected:} Five lines printed, each showing an index and a random
integer between 0 and~99.  The values differ every time you \cmd{RUN}.
\end{tryitbox}

\noindent
Try modifying the program: change \texttt{100} to \texttt{10} to get
single-digit values, or change \texttt{5} to \texttt{20} to see a longer
array.  The structure stays the same; only the constants change.
