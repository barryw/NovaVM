% =============================================================================
% NovaBASIC v1.0 User Guide — Chapter 1: Welcome to NovaBASIC
% =============================================================================
\chapter{Welcome to NovaBASIC}

\epigraph{%
  Every great journey starts with a single command.
}{%
  \textit{Unknown programmer, circa~1982}%
}

\section{What NovaBASIC Is}

NovaBASIC v1.0 is a modernized 6502 BASIC interpreter for the e6502 virtual
computer.  It is derived from Lee Davison's \textit{Enhanced BASIC~2.22p5}
(EhBASIC), one of the cleanest and most complete open-source 6502 BASIC
implementations ever written.  NovaBASIC keeps the entire EhBASIC core --- every
numeric function, every string operation, the full \cmd{DO}/\cmd{LOOP} and
\cmd{FOR}/\cmd{NEXT} machinery --- and adds a hardware command layer that lets
you drive the e6502's graphics, sound, file system, and expansion memory
directly from BASIC.

The programming model is classic line-numbered BASIC.  You type a line with a
number at the front, press Return, and it is stored in the program.  You type
\cmd{RUN} and it executes from the lowest line number.  Simple, immediate, and
still deeply satisfying.

When NovaBASIC boots, the screen clears to a blue background with white text and
you see:

\begin{retrobox}
\texttt{NovaBASIC v1.0}\\
\texttt{Derived from EhBASIC 2.22p5}\\[0.4ex]
\texttt{xxxxx BASIC bytes free}
\end{retrobox}

\noindent
The number shown is the amount of free BASIC program memory.  The cursor waits
for your first command.

\begin{notebox}
  Boot sets the screen background to color~6 (blue) and the foreground text
  color to color~1 (white).  You can change these at any time with
  \cmd{COLOR~\textit{fg},\textit{bg}}.
\end{notebox}

% ---------------------------------------------------------------------------
\section{What You Can Do}

Here is what NovaBASIC puts at your fingertips:

\begin{itemize}
  \item Write and run classic line-numbered BASIC programs in the tradition
        of the Commodore~64, Apple~II, and BBC~Micro.
  \item Draw directly to a \textbf{320$\times$200 pixel bitmap} with
        \textbf{16 colors} using \cmd{PLOT}, \cmd{LINE}, \cmd{CIRCLE},
        \cmd{RECT}, \cmd{FILL}, and \cmd{PAINT}.
  \item Animate up to \textbf{16 hardware sprites}, each 16$\times$16 pixels
        and multicolor, with per-sprite shape, position, and flip control.
  \item Play \textbf{4-voice synthesized sound} with ADSR envelopes and five
        selectable waveforms --- square, sawtooth, triangle, sine, and noise.
  \item Save and load programs and data files with \cmd{SAVE}, \cmd{LOAD},
        \cmd{DIR}, and \cmd{DEL}.
  \item Access \textbf{512\,KB of banked expansion memory} for large data
        sets, tile maps, sprite sheets, or music sequences.
  \item Drop into \textbf{6502 assembly language} via \cmd{CALL} and
        \cmd{POKE}/\cmd{PEEK} for performance-critical inner loops.
\end{itemize}

% ---------------------------------------------------------------------------
\section{How to Read This Guide}

This manual is organized so each chapter builds on the previous one.  You do
not need to read it straight through --- jump to whichever section you need ---
but if you are new to NovaBASIC the order makes sense:

\begin{description}
  \item[Chapters~1--3] Getting started, the edit-run workflow, and the core
    language: variables, arrays, control flow, operators, and built-in
    functions.
  \item[Chapters~4--5] Graphics and sprites; sound and music.  These two
    chapters cover everything you need to build a real game or demo.
  \item[Chapters~6--7] Expansion memory and low-level access.  Read these
    when your projects outgrow the built-in 64\,KB address space or when
    you want to call hand-written assembly routines.
  \item[Appendices] Complete command reference, memory map, hardware
    register summary, error codes, and system limits.
\end{description}

Cross-references appear as ``see Section~X.Y'' or ``see Appendix~A.''  Every
command name is typeset in \cmd{this style} throughout the text.

% ---------------------------------------------------------------------------
\section{Your First Program}

Let us get something on screen right now.  Type each line exactly as shown,
pressing~Return after each one, then type \cmd{RUN} and press Return.

\begin{tryitbox}
\begin{lstlisting}[style=basiclisting,numbers=none]
10 PRINT "HELLO, NOVABASIC!"
20 FOR I=1 TO 5
30   PRINT "COUNT: ";I
40 NEXT I
RUN
\end{lstlisting}
\textbf{Expected output:}\\[0.4ex]
\texttt{HELLO, NOVABASIC!}\\
\texttt{COUNT: 1}\\
\texttt{COUNT: 2}\\
\texttt{COUNT: 3}\\
\texttt{COUNT: 4}\\
\texttt{COUNT: 5}
\end{tryitbox}

\noindent
Six lines of output, then the cursor returns to the \texttt{Ready} prompt.
Congratulations --- you have just run your first NovaBASIC program.
